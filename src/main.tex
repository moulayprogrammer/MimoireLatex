%! Author = moulay
%! Date = 10/21/19

% Preamble
\documentclass[english,a4,12pt]{report}

% Packages
\usepackage{amsmath}
\usepackage[utf8]{inputenc}
\usepackage{xcolor}
\usepackage[T1]{fontenc}
\usepackage{arabtex}
\usepackage{sectsty}
\usepackage[cyr]{aeguill}
\usepackage{rotating}
\usepackage{multirow}
\usepackage{tabulary}
\usepackage{tabularht}
\usepackage{acronym}
\usepackage{fancyhdr}
\usepackage{lscape}
\usepackage{amssymb}
\usepackage{pifont}
\usepackage[most]{tcolorbox}
\usepackage{slashbox}
\usepackage{multido}
\usepackage{caption}
\usepackage{graphicx,wrapfig,lipsum}
\usepackage{enumitem}
\usepackage {fancybox}
\usepackage{array,tabularx}
\usepackage{colortbl}
\usepackage{float}
\usepackage[noend]{algorithmic}
\usepackage[linesnumbered,ruled,vlined,boxed,commentsnumbered]{algorithm2e}
%\usepackage{algpseudocode}
\DeclareGraphicsExtensions{.jpg,.pdf,.PNG,.gif}
\usepackage[pdftex,colorlinks=true,linkcolor=black,citecolor=black,urlcolor=black]{hyperref}
\usepackage{pgfplots}
\usepackage{pgfplotstable}
\usepackage{subcaption}
\usepackage[tikz]{bclogo}
\pgfplotsset{grid style={dashed,gray}}
\pgfplotsset{minor grid style={dotted,green!50!black}}
\pgfplotsset{major grid style={dotted,green!50!black}}
\usepackage{anysize}
\marginsize{30mm}{20mm}{15mm}{15mm}
\newcolumntype{Y}{>{\raggedleft\arraybackslash}X}
\renewcommand{\baselinestretch}{1.5}
\newcolumntype{P}[1]{>{\raggedright}p{#1}}
\newcolumntype{M}[1]{>{\raggedright}m{#1}}%%declaration de page de garde
\newcommand*\rfrac[2]{{}^{#1}\!/_{#2}}
\setcounter{secnumdepth}{3}
\setcounter{tocdepth}{3}
\newcommand{\cmark}{\ding{51}}%
\newcommand{\xmark}{\ding{59}}%

% Document
\begin{document}
    \sloppy
    \begin{titlepage}
        \renewcommand{\baselinestretch}{1}
        \begin{center}
            \begin{RLtext}
                AljmhwryT AljzA'iryT AldymqrA.tyT Al^s`byT
            \end{RLtext}
            {PEOPLE'S DEMOCRATIC REPUBLIC OF ALGERIA}
            \begin{RLtext}
            {wzArT Alt`lym Al`Aly w Alb.h_t Al`lmy}
            \end{RLtext}
            {DEPARTMENT OF HIGHER EDUCATION AND SCIENTIFIC RESEARCH}
            \begin{RLtext}
                jAm`T mus.taf_A s.tmbOly bim`skar
            \end{RLtext}
            {UNIVERSITY OF MUSTAPHA STAMBOULI MASCARA}
            \begin{figure}[h]
            	\centering
            		\includegraphics[width=4cm]{figurs/logouniv.jpeg}
            \end{figure}
            \begin{RLtext}
                klyT Al`lwm AldaqyqaT
            \end{RLtext}
            {FACULTY OF EXACT SCIENCES}\\
            {COMPUTER SCIENCE DEPARTMENT}\\
            \textsc{\textbf{ \large Master memory} }
        \end{center}
        {\bfseries field :} Mathematics and Computer Science\\
        {\bfseries Faculty  {\hspace*{0.34cm}} :} Computer Science\\
        {\bfseries Option {\hspace*{0.26cm}} :} Networks and distributed system
        \vspace{0.5cm}
        \begin{center}
            \textsc{\textbf{ \large Realized by :}}\\
            {\bfseries Elkaim Moulay Abdellah }{\hspace*{5cm}}{\bfseries Mazouz  Nail}   \\
            \vspace{1cm}
            \Large {\bfseries Theme }
        \end{center}
        \begin{center}
            \hrule width 460pt
            \bigskip
            \Large  \centering \textbf{ \textsc{ ................. } }
            \bigskip
            \hrule width 460pt
        \end{center}
        \raggedright
        \bigskip
        \vspace{1cm}
        \textit{\bfseries Proposed by : Dr. Madber Hayat }
        \vspace{2cm}
        \begin{center}
            \textit{\bfseries College year 2019/2020}
        \end{center}
    \end{titlepage}

    %%Dedicaces%%%%%%%%%%%%%%%%%%%%%%%%%%%%%%%%%%%%%%%%%%%%%%%%%%%%%%%%%%%%%%%%%%%%%%%%%%%%%%%%%%%%%%%%%%%%%%%%%%%%%%%%%%%%%%%%

    \thispagestyle{empty}
    \begin{figure}[h]
    	\centering
    		\includegraphics[width=14cm]{figurs/B.PNG}
    \end{figure}
    \newpage
    \thispagestyle{empty}
    \clearpage
    \newpage
    \pagenumbering{roman}
    \begin{center}
        \textbf{\\}
        \textbf{\\}
        \textbf{\huge \textsc{\itshape Dedications}}\\
        \textbf{\\}
        \textbf{\\}
        \begin{flushleft}
            \textsf{\qquad I dedicate this work to ............. }\\
        \end{flushleft}
    \end{center}
    \begin{flushright}
        \textbf{\textsc{\itshape Elkaim Moulay Abdellah}}
    \end{flushright}
    \newpage
    \begin{center}
        \textbf{\\}
        \textbf{\\}
        \textbf{\huge \textsc{\itshape Dedications}}\\
        \textbf{\\}
        \textbf{\\}

        \begin{flushleft}
            \textsf{\qquad I dedicate this work to ...................... }
        \end{flushleft}

        \normalsize{\itshape .....}
        \textbf{\\}
        \textbf{\\}
    \end{center}
    \begin{flushright}
        \textbf{\textsc{\itshape Mazouz Nail}}
    \end{flushright}

    %%Remerciement%%%%%%%%%%%%%%%%%%%%%%%%%%%%%%%%%%%%%%%%%%%%%%%%%%%%%%%%%%%%%%%%%%%%%%%%%%%%%%%%%%%%%%%%%%%%%%%%%%%%%%%%%%%%%%%%
    \newpage
    \begin{center}
        %\thispagestyle{myheadings}
        %\markboth{droite}{ }
        \textbf{\huge \textsc{\itshape thanks}}
    \end{center}
    \textsf{\qquad We thank .......... }

    %%Abstract %%%%%%%%%%%%%%%%%%%%%%%%%%%%%%%%%%%%%%%%%%%%%%%%%%%%%%%%%%%%%%%%%%%%%%%%%%%%%%%%%%%%%%%%%%%%%%%%%%%%%%%%%%%%%%%%
    \newpage
    \markboth{droite}{ }
    \begin{center}
    \textbf{\huge \textsc{\itshape \textit Abstract}}\\
    \end{center}
    \qquad Image segmentation by active contours is a method to locate the boundaries of objects,
    using a curve that deforms by minimization of energies, which take a considerable time for execution.
    In this paper, we propose a parallel approach to reduce computational time by using the GVF Snake.
    The proposed approach has two main steps an initialization using k-means, and a parallel execution of the contour.
    Our approach was tested using a different image: the obtained results were encouraging.

    %%resume %%%%%%%%%%%%%%%%%%%%%%%%%%%%%%%%%%%%%%%%%%%%%%%%%%%%%%%%%%%%%%%%%%%%%%%%%%%%%%%%%%%%%%%%%%%%%%%%%%%%%%%%%%%%%%%%
    \newpage
    \newcommand{\enteteresume}{\markboth{Resume}{Resume}} %il faut ajouter les commandes
    \begin{center}
    \textbf{\huge \textsc{\itshape \textit Résumé}}\\
    \end{center}
    \qquad ...............................

    % La table des matieres%%%%%%%%%%%%%%%%%%%%%%%%%%%%%%%%%%%%%%%%%%%%%%%%%%%%%%%%%%%%%%%%%%%%%%%%%%%%%%%%%%%%%%%%%%%%%%%%%%%%%%%%%%%%%%%%%%%%%%%%%%%%%%%%%%%%%
    \tableofcontents
    \listoffigures
    \listoftables

    %% introduction generale%%%%%%%%%%%%%%%%%%%%%%%%%%%%%%%%%%%%%%%%%%%%%%%%%%%%%%%%%%%%%%%%%%%%%%%%%%%%%%%%%%%%%%%%%%%%%%%%%%%%%%%%%%%%%%%%%%%%%%%%%%%%%%%%%%%%%%%%%%%
    \chapter*{General Introduction}
    \addcontentsline{toc}{chapter}{Introduction générale}
    \setcounter{page}{1}
    \lhead{}
    \cfoot{\bfseries \thepage}
    \rhead{Introduction générale}
    \pagenumbering{arabic}

    Image segmentation is one of the most important stages in digital image analysis,
    which its objective is to separate the different shapes, structures, or objects that are
    located inside the image. There are different techniques used for the segmentation of pixels
    of interest from the image. Active contour is one of the active models in segmentation techniques,
    which use energy constraints and forces in the image for separation of the interest region.
    Although the computational time of its algorithm is commonly considered as a drawback to the use
    of it in some applications, a lot of research proposed methods to reduce the computational time.\\
    In this work, we propose a parallel implementation approach for the active contour that can perform
    efficiently and reduce the computational time by using an improved model of the classic active contour
    the GVF Snake. The approach we are proposing involves two main steps:  an initialization using the
    k-means algorithm and parallel execution of the GVF snake.\\
    The manuscript is organized as follows. Section (\ref{ch:generalities-on-image-processing}) is devoted to
    image generalizations by citing some definitions and methods of image segmentation and image enhancement.
    Section (\ref{ch:active-contours-:-gvf-snakes}) concentrates mainly on active contour and their
    limit and its improved descendant model GVF snake, which we use in our approach. Sections (\ref{ch:proposed-approach}) describes
    the study and implementation of  the proposed approach. We detail the different steps we have gone through,
    and evaluate the accuracy and applicability of the proposed approach via an experiment. A conclusion is given
    in Section (4).

    %% chapitre 1%%%%%%%%%%%%%%%%%%%%%%%%%%%%%%%%%%%%%%%%%%%%%%%%%%%%%%%%%%%%%%%%%%%%%%%%%%%%%%%%%%%%%%%%%%%%%%%%%%%%%%%%%%%%%%%%%%%%%%%%%%%%%%%%%%%%%%%%%%%%%%%%%%%%%
    %! Author = moulay
%! Date = 10/21/19

% Preamble
\documentclass[english,a4,12pt]{report}

% Packages
\usepackage{amsmath}
\usepackage[utf8]{inputenc}
\usepackage{xcolor}
\usepackage[T1]{fontenc}
\usepackage{arabtex}
\usepackage{sectsty}
\usepackage[cyr]{aeguill}
\usepackage{rotating}
\usepackage{multirow}
\usepackage{tabulary}
\usepackage{tabularht}
\usepackage{acronym}
\usepackage{fancyhdr}
\usepackage{lscape}
\usepackage{amssymb}
\usepackage{pifont}
\usepackage[most]{tcolorbox}
\usepackage{slashbox}
\usepackage{multido}
\usepackage{caption}
\usepackage{graphicx,wrapfig,lipsum}
\usepackage{enumitem}
\usepackage {fancybox}
\usepackage{array,tabularx}
\usepackage{colortbl}
\usepackage[noend]{algorithmic}
\usepackage[linesnumbered,ruled,vlined,boxed,commentsnumbered]{algorithm2e}
\DeclareGraphicsExtensions{.jpg,.pdf,.PNG,.gif}
\usepackage[pdftex,colorlinks=true,linkcolor=black,citecolor=black,urlcolor=black]{hyperref}
\usepackage{pgfplots}
\usepackage{pgfplotstable}
\usepackage{subcaption}
\usepackage[tikz]{bclogo}
\pgfplotsset{grid style={dashed,gray}}
\pgfplotsset{minor grid style={dotted,green!50!black}}
\pgfplotsset{major grid style={dotted,green!50!black}}
\usepackage{anysize}
\marginsize{30mm}{20mm}{15mm}{15mm}
\newcolumntype{Y}{>{\raggedleft\arraybackslash}X}
\renewcommand{\baselinestretch}{1.5}
\newcolumntype{P}[1]{>{\raggedright}p{#1}}
\newcolumntype{M}[1]{>{\raggedright}m{#1}}%%declaration de page de garde
\newcommand*\rfrac[2]{{}^{#1}\!/_{#2}}
\setcounter{secnumdepth}{3}
\setcounter{tocdepth}{3}
\newcommand{\cmark}{\ding{51}}%
\newcommand{\xmark}{\ding{59}}%

% Document
\begin{document}
    \sloppy
    \begin{titlepage}
        \renewcommand{\baselinestretch}{1}
        \begin{center}
            \begin{RLtext}
                AljmhwryT AljzA'iryT AldymqrA.tyT Al^s`byT
            \end{RLtext}
            {PEOPLE'S DEMOCRATIC REPUBLIC OF ALGERIA}
            \begin{RLtext}
            {wzArT Alt`lym Al`Aly w Alb.h_t Al`lmy}
            \end{RLtext}
            {DEPARTMENT OF HIGHER EDUCATION AND SCIENTIFIC RESEARCH}
            \begin{RLtext}
                jAm`T mus.taf_A s.tmbOly bim`skar
            \end{RLtext}
            {UNIVERSITY OF MUSTAPHA STAMBOULI MASCARA}
            \begin{figure}[h]
            	\centering
            		\includegraphics[width=4cm]{figurs/logouniv.jpeg}
            \end{figure}
            \begin{RLtext}
                klyT Al`lwm AldaqyqaT
            \end{RLtext}
            {FACULTY OF EXACT SCIENCES}\\
            {COMPUTER SCIENCE DEPARTMENT}\\
            \textsc{\textbf{ \large Master memory} }
        \end{center}
        {\bfseries field :} Mathematics and Computer Science\\
        {\bfseries Faculty  {\hspace*{0.34cm}} :} Computer Science\\
        {\bfseries Option {\hspace*{0.26cm}} :} Networks and distributed system
        \vspace{0.5cm}
        \begin{center}
            \textsc{\textbf{ \large Realized by :}}\\
            {\bfseries Elkaim Moulay Abdellah }{\hspace*{5cm}}{\bfseries Mazouz  Nail}   \\
            \vspace{1cm}
            \Large {\bfseries Theme }
        \end{center}
        \begin{center}
            \hrule width 460pt
            \bigskip
            \Large  \centering \textbf{ \textsc{ ................. } }
            \bigskip
            \hrule width 460pt
        \end{center}
        \raggedright
        \bigskip
        \vspace{1cm}
        \textit{\bfseries Proposed by : Dr. Madber Hayat }
        \vspace{2cm}
        \begin{center}
            \textit{\bfseries College year 2019/2020}
        \end{center}
    \end{titlepage}

    %%Dedicaces%%%%%%%%%%%%%%%%%%%%%%%%%%%%%%%%%%%%%%%%%%%%%%%%%%%%%%%%%%%%%%%%%%%%%%%%%%%%%%%%%%%%%%%%%%%%%%%%%%%%%%%%%%%%%%%%

    \thispagestyle{empty}
    \begin{figure}[h]
    	\centering
    		\includegraphics[width=14cm]{figurs/B.PNG}
    \end{figure}
    \newpage
    \thispagestyle{empty}
    \clearpage
    \newpage
    \pagenumbering{roman}
    \begin{center}
        \textbf{\\}
        \textbf{\\}
        \textbf{\huge \textsc{\itshape Dedications}}\\
        \textbf{\\}
        \textbf{\\}
        \begin{flushleft}
            \textsf{\qquad I dedicate this work to ............. }\\
        \end{flushleft}
    \end{center}
    \begin{flushright}
        \textbf{\textsc{\itshape Elkaim Moulay Abdellah}}
    \end{flushright}
    \newpage
    \begin{center}
        \textbf{\\}
        \textbf{\\}
        \textbf{\huge \textsc{\itshape Dedications}}\\
        \textbf{\\}
        \textbf{\\}

        \begin{flushleft}
            \textsf{\qquad I dedicate this work to ...................... }
        \end{flushleft}

        \normalsize{\itshape .....}
        \textbf{\\}
        \textbf{\\}
    \end{center}
    \begin{flushright}
        \textbf{\textsc{\itshape Mazouz Nail}}
    \end{flushright}

    %%Remerciement%%%%%%%%%%%%%%%%%%%%%%%%%%%%%%%%%%%%%%%%%%%%%%%%%%%%%%%%%%%%%%%%%%%%%%%%%%%%%%%%%%%%%%%%%%%%%%%%%%%%%%%%%%%%%%%%
    \newpage
    \begin{center}
        %\thispagestyle{myheadings}
        %\markboth{droite}{ }
        \textbf{\huge \textsc{\itshape thanks}}
    \end{center}
    \textsf{\qquad We thank .......... }

    %%Abstract %%%%%%%%%%%%%%%%%%%%%%%%%%%%%%%%%%%%%%%%%%%%%%%%%%%%%%%%%%%%%%%%%%%%%%%%%%%%%%%%%%%%%%%%%%%%%%%%%%%%%%%%%%%%%%%%
    \newpage
    \markboth{droite}{ }
    \begin{center}
    \textbf{\huge \textsc{\itshape \textit Abstract}}\\
    \end{center}
    \qquad ...........................

    %%resume %%%%%%%%%%%%%%%%%%%%%%%%%%%%%%%%%%%%%%%%%%%%%%%%%%%%%%%%%%%%%%%%%%%%%%%%%%%%%%%%%%%%%%%%%%%%%%%%%%%%%%%%%%%%%%%%
    \newpage
    \newcommand{\enteteresume}{\markboth{Resume}{Resume}} %il faut ajouter les commandes
    \begin{center}
    \textbf{\huge \textsc{\itshape \textit Résumé}}\\
    \end{center}
    \qquad ...............................

    % La table des matieres%%%%%%%%%%%%%%%%%%%%%%%%%%%%%%%%%%%%%%%%%%%%%%%%%%%%%%%%%%%%%%%%%%%%%%%%%%%%%%%%%%%%%%%%%%%%%%%%%%%%%%%%%%%%%%%%%%%%%%%%%%%%%%%%%%%%%
    \tableofcontents
    \listoffigures
    \listoftables

    %% introduction generale%%%%%%%%%%%%%%%%%%%%%%%%%%%%%%%%%%%%%%%%%%%%%%%%%%%%%%%%%%%%%%%%%%%%%%%%%%%%%%%%%%%%%%%%%%%%%%%%%%%%%%%%%%%%%%%%%%%%%%%%%%%%%%%%%%%%%%%%%%%
    \chapter*{General Introduction}
    \addcontentsline{toc}{chapter}{Introduction générale}
    \setcounter{page}{1}
    \lhead{}
    \cfoot{\bfseries \thepage}
    \rhead{Introduction générale}
    \pagenumbering{arabic}

    ..................................................

    %% chapitre 1%%%%%%%%%%%%%%%%%%%%%%%%%%%%%%%%%%%%%%%%%%%%%%%%%%%%%%%%%%%%%%%%%%%%%%%%%%%%%%%%%%%%%%%%%%%%%%%%%%%%%%%%%%%%%%%%%%%%%%%%%%%%%%%%%%%%%%%%%%%%%%%%%%%%%
    %! Author = moulay
%! Date = 10/21/19

% Preamble
\documentclass[english,a4,12pt]{report}

% Packages
\usepackage{amsmath}
\usepackage[utf8]{inputenc}
\usepackage{xcolor}
\usepackage[T1]{fontenc}
\usepackage{arabtex}
\usepackage{sectsty}
\usepackage[cyr]{aeguill}
\usepackage{rotating}
\usepackage{multirow}
\usepackage{tabulary}
\usepackage{tabularht}
\usepackage{acronym}
\usepackage{fancyhdr}
\usepackage{lscape}
\usepackage{amssymb}
\usepackage{pifont}
\usepackage[most]{tcolorbox}
\usepackage{slashbox}
\usepackage{multido}
\usepackage{caption}
\usepackage{graphicx,wrapfig,lipsum}
\usepackage{enumitem}
\usepackage {fancybox}
\usepackage{array,tabularx}
\usepackage{colortbl}
\usepackage[noend]{algorithmic}
\usepackage[linesnumbered,ruled,vlined,boxed,commentsnumbered]{algorithm2e}
\DeclareGraphicsExtensions{.jpg,.pdf,.PNG,.gif}
\usepackage[pdftex,colorlinks=true,linkcolor=black,citecolor=black,urlcolor=black]{hyperref}
\usepackage{pgfplots}
\usepackage{pgfplotstable}
\usepackage{subcaption}
\usepackage[tikz]{bclogo}
\pgfplotsset{grid style={dashed,gray}}
\pgfplotsset{minor grid style={dotted,green!50!black}}
\pgfplotsset{major grid style={dotted,green!50!black}}
\usepackage{anysize}
\marginsize{30mm}{20mm}{15mm}{15mm}
\newcolumntype{Y}{>{\raggedleft\arraybackslash}X}
\renewcommand{\baselinestretch}{1.5}
\newcolumntype{P}[1]{>{\raggedright}p{#1}}
\newcolumntype{M}[1]{>{\raggedright}m{#1}}%%declaration de page de garde
\newcommand*\rfrac[2]{{}^{#1}\!/_{#2}}
\setcounter{secnumdepth}{3}
\setcounter{tocdepth}{3}
\newcommand{\cmark}{\ding{51}}%
\newcommand{\xmark}{\ding{59}}%

% Document
\begin{document}
    \sloppy
    \begin{titlepage}
        \renewcommand{\baselinestretch}{1}
        \begin{center}
            \begin{RLtext}
                AljmhwryT AljzA'iryT AldymqrA.tyT Al^s`byT
            \end{RLtext}
            {PEOPLE'S DEMOCRATIC REPUBLIC OF ALGERIA}
            \begin{RLtext}
            {wzArT Alt`lym Al`Aly w Alb.h_t Al`lmy}
            \end{RLtext}
            {DEPARTMENT OF HIGHER EDUCATION AND SCIENTIFIC RESEARCH}
            \begin{RLtext}
                jAm`T mus.taf_A s.tmbOly bim`skar
            \end{RLtext}
            {UNIVERSITY OF MUSTAPHA STAMBOULI MASCARA}
            \begin{figure}[h]
            	\centering
            		\includegraphics[width=4cm]{figurs/logouniv.jpeg}
            \end{figure}
            \begin{RLtext}
                klyT Al`lwm AldaqyqaT
            \end{RLtext}
            {FACULTY OF EXACT SCIENCES}\\
            {COMPUTER SCIENCE DEPARTMENT}\\
            \textsc{\textbf{ \large Master memory} }
        \end{center}
        {\bfseries field :} Mathematics and Computer Science\\
        {\bfseries Faculty  {\hspace*{0.34cm}} :} Computer Science\\
        {\bfseries Option {\hspace*{0.26cm}} :} Networks and distributed system
        \vspace{0.5cm}
        \begin{center}
            \textsc{\textbf{ \large Realized by :}}\\
            {\bfseries Elkaim Moulay Abdellah }{\hspace*{5cm}}{\bfseries Mazouz  Nail}   \\
            \vspace{1cm}
            \Large {\bfseries Theme }
        \end{center}
        \begin{center}
            \hrule width 460pt
            \bigskip
            \Large  \centering \textbf{ \textsc{ ................. } }
            \bigskip
            \hrule width 460pt
        \end{center}
        \raggedright
        \bigskip
        \vspace{1cm}
        \textit{\bfseries Proposed by : Dr. Madber Hayat }
        \vspace{2cm}
        \begin{center}
            \textit{\bfseries College year 2019/2020}
        \end{center}
    \end{titlepage}

    %%Dedicaces%%%%%%%%%%%%%%%%%%%%%%%%%%%%%%%%%%%%%%%%%%%%%%%%%%%%%%%%%%%%%%%%%%%%%%%%%%%%%%%%%%%%%%%%%%%%%%%%%%%%%%%%%%%%%%%%

    \thispagestyle{empty}
    \begin{figure}[h]
    	\centering
    		\includegraphics[width=14cm]{figurs/B.PNG}
    \end{figure}
    \newpage
    \thispagestyle{empty}
    \clearpage
    \newpage
    \pagenumbering{roman}
    \begin{center}
        \textbf{\\}
        \textbf{\\}
        \textbf{\huge \textsc{\itshape Dedications}}\\
        \textbf{\\}
        \textbf{\\}
        \begin{flushleft}
            \textsf{\qquad I dedicate this work to ............. }\\
        \end{flushleft}
    \end{center}
    \begin{flushright}
        \textbf{\textsc{\itshape Elkaim Moulay Abdellah}}
    \end{flushright}
    \newpage
    \begin{center}
        \textbf{\\}
        \textbf{\\}
        \textbf{\huge \textsc{\itshape Dedications}}\\
        \textbf{\\}
        \textbf{\\}

        \begin{flushleft}
            \textsf{\qquad I dedicate this work to ...................... }
        \end{flushleft}

        \normalsize{\itshape .....}
        \textbf{\\}
        \textbf{\\}
    \end{center}
    \begin{flushright}
        \textbf{\textsc{\itshape Mazouz Nail}}
    \end{flushright}

    %%Remerciement%%%%%%%%%%%%%%%%%%%%%%%%%%%%%%%%%%%%%%%%%%%%%%%%%%%%%%%%%%%%%%%%%%%%%%%%%%%%%%%%%%%%%%%%%%%%%%%%%%%%%%%%%%%%%%%%
    \newpage
    \begin{center}
        %\thispagestyle{myheadings}
        %\markboth{droite}{ }
        \textbf{\huge \textsc{\itshape thanks}}
    \end{center}
    \textsf{\qquad We thank .......... }

    %%Abstract %%%%%%%%%%%%%%%%%%%%%%%%%%%%%%%%%%%%%%%%%%%%%%%%%%%%%%%%%%%%%%%%%%%%%%%%%%%%%%%%%%%%%%%%%%%%%%%%%%%%%%%%%%%%%%%%
    \newpage
    \markboth{droite}{ }
    \begin{center}
    \textbf{\huge \textsc{\itshape \textit Abstract}}\\
    \end{center}
    \qquad ...........................

    %%resume %%%%%%%%%%%%%%%%%%%%%%%%%%%%%%%%%%%%%%%%%%%%%%%%%%%%%%%%%%%%%%%%%%%%%%%%%%%%%%%%%%%%%%%%%%%%%%%%%%%%%%%%%%%%%%%%
    \newpage
    \newcommand{\enteteresume}{\markboth{Resume}{Resume}} %il faut ajouter les commandes
    \begin{center}
    \textbf{\huge \textsc{\itshape \textit Résumé}}\\
    \end{center}
    \qquad ...............................

    % La table des matieres%%%%%%%%%%%%%%%%%%%%%%%%%%%%%%%%%%%%%%%%%%%%%%%%%%%%%%%%%%%%%%%%%%%%%%%%%%%%%%%%%%%%%%%%%%%%%%%%%%%%%%%%%%%%%%%%%%%%%%%%%%%%%%%%%%%%%
    \tableofcontents
    \listoffigures
    \listoftables

    %% introduction generale%%%%%%%%%%%%%%%%%%%%%%%%%%%%%%%%%%%%%%%%%%%%%%%%%%%%%%%%%%%%%%%%%%%%%%%%%%%%%%%%%%%%%%%%%%%%%%%%%%%%%%%%%%%%%%%%%%%%%%%%%%%%%%%%%%%%%%%%%%%
    \chapter*{General Introduction}
    \addcontentsline{toc}{chapter}{Introduction générale}
    \setcounter{page}{1}
    \lhead{}
    \cfoot{\bfseries \thepage}
    \rhead{Introduction générale}
    \pagenumbering{arabic}

    ..................................................

    %% chapitre 1%%%%%%%%%%%%%%%%%%%%%%%%%%%%%%%%%%%%%%%%%%%%%%%%%%%%%%%%%%%%%%%%%%%%%%%%%%%%%%%%%%%%%%%%%%%%%%%%%%%%%%%%%%%%%%%%%%%%%%%%%%%%%%%%%%%%%%%%%%%%%%%%%%%%%
    %! Author = moulay
%! Date = 10/21/19

% Preamble
\documentclass[english,a4,12pt]{report}

% Packages
\usepackage{amsmath}
\usepackage[utf8]{inputenc}
\usepackage{xcolor}
\usepackage[T1]{fontenc}
\usepackage{arabtex}
\usepackage{sectsty}
\usepackage[cyr]{aeguill}
\usepackage{rotating}
\usepackage{multirow}
\usepackage{tabulary}
\usepackage{tabularht}
\usepackage{acronym}
\usepackage{fancyhdr}
\usepackage{lscape}
\usepackage{amssymb}
\usepackage{pifont}
\usepackage[most]{tcolorbox}
\usepackage{slashbox}
\usepackage{multido}
\usepackage{caption}
\usepackage{graphicx,wrapfig,lipsum}
\usepackage{enumitem}
\usepackage {fancybox}
\usepackage{array,tabularx}
\usepackage{colortbl}
\usepackage[noend]{algorithmic}
\usepackage[linesnumbered,ruled,vlined,boxed,commentsnumbered]{algorithm2e}
\DeclareGraphicsExtensions{.jpg,.pdf,.PNG,.gif}
\usepackage[pdftex,colorlinks=true,linkcolor=black,citecolor=black,urlcolor=black]{hyperref}
\usepackage{pgfplots}
\usepackage{pgfplotstable}
\usepackage{subcaption}
\usepackage[tikz]{bclogo}
\pgfplotsset{grid style={dashed,gray}}
\pgfplotsset{minor grid style={dotted,green!50!black}}
\pgfplotsset{major grid style={dotted,green!50!black}}
\usepackage{anysize}
\marginsize{30mm}{20mm}{15mm}{15mm}
\newcolumntype{Y}{>{\raggedleft\arraybackslash}X}
\renewcommand{\baselinestretch}{1.5}
\newcolumntype{P}[1]{>{\raggedright}p{#1}}
\newcolumntype{M}[1]{>{\raggedright}m{#1}}%%declaration de page de garde
\newcommand*\rfrac[2]{{}^{#1}\!/_{#2}}
\setcounter{secnumdepth}{3}
\setcounter{tocdepth}{3}
\newcommand{\cmark}{\ding{51}}%
\newcommand{\xmark}{\ding{59}}%

% Document
\begin{document}
    \sloppy
    \begin{titlepage}
        \renewcommand{\baselinestretch}{1}
        \begin{center}
            \begin{RLtext}
                AljmhwryT AljzA'iryT AldymqrA.tyT Al^s`byT
            \end{RLtext}
            {PEOPLE'S DEMOCRATIC REPUBLIC OF ALGERIA}
            \begin{RLtext}
            {wzArT Alt`lym Al`Aly w Alb.h_t Al`lmy}
            \end{RLtext}
            {DEPARTMENT OF HIGHER EDUCATION AND SCIENTIFIC RESEARCH}
            \begin{RLtext}
                jAm`T mus.taf_A s.tmbOly bim`skar
            \end{RLtext}
            {UNIVERSITY OF MUSTAPHA STAMBOULI MASCARA}
            \begin{figure}[h]
            	\centering
            		\includegraphics[width=4cm]{figurs/logouniv.jpeg}
            \end{figure}
            \begin{RLtext}
                klyT Al`lwm AldaqyqaT
            \end{RLtext}
            {FACULTY OF EXACT SCIENCES}\\
            {COMPUTER SCIENCE DEPARTMENT}\\
            \textsc{\textbf{ \large Master memory} }
        \end{center}
        {\bfseries field :} Mathematics and Computer Science\\
        {\bfseries Faculty  {\hspace*{0.34cm}} :} Computer Science\\
        {\bfseries Option {\hspace*{0.26cm}} :} Networks and distributed system
        \vspace{0.5cm}
        \begin{center}
            \textsc{\textbf{ \large Realized by :}}\\
            {\bfseries Elkaim Moulay Abdellah }{\hspace*{5cm}}{\bfseries Mazouz  Nail}   \\
            \vspace{1cm}
            \Large {\bfseries Theme }
        \end{center}
        \begin{center}
            \hrule width 460pt
            \bigskip
            \Large  \centering \textbf{ \textsc{ ................. } }
            \bigskip
            \hrule width 460pt
        \end{center}
        \raggedright
        \bigskip
        \vspace{1cm}
        \textit{\bfseries Proposed by : Dr. Madber Hayat }
        \vspace{2cm}
        \begin{center}
            \textit{\bfseries College year 2019/2020}
        \end{center}
    \end{titlepage}

    %%Dedicaces%%%%%%%%%%%%%%%%%%%%%%%%%%%%%%%%%%%%%%%%%%%%%%%%%%%%%%%%%%%%%%%%%%%%%%%%%%%%%%%%%%%%%%%%%%%%%%%%%%%%%%%%%%%%%%%%

    \thispagestyle{empty}
    \begin{figure}[h]
    	\centering
    		\includegraphics[width=14cm]{figurs/B.PNG}
    \end{figure}
    \newpage
    \thispagestyle{empty}
    \clearpage
    \newpage
    \pagenumbering{roman}
    \begin{center}
        \textbf{\\}
        \textbf{\\}
        \textbf{\huge \textsc{\itshape Dedications}}\\
        \textbf{\\}
        \textbf{\\}
        \begin{flushleft}
            \textsf{\qquad I dedicate this work to ............. }\\
        \end{flushleft}
    \end{center}
    \begin{flushright}
        \textbf{\textsc{\itshape Elkaim Moulay Abdellah}}
    \end{flushright}
    \newpage
    \begin{center}
        \textbf{\\}
        \textbf{\\}
        \textbf{\huge \textsc{\itshape Dedications}}\\
        \textbf{\\}
        \textbf{\\}

        \begin{flushleft}
            \textsf{\qquad I dedicate this work to ...................... }
        \end{flushleft}

        \normalsize{\itshape .....}
        \textbf{\\}
        \textbf{\\}
    \end{center}
    \begin{flushright}
        \textbf{\textsc{\itshape Mazouz Nail}}
    \end{flushright}

    %%Remerciement%%%%%%%%%%%%%%%%%%%%%%%%%%%%%%%%%%%%%%%%%%%%%%%%%%%%%%%%%%%%%%%%%%%%%%%%%%%%%%%%%%%%%%%%%%%%%%%%%%%%%%%%%%%%%%%%
    \newpage
    \begin{center}
        %\thispagestyle{myheadings}
        %\markboth{droite}{ }
        \textbf{\huge \textsc{\itshape thanks}}
    \end{center}
    \textsf{\qquad We thank .......... }

    %%Abstract %%%%%%%%%%%%%%%%%%%%%%%%%%%%%%%%%%%%%%%%%%%%%%%%%%%%%%%%%%%%%%%%%%%%%%%%%%%%%%%%%%%%%%%%%%%%%%%%%%%%%%%%%%%%%%%%
    \newpage
    \markboth{droite}{ }
    \begin{center}
    \textbf{\huge \textsc{\itshape \textit Abstract}}\\
    \end{center}
    \qquad ...........................

    %%resume %%%%%%%%%%%%%%%%%%%%%%%%%%%%%%%%%%%%%%%%%%%%%%%%%%%%%%%%%%%%%%%%%%%%%%%%%%%%%%%%%%%%%%%%%%%%%%%%%%%%%%%%%%%%%%%%
    \newpage
    \newcommand{\enteteresume}{\markboth{Resume}{Resume}} %il faut ajouter les commandes
    \begin{center}
    \textbf{\huge \textsc{\itshape \textit Résumé}}\\
    \end{center}
    \qquad ...............................

    % La table des matieres%%%%%%%%%%%%%%%%%%%%%%%%%%%%%%%%%%%%%%%%%%%%%%%%%%%%%%%%%%%%%%%%%%%%%%%%%%%%%%%%%%%%%%%%%%%%%%%%%%%%%%%%%%%%%%%%%%%%%%%%%%%%%%%%%%%%%
    \tableofcontents
    \listoffigures
    \listoftables

    %% introduction generale%%%%%%%%%%%%%%%%%%%%%%%%%%%%%%%%%%%%%%%%%%%%%%%%%%%%%%%%%%%%%%%%%%%%%%%%%%%%%%%%%%%%%%%%%%%%%%%%%%%%%%%%%%%%%%%%%%%%%%%%%%%%%%%%%%%%%%%%%%%
    \chapter*{General Introduction}
    \addcontentsline{toc}{chapter}{Introduction générale}
    \setcounter{page}{1}
    \lhead{}
    \cfoot{\bfseries \thepage}
    \rhead{Introduction générale}
    \pagenumbering{arabic}

    ..................................................

    %% chapitre 1%%%%%%%%%%%%%%%%%%%%%%%%%%%%%%%%%%%%%%%%%%%%%%%%%%%%%%%%%%%%%%%%%%%%%%%%%%%%%%%%%%%%%%%%%%%%%%%%%%%%%%%%%%%%%%%%%%%%%%%%%%%%%%%%%%%%%%%%%%%%%%%%%%%%%
    \include{chapiter1/main}

    %% chapitre 2%%%%%%%%%%%%%%%%%%%%%%%%%%%%%%%%%%%%%%%%%%%%%%%%%%%%%%%%%%%%%%%%%%%%%%%%%%%%%%%%%%%%%%%%%%%%%%%%%%%%%%%%%%%%%%%%%%%%%%%%%%%%%%%%%%%%%%%%%%%%%%%%%%%%%
    \include{chapiter2/main}

    %% conclusion generale%%%%%%%%%%%%%%%%%%%%%%%%%%%%%%%%%%%%%%%%%%%%%%%%%%%%%%%%%%%%%%%%%%%%%%%%%%%%%%%%%%%%%%%%%%%%%%%%%%%%%%%%%%%%%%%%%%%%%%%%%%%%%%%%%%%%%%%%%%%
    \chapter*{General conclusion}
    \addcontentsline{toc}{chapter}{General conclusion}
    \lhead{}
    \cfoot{\bfseries \thepage}
    \rhead{General conclusion}
    \markboth{droite}{General conclusion}

    .............................................

    %%References%%%%%%%%%%%%%%%%%%%%%%%%%%%%%%%%%%%%%%%%%%%%%%%%%%%%%%%%%%%%%%%%%%%%%%%%%%%%%%%%%%%%%%%%%%%%%%%%%%%%%%%%%%%%%%%%%%%%%%%%%%%%%%
    \clearpage
    \pagestyle{fancy}
    \addcontentsline{toc}{chapter}{Bibliography}
    \begin{thebibliography}{99}
        \lhead{}
        \cfoot{\bfseries \thepage}
        \rhead{Bibliography}

        \bibitem[1]{1}
        H. Altaama,
        Application Mobile Guide,
        Mémoire de Master en Informatique,
        Université Abou Bakr Belkaid de Tlemcen,
        2016.
        \bibitem[2]{2}
        \url{http://generationmobiles.net/2014/11/les-differents-types-dapps-mobiles/}, [consulté le 14/03/2018].

    \end{thebibliography}

\end{document}

    %% chapitre 2%%%%%%%%%%%%%%%%%%%%%%%%%%%%%%%%%%%%%%%%%%%%%%%%%%%%%%%%%%%%%%%%%%%%%%%%%%%%%%%%%%%%%%%%%%%%%%%%%%%%%%%%%%%%%%%%%%%%%%%%%%%%%%%%%%%%%%%%%%%%%%%%%%%%%
    %! Author = moulay
%! Date = 10/21/19

% Preamble
\documentclass[english,a4,12pt]{report}

% Packages
\usepackage{amsmath}
\usepackage[utf8]{inputenc}
\usepackage{xcolor}
\usepackage[T1]{fontenc}
\usepackage{arabtex}
\usepackage{sectsty}
\usepackage[cyr]{aeguill}
\usepackage{rotating}
\usepackage{multirow}
\usepackage{tabulary}
\usepackage{tabularht}
\usepackage{acronym}
\usepackage{fancyhdr}
\usepackage{lscape}
\usepackage{amssymb}
\usepackage{pifont}
\usepackage[most]{tcolorbox}
\usepackage{slashbox}
\usepackage{multido}
\usepackage{caption}
\usepackage{graphicx,wrapfig,lipsum}
\usepackage{enumitem}
\usepackage {fancybox}
\usepackage{array,tabularx}
\usepackage{colortbl}
\usepackage[noend]{algorithmic}
\usepackage[linesnumbered,ruled,vlined,boxed,commentsnumbered]{algorithm2e}
\DeclareGraphicsExtensions{.jpg,.pdf,.PNG,.gif}
\usepackage[pdftex,colorlinks=true,linkcolor=black,citecolor=black,urlcolor=black]{hyperref}
\usepackage{pgfplots}
\usepackage{pgfplotstable}
\usepackage{subcaption}
\usepackage[tikz]{bclogo}
\pgfplotsset{grid style={dashed,gray}}
\pgfplotsset{minor grid style={dotted,green!50!black}}
\pgfplotsset{major grid style={dotted,green!50!black}}
\usepackage{anysize}
\marginsize{30mm}{20mm}{15mm}{15mm}
\newcolumntype{Y}{>{\raggedleft\arraybackslash}X}
\renewcommand{\baselinestretch}{1.5}
\newcolumntype{P}[1]{>{\raggedright}p{#1}}
\newcolumntype{M}[1]{>{\raggedright}m{#1}}%%declaration de page de garde
\newcommand*\rfrac[2]{{}^{#1}\!/_{#2}}
\setcounter{secnumdepth}{3}
\setcounter{tocdepth}{3}
\newcommand{\cmark}{\ding{51}}%
\newcommand{\xmark}{\ding{59}}%

% Document
\begin{document}
    \sloppy
    \begin{titlepage}
        \renewcommand{\baselinestretch}{1}
        \begin{center}
            \begin{RLtext}
                AljmhwryT AljzA'iryT AldymqrA.tyT Al^s`byT
            \end{RLtext}
            {PEOPLE'S DEMOCRATIC REPUBLIC OF ALGERIA}
            \begin{RLtext}
            {wzArT Alt`lym Al`Aly w Alb.h_t Al`lmy}
            \end{RLtext}
            {DEPARTMENT OF HIGHER EDUCATION AND SCIENTIFIC RESEARCH}
            \begin{RLtext}
                jAm`T mus.taf_A s.tmbOly bim`skar
            \end{RLtext}
            {UNIVERSITY OF MUSTAPHA STAMBOULI MASCARA}
            \begin{figure}[h]
            	\centering
            		\includegraphics[width=4cm]{figurs/logouniv.jpeg}
            \end{figure}
            \begin{RLtext}
                klyT Al`lwm AldaqyqaT
            \end{RLtext}
            {FACULTY OF EXACT SCIENCES}\\
            {COMPUTER SCIENCE DEPARTMENT}\\
            \textsc{\textbf{ \large Master memory} }
        \end{center}
        {\bfseries field :} Mathematics and Computer Science\\
        {\bfseries Faculty  {\hspace*{0.34cm}} :} Computer Science\\
        {\bfseries Option {\hspace*{0.26cm}} :} Networks and distributed system
        \vspace{0.5cm}
        \begin{center}
            \textsc{\textbf{ \large Realized by :}}\\
            {\bfseries Elkaim Moulay Abdellah }{\hspace*{5cm}}{\bfseries Mazouz  Nail}   \\
            \vspace{1cm}
            \Large {\bfseries Theme }
        \end{center}
        \begin{center}
            \hrule width 460pt
            \bigskip
            \Large  \centering \textbf{ \textsc{ ................. } }
            \bigskip
            \hrule width 460pt
        \end{center}
        \raggedright
        \bigskip
        \vspace{1cm}
        \textit{\bfseries Proposed by : Dr. Madber Hayat }
        \vspace{2cm}
        \begin{center}
            \textit{\bfseries College year 2019/2020}
        \end{center}
    \end{titlepage}

    %%Dedicaces%%%%%%%%%%%%%%%%%%%%%%%%%%%%%%%%%%%%%%%%%%%%%%%%%%%%%%%%%%%%%%%%%%%%%%%%%%%%%%%%%%%%%%%%%%%%%%%%%%%%%%%%%%%%%%%%

    \thispagestyle{empty}
    \begin{figure}[h]
    	\centering
    		\includegraphics[width=14cm]{figurs/B.PNG}
    \end{figure}
    \newpage
    \thispagestyle{empty}
    \clearpage
    \newpage
    \pagenumbering{roman}
    \begin{center}
        \textbf{\\}
        \textbf{\\}
        \textbf{\huge \textsc{\itshape Dedications}}\\
        \textbf{\\}
        \textbf{\\}
        \begin{flushleft}
            \textsf{\qquad I dedicate this work to ............. }\\
        \end{flushleft}
    \end{center}
    \begin{flushright}
        \textbf{\textsc{\itshape Elkaim Moulay Abdellah}}
    \end{flushright}
    \newpage
    \begin{center}
        \textbf{\\}
        \textbf{\\}
        \textbf{\huge \textsc{\itshape Dedications}}\\
        \textbf{\\}
        \textbf{\\}

        \begin{flushleft}
            \textsf{\qquad I dedicate this work to ...................... }
        \end{flushleft}

        \normalsize{\itshape .....}
        \textbf{\\}
        \textbf{\\}
    \end{center}
    \begin{flushright}
        \textbf{\textsc{\itshape Mazouz Nail}}
    \end{flushright}

    %%Remerciement%%%%%%%%%%%%%%%%%%%%%%%%%%%%%%%%%%%%%%%%%%%%%%%%%%%%%%%%%%%%%%%%%%%%%%%%%%%%%%%%%%%%%%%%%%%%%%%%%%%%%%%%%%%%%%%%
    \newpage
    \begin{center}
        %\thispagestyle{myheadings}
        %\markboth{droite}{ }
        \textbf{\huge \textsc{\itshape thanks}}
    \end{center}
    \textsf{\qquad We thank .......... }

    %%Abstract %%%%%%%%%%%%%%%%%%%%%%%%%%%%%%%%%%%%%%%%%%%%%%%%%%%%%%%%%%%%%%%%%%%%%%%%%%%%%%%%%%%%%%%%%%%%%%%%%%%%%%%%%%%%%%%%
    \newpage
    \markboth{droite}{ }
    \begin{center}
    \textbf{\huge \textsc{\itshape \textit Abstract}}\\
    \end{center}
    \qquad ...........................

    %%resume %%%%%%%%%%%%%%%%%%%%%%%%%%%%%%%%%%%%%%%%%%%%%%%%%%%%%%%%%%%%%%%%%%%%%%%%%%%%%%%%%%%%%%%%%%%%%%%%%%%%%%%%%%%%%%%%
    \newpage
    \newcommand{\enteteresume}{\markboth{Resume}{Resume}} %il faut ajouter les commandes
    \begin{center}
    \textbf{\huge \textsc{\itshape \textit Résumé}}\\
    \end{center}
    \qquad ...............................

    % La table des matieres%%%%%%%%%%%%%%%%%%%%%%%%%%%%%%%%%%%%%%%%%%%%%%%%%%%%%%%%%%%%%%%%%%%%%%%%%%%%%%%%%%%%%%%%%%%%%%%%%%%%%%%%%%%%%%%%%%%%%%%%%%%%%%%%%%%%%
    \tableofcontents
    \listoffigures
    \listoftables

    %% introduction generale%%%%%%%%%%%%%%%%%%%%%%%%%%%%%%%%%%%%%%%%%%%%%%%%%%%%%%%%%%%%%%%%%%%%%%%%%%%%%%%%%%%%%%%%%%%%%%%%%%%%%%%%%%%%%%%%%%%%%%%%%%%%%%%%%%%%%%%%%%%
    \chapter*{General Introduction}
    \addcontentsline{toc}{chapter}{Introduction générale}
    \setcounter{page}{1}
    \lhead{}
    \cfoot{\bfseries \thepage}
    \rhead{Introduction générale}
    \pagenumbering{arabic}

    ..................................................

    %% chapitre 1%%%%%%%%%%%%%%%%%%%%%%%%%%%%%%%%%%%%%%%%%%%%%%%%%%%%%%%%%%%%%%%%%%%%%%%%%%%%%%%%%%%%%%%%%%%%%%%%%%%%%%%%%%%%%%%%%%%%%%%%%%%%%%%%%%%%%%%%%%%%%%%%%%%%%
    \include{chapiter1/main}

    %% chapitre 2%%%%%%%%%%%%%%%%%%%%%%%%%%%%%%%%%%%%%%%%%%%%%%%%%%%%%%%%%%%%%%%%%%%%%%%%%%%%%%%%%%%%%%%%%%%%%%%%%%%%%%%%%%%%%%%%%%%%%%%%%%%%%%%%%%%%%%%%%%%%%%%%%%%%%
    \include{chapiter2/main}

    %% conclusion generale%%%%%%%%%%%%%%%%%%%%%%%%%%%%%%%%%%%%%%%%%%%%%%%%%%%%%%%%%%%%%%%%%%%%%%%%%%%%%%%%%%%%%%%%%%%%%%%%%%%%%%%%%%%%%%%%%%%%%%%%%%%%%%%%%%%%%%%%%%%
    \chapter*{General conclusion}
    \addcontentsline{toc}{chapter}{General conclusion}
    \lhead{}
    \cfoot{\bfseries \thepage}
    \rhead{General conclusion}
    \markboth{droite}{General conclusion}

    .............................................

    %%References%%%%%%%%%%%%%%%%%%%%%%%%%%%%%%%%%%%%%%%%%%%%%%%%%%%%%%%%%%%%%%%%%%%%%%%%%%%%%%%%%%%%%%%%%%%%%%%%%%%%%%%%%%%%%%%%%%%%%%%%%%%%%%
    \clearpage
    \pagestyle{fancy}
    \addcontentsline{toc}{chapter}{Bibliography}
    \begin{thebibliography}{99}
        \lhead{}
        \cfoot{\bfseries \thepage}
        \rhead{Bibliography}

        \bibitem[1]{1}
        H. Altaama,
        Application Mobile Guide,
        Mémoire de Master en Informatique,
        Université Abou Bakr Belkaid de Tlemcen,
        2016.
        \bibitem[2]{2}
        \url{http://generationmobiles.net/2014/11/les-differents-types-dapps-mobiles/}, [consulté le 14/03/2018].

    \end{thebibliography}

\end{document}

    %% conclusion generale%%%%%%%%%%%%%%%%%%%%%%%%%%%%%%%%%%%%%%%%%%%%%%%%%%%%%%%%%%%%%%%%%%%%%%%%%%%%%%%%%%%%%%%%%%%%%%%%%%%%%%%%%%%%%%%%%%%%%%%%%%%%%%%%%%%%%%%%%%%
    \chapter*{General conclusion}
    \addcontentsline{toc}{chapter}{General conclusion}
    \lhead{}
    \cfoot{\bfseries \thepage}
    \rhead{General conclusion}
    \markboth{droite}{General conclusion}

    .............................................

    %%References%%%%%%%%%%%%%%%%%%%%%%%%%%%%%%%%%%%%%%%%%%%%%%%%%%%%%%%%%%%%%%%%%%%%%%%%%%%%%%%%%%%%%%%%%%%%%%%%%%%%%%%%%%%%%%%%%%%%%%%%%%%%%%
    \clearpage
    \pagestyle{fancy}
    \addcontentsline{toc}{chapter}{Bibliography}
    \begin{thebibliography}{99}
        \lhead{}
        \cfoot{\bfseries \thepage}
        \rhead{Bibliography}

        \bibitem[1]{1}
        H. Altaama,
        Application Mobile Guide,
        Mémoire de Master en Informatique,
        Université Abou Bakr Belkaid de Tlemcen,
        2016.
        \bibitem[2]{2}
        \url{http://generationmobiles.net/2014/11/les-differents-types-dapps-mobiles/}, [consulté le 14/03/2018].

    \end{thebibliography}

\end{document}

    %% chapitre 2%%%%%%%%%%%%%%%%%%%%%%%%%%%%%%%%%%%%%%%%%%%%%%%%%%%%%%%%%%%%%%%%%%%%%%%%%%%%%%%%%%%%%%%%%%%%%%%%%%%%%%%%%%%%%%%%%%%%%%%%%%%%%%%%%%%%%%%%%%%%%%%%%%%%%
    %! Author = moulay
%! Date = 10/21/19

% Preamble
\documentclass[english,a4,12pt]{report}

% Packages
\usepackage{amsmath}
\usepackage[utf8]{inputenc}
\usepackage{xcolor}
\usepackage[T1]{fontenc}
\usepackage{arabtex}
\usepackage{sectsty}
\usepackage[cyr]{aeguill}
\usepackage{rotating}
\usepackage{multirow}
\usepackage{tabulary}
\usepackage{tabularht}
\usepackage{acronym}
\usepackage{fancyhdr}
\usepackage{lscape}
\usepackage{amssymb}
\usepackage{pifont}
\usepackage[most]{tcolorbox}
\usepackage{slashbox}
\usepackage{multido}
\usepackage{caption}
\usepackage{graphicx,wrapfig,lipsum}
\usepackage{enumitem}
\usepackage {fancybox}
\usepackage{array,tabularx}
\usepackage{colortbl}
\usepackage[noend]{algorithmic}
\usepackage[linesnumbered,ruled,vlined,boxed,commentsnumbered]{algorithm2e}
\DeclareGraphicsExtensions{.jpg,.pdf,.PNG,.gif}
\usepackage[pdftex,colorlinks=true,linkcolor=black,citecolor=black,urlcolor=black]{hyperref}
\usepackage{pgfplots}
\usepackage{pgfplotstable}
\usepackage{subcaption}
\usepackage[tikz]{bclogo}
\pgfplotsset{grid style={dashed,gray}}
\pgfplotsset{minor grid style={dotted,green!50!black}}
\pgfplotsset{major grid style={dotted,green!50!black}}
\usepackage{anysize}
\marginsize{30mm}{20mm}{15mm}{15mm}
\newcolumntype{Y}{>{\raggedleft\arraybackslash}X}
\renewcommand{\baselinestretch}{1.5}
\newcolumntype{P}[1]{>{\raggedright}p{#1}}
\newcolumntype{M}[1]{>{\raggedright}m{#1}}%%declaration de page de garde
\newcommand*\rfrac[2]{{}^{#1}\!/_{#2}}
\setcounter{secnumdepth}{3}
\setcounter{tocdepth}{3}
\newcommand{\cmark}{\ding{51}}%
\newcommand{\xmark}{\ding{59}}%

% Document
\begin{document}
    \sloppy
    \begin{titlepage}
        \renewcommand{\baselinestretch}{1}
        \begin{center}
            \begin{RLtext}
                AljmhwryT AljzA'iryT AldymqrA.tyT Al^s`byT
            \end{RLtext}
            {PEOPLE'S DEMOCRATIC REPUBLIC OF ALGERIA}
            \begin{RLtext}
            {wzArT Alt`lym Al`Aly w Alb.h_t Al`lmy}
            \end{RLtext}
            {DEPARTMENT OF HIGHER EDUCATION AND SCIENTIFIC RESEARCH}
            \begin{RLtext}
                jAm`T mus.taf_A s.tmbOly bim`skar
            \end{RLtext}
            {UNIVERSITY OF MUSTAPHA STAMBOULI MASCARA}
            \begin{figure}[h]
            	\centering
            		\includegraphics[width=4cm]{figurs/logouniv.jpeg}
            \end{figure}
            \begin{RLtext}
                klyT Al`lwm AldaqyqaT
            \end{RLtext}
            {FACULTY OF EXACT SCIENCES}\\
            {COMPUTER SCIENCE DEPARTMENT}\\
            \textsc{\textbf{ \large Master memory} }
        \end{center}
        {\bfseries field :} Mathematics and Computer Science\\
        {\bfseries Faculty  {\hspace*{0.34cm}} :} Computer Science\\
        {\bfseries Option {\hspace*{0.26cm}} :} Networks and distributed system
        \vspace{0.5cm}
        \begin{center}
            \textsc{\textbf{ \large Realized by :}}\\
            {\bfseries Elkaim Moulay Abdellah }{\hspace*{5cm}}{\bfseries Mazouz  Nail}   \\
            \vspace{1cm}
            \Large {\bfseries Theme }
        \end{center}
        \begin{center}
            \hrule width 460pt
            \bigskip
            \Large  \centering \textbf{ \textsc{ ................. } }
            \bigskip
            \hrule width 460pt
        \end{center}
        \raggedright
        \bigskip
        \vspace{1cm}
        \textit{\bfseries Proposed by : Dr. Madber Hayat }
        \vspace{2cm}
        \begin{center}
            \textit{\bfseries College year 2019/2020}
        \end{center}
    \end{titlepage}

    %%Dedicaces%%%%%%%%%%%%%%%%%%%%%%%%%%%%%%%%%%%%%%%%%%%%%%%%%%%%%%%%%%%%%%%%%%%%%%%%%%%%%%%%%%%%%%%%%%%%%%%%%%%%%%%%%%%%%%%%

    \thispagestyle{empty}
    \begin{figure}[h]
    	\centering
    		\includegraphics[width=14cm]{figurs/B.PNG}
    \end{figure}
    \newpage
    \thispagestyle{empty}
    \clearpage
    \newpage
    \pagenumbering{roman}
    \begin{center}
        \textbf{\\}
        \textbf{\\}
        \textbf{\huge \textsc{\itshape Dedications}}\\
        \textbf{\\}
        \textbf{\\}
        \begin{flushleft}
            \textsf{\qquad I dedicate this work to ............. }\\
        \end{flushleft}
    \end{center}
    \begin{flushright}
        \textbf{\textsc{\itshape Elkaim Moulay Abdellah}}
    \end{flushright}
    \newpage
    \begin{center}
        \textbf{\\}
        \textbf{\\}
        \textbf{\huge \textsc{\itshape Dedications}}\\
        \textbf{\\}
        \textbf{\\}

        \begin{flushleft}
            \textsf{\qquad I dedicate this work to ...................... }
        \end{flushleft}

        \normalsize{\itshape .....}
        \textbf{\\}
        \textbf{\\}
    \end{center}
    \begin{flushright}
        \textbf{\textsc{\itshape Mazouz Nail}}
    \end{flushright}

    %%Remerciement%%%%%%%%%%%%%%%%%%%%%%%%%%%%%%%%%%%%%%%%%%%%%%%%%%%%%%%%%%%%%%%%%%%%%%%%%%%%%%%%%%%%%%%%%%%%%%%%%%%%%%%%%%%%%%%%
    \newpage
    \begin{center}
        %\thispagestyle{myheadings}
        %\markboth{droite}{ }
        \textbf{\huge \textsc{\itshape thanks}}
    \end{center}
    \textsf{\qquad We thank .......... }

    %%Abstract %%%%%%%%%%%%%%%%%%%%%%%%%%%%%%%%%%%%%%%%%%%%%%%%%%%%%%%%%%%%%%%%%%%%%%%%%%%%%%%%%%%%%%%%%%%%%%%%%%%%%%%%%%%%%%%%
    \newpage
    \markboth{droite}{ }
    \begin{center}
    \textbf{\huge \textsc{\itshape \textit Abstract}}\\
    \end{center}
    \qquad ...........................

    %%resume %%%%%%%%%%%%%%%%%%%%%%%%%%%%%%%%%%%%%%%%%%%%%%%%%%%%%%%%%%%%%%%%%%%%%%%%%%%%%%%%%%%%%%%%%%%%%%%%%%%%%%%%%%%%%%%%
    \newpage
    \newcommand{\enteteresume}{\markboth{Resume}{Resume}} %il faut ajouter les commandes
    \begin{center}
    \textbf{\huge \textsc{\itshape \textit Résumé}}\\
    \end{center}
    \qquad ...............................

    % La table des matieres%%%%%%%%%%%%%%%%%%%%%%%%%%%%%%%%%%%%%%%%%%%%%%%%%%%%%%%%%%%%%%%%%%%%%%%%%%%%%%%%%%%%%%%%%%%%%%%%%%%%%%%%%%%%%%%%%%%%%%%%%%%%%%%%%%%%%
    \tableofcontents
    \listoffigures
    \listoftables

    %% introduction generale%%%%%%%%%%%%%%%%%%%%%%%%%%%%%%%%%%%%%%%%%%%%%%%%%%%%%%%%%%%%%%%%%%%%%%%%%%%%%%%%%%%%%%%%%%%%%%%%%%%%%%%%%%%%%%%%%%%%%%%%%%%%%%%%%%%%%%%%%%%
    \chapter*{General Introduction}
    \addcontentsline{toc}{chapter}{Introduction générale}
    \setcounter{page}{1}
    \lhead{}
    \cfoot{\bfseries \thepage}
    \rhead{Introduction générale}
    \pagenumbering{arabic}

    ..................................................

    %% chapitre 1%%%%%%%%%%%%%%%%%%%%%%%%%%%%%%%%%%%%%%%%%%%%%%%%%%%%%%%%%%%%%%%%%%%%%%%%%%%%%%%%%%%%%%%%%%%%%%%%%%%%%%%%%%%%%%%%%%%%%%%%%%%%%%%%%%%%%%%%%%%%%%%%%%%%%
    %! Author = moulay
%! Date = 10/21/19

% Preamble
\documentclass[english,a4,12pt]{report}

% Packages
\usepackage{amsmath}
\usepackage[utf8]{inputenc}
\usepackage{xcolor}
\usepackage[T1]{fontenc}
\usepackage{arabtex}
\usepackage{sectsty}
\usepackage[cyr]{aeguill}
\usepackage{rotating}
\usepackage{multirow}
\usepackage{tabulary}
\usepackage{tabularht}
\usepackage{acronym}
\usepackage{fancyhdr}
\usepackage{lscape}
\usepackage{amssymb}
\usepackage{pifont}
\usepackage[most]{tcolorbox}
\usepackage{slashbox}
\usepackage{multido}
\usepackage{caption}
\usepackage{graphicx,wrapfig,lipsum}
\usepackage{enumitem}
\usepackage {fancybox}
\usepackage{array,tabularx}
\usepackage{colortbl}
\usepackage[noend]{algorithmic}
\usepackage[linesnumbered,ruled,vlined,boxed,commentsnumbered]{algorithm2e}
\DeclareGraphicsExtensions{.jpg,.pdf,.PNG,.gif}
\usepackage[pdftex,colorlinks=true,linkcolor=black,citecolor=black,urlcolor=black]{hyperref}
\usepackage{pgfplots}
\usepackage{pgfplotstable}
\usepackage{subcaption}
\usepackage[tikz]{bclogo}
\pgfplotsset{grid style={dashed,gray}}
\pgfplotsset{minor grid style={dotted,green!50!black}}
\pgfplotsset{major grid style={dotted,green!50!black}}
\usepackage{anysize}
\marginsize{30mm}{20mm}{15mm}{15mm}
\newcolumntype{Y}{>{\raggedleft\arraybackslash}X}
\renewcommand{\baselinestretch}{1.5}
\newcolumntype{P}[1]{>{\raggedright}p{#1}}
\newcolumntype{M}[1]{>{\raggedright}m{#1}}%%declaration de page de garde
\newcommand*\rfrac[2]{{}^{#1}\!/_{#2}}
\setcounter{secnumdepth}{3}
\setcounter{tocdepth}{3}
\newcommand{\cmark}{\ding{51}}%
\newcommand{\xmark}{\ding{59}}%

% Document
\begin{document}
    \sloppy
    \begin{titlepage}
        \renewcommand{\baselinestretch}{1}
        \begin{center}
            \begin{RLtext}
                AljmhwryT AljzA'iryT AldymqrA.tyT Al^s`byT
            \end{RLtext}
            {PEOPLE'S DEMOCRATIC REPUBLIC OF ALGERIA}
            \begin{RLtext}
            {wzArT Alt`lym Al`Aly w Alb.h_t Al`lmy}
            \end{RLtext}
            {DEPARTMENT OF HIGHER EDUCATION AND SCIENTIFIC RESEARCH}
            \begin{RLtext}
                jAm`T mus.taf_A s.tmbOly bim`skar
            \end{RLtext}
            {UNIVERSITY OF MUSTAPHA STAMBOULI MASCARA}
            \begin{figure}[h]
            	\centering
            		\includegraphics[width=4cm]{figurs/logouniv.jpeg}
            \end{figure}
            \begin{RLtext}
                klyT Al`lwm AldaqyqaT
            \end{RLtext}
            {FACULTY OF EXACT SCIENCES}\\
            {COMPUTER SCIENCE DEPARTMENT}\\
            \textsc{\textbf{ \large Master memory} }
        \end{center}
        {\bfseries field :} Mathematics and Computer Science\\
        {\bfseries Faculty  {\hspace*{0.34cm}} :} Computer Science\\
        {\bfseries Option {\hspace*{0.26cm}} :} Networks and distributed system
        \vspace{0.5cm}
        \begin{center}
            \textsc{\textbf{ \large Realized by :}}\\
            {\bfseries Elkaim Moulay Abdellah }{\hspace*{5cm}}{\bfseries Mazouz  Nail}   \\
            \vspace{1cm}
            \Large {\bfseries Theme }
        \end{center}
        \begin{center}
            \hrule width 460pt
            \bigskip
            \Large  \centering \textbf{ \textsc{ ................. } }
            \bigskip
            \hrule width 460pt
        \end{center}
        \raggedright
        \bigskip
        \vspace{1cm}
        \textit{\bfseries Proposed by : Dr. Madber Hayat }
        \vspace{2cm}
        \begin{center}
            \textit{\bfseries College year 2019/2020}
        \end{center}
    \end{titlepage}

    %%Dedicaces%%%%%%%%%%%%%%%%%%%%%%%%%%%%%%%%%%%%%%%%%%%%%%%%%%%%%%%%%%%%%%%%%%%%%%%%%%%%%%%%%%%%%%%%%%%%%%%%%%%%%%%%%%%%%%%%

    \thispagestyle{empty}
    \begin{figure}[h]
    	\centering
    		\includegraphics[width=14cm]{figurs/B.PNG}
    \end{figure}
    \newpage
    \thispagestyle{empty}
    \clearpage
    \newpage
    \pagenumbering{roman}
    \begin{center}
        \textbf{\\}
        \textbf{\\}
        \textbf{\huge \textsc{\itshape Dedications}}\\
        \textbf{\\}
        \textbf{\\}
        \begin{flushleft}
            \textsf{\qquad I dedicate this work to ............. }\\
        \end{flushleft}
    \end{center}
    \begin{flushright}
        \textbf{\textsc{\itshape Elkaim Moulay Abdellah}}
    \end{flushright}
    \newpage
    \begin{center}
        \textbf{\\}
        \textbf{\\}
        \textbf{\huge \textsc{\itshape Dedications}}\\
        \textbf{\\}
        \textbf{\\}

        \begin{flushleft}
            \textsf{\qquad I dedicate this work to ...................... }
        \end{flushleft}

        \normalsize{\itshape .....}
        \textbf{\\}
        \textbf{\\}
    \end{center}
    \begin{flushright}
        \textbf{\textsc{\itshape Mazouz Nail}}
    \end{flushright}

    %%Remerciement%%%%%%%%%%%%%%%%%%%%%%%%%%%%%%%%%%%%%%%%%%%%%%%%%%%%%%%%%%%%%%%%%%%%%%%%%%%%%%%%%%%%%%%%%%%%%%%%%%%%%%%%%%%%%%%%
    \newpage
    \begin{center}
        %\thispagestyle{myheadings}
        %\markboth{droite}{ }
        \textbf{\huge \textsc{\itshape thanks}}
    \end{center}
    \textsf{\qquad We thank .......... }

    %%Abstract %%%%%%%%%%%%%%%%%%%%%%%%%%%%%%%%%%%%%%%%%%%%%%%%%%%%%%%%%%%%%%%%%%%%%%%%%%%%%%%%%%%%%%%%%%%%%%%%%%%%%%%%%%%%%%%%
    \newpage
    \markboth{droite}{ }
    \begin{center}
    \textbf{\huge \textsc{\itshape \textit Abstract}}\\
    \end{center}
    \qquad ...........................

    %%resume %%%%%%%%%%%%%%%%%%%%%%%%%%%%%%%%%%%%%%%%%%%%%%%%%%%%%%%%%%%%%%%%%%%%%%%%%%%%%%%%%%%%%%%%%%%%%%%%%%%%%%%%%%%%%%%%
    \newpage
    \newcommand{\enteteresume}{\markboth{Resume}{Resume}} %il faut ajouter les commandes
    \begin{center}
    \textbf{\huge \textsc{\itshape \textit Résumé}}\\
    \end{center}
    \qquad ...............................

    % La table des matieres%%%%%%%%%%%%%%%%%%%%%%%%%%%%%%%%%%%%%%%%%%%%%%%%%%%%%%%%%%%%%%%%%%%%%%%%%%%%%%%%%%%%%%%%%%%%%%%%%%%%%%%%%%%%%%%%%%%%%%%%%%%%%%%%%%%%%
    \tableofcontents
    \listoffigures
    \listoftables

    %% introduction generale%%%%%%%%%%%%%%%%%%%%%%%%%%%%%%%%%%%%%%%%%%%%%%%%%%%%%%%%%%%%%%%%%%%%%%%%%%%%%%%%%%%%%%%%%%%%%%%%%%%%%%%%%%%%%%%%%%%%%%%%%%%%%%%%%%%%%%%%%%%
    \chapter*{General Introduction}
    \addcontentsline{toc}{chapter}{Introduction générale}
    \setcounter{page}{1}
    \lhead{}
    \cfoot{\bfseries \thepage}
    \rhead{Introduction générale}
    \pagenumbering{arabic}

    ..................................................

    %% chapitre 1%%%%%%%%%%%%%%%%%%%%%%%%%%%%%%%%%%%%%%%%%%%%%%%%%%%%%%%%%%%%%%%%%%%%%%%%%%%%%%%%%%%%%%%%%%%%%%%%%%%%%%%%%%%%%%%%%%%%%%%%%%%%%%%%%%%%%%%%%%%%%%%%%%%%%
    \include{chapiter1/main}

    %% chapitre 2%%%%%%%%%%%%%%%%%%%%%%%%%%%%%%%%%%%%%%%%%%%%%%%%%%%%%%%%%%%%%%%%%%%%%%%%%%%%%%%%%%%%%%%%%%%%%%%%%%%%%%%%%%%%%%%%%%%%%%%%%%%%%%%%%%%%%%%%%%%%%%%%%%%%%
    \include{chapiter2/main}

    %% conclusion generale%%%%%%%%%%%%%%%%%%%%%%%%%%%%%%%%%%%%%%%%%%%%%%%%%%%%%%%%%%%%%%%%%%%%%%%%%%%%%%%%%%%%%%%%%%%%%%%%%%%%%%%%%%%%%%%%%%%%%%%%%%%%%%%%%%%%%%%%%%%
    \chapter*{General conclusion}
    \addcontentsline{toc}{chapter}{General conclusion}
    \lhead{}
    \cfoot{\bfseries \thepage}
    \rhead{General conclusion}
    \markboth{droite}{General conclusion}

    .............................................

    %%References%%%%%%%%%%%%%%%%%%%%%%%%%%%%%%%%%%%%%%%%%%%%%%%%%%%%%%%%%%%%%%%%%%%%%%%%%%%%%%%%%%%%%%%%%%%%%%%%%%%%%%%%%%%%%%%%%%%%%%%%%%%%%%
    \clearpage
    \pagestyle{fancy}
    \addcontentsline{toc}{chapter}{Bibliography}
    \begin{thebibliography}{99}
        \lhead{}
        \cfoot{\bfseries \thepage}
        \rhead{Bibliography}

        \bibitem[1]{1}
        H. Altaama,
        Application Mobile Guide,
        Mémoire de Master en Informatique,
        Université Abou Bakr Belkaid de Tlemcen,
        2016.
        \bibitem[2]{2}
        \url{http://generationmobiles.net/2014/11/les-differents-types-dapps-mobiles/}, [consulté le 14/03/2018].

    \end{thebibliography}

\end{document}

    %% chapitre 2%%%%%%%%%%%%%%%%%%%%%%%%%%%%%%%%%%%%%%%%%%%%%%%%%%%%%%%%%%%%%%%%%%%%%%%%%%%%%%%%%%%%%%%%%%%%%%%%%%%%%%%%%%%%%%%%%%%%%%%%%%%%%%%%%%%%%%%%%%%%%%%%%%%%%
    %! Author = moulay
%! Date = 10/21/19

% Preamble
\documentclass[english,a4,12pt]{report}

% Packages
\usepackage{amsmath}
\usepackage[utf8]{inputenc}
\usepackage{xcolor}
\usepackage[T1]{fontenc}
\usepackage{arabtex}
\usepackage{sectsty}
\usepackage[cyr]{aeguill}
\usepackage{rotating}
\usepackage{multirow}
\usepackage{tabulary}
\usepackage{tabularht}
\usepackage{acronym}
\usepackage{fancyhdr}
\usepackage{lscape}
\usepackage{amssymb}
\usepackage{pifont}
\usepackage[most]{tcolorbox}
\usepackage{slashbox}
\usepackage{multido}
\usepackage{caption}
\usepackage{graphicx,wrapfig,lipsum}
\usepackage{enumitem}
\usepackage {fancybox}
\usepackage{array,tabularx}
\usepackage{colortbl}
\usepackage[noend]{algorithmic}
\usepackage[linesnumbered,ruled,vlined,boxed,commentsnumbered]{algorithm2e}
\DeclareGraphicsExtensions{.jpg,.pdf,.PNG,.gif}
\usepackage[pdftex,colorlinks=true,linkcolor=black,citecolor=black,urlcolor=black]{hyperref}
\usepackage{pgfplots}
\usepackage{pgfplotstable}
\usepackage{subcaption}
\usepackage[tikz]{bclogo}
\pgfplotsset{grid style={dashed,gray}}
\pgfplotsset{minor grid style={dotted,green!50!black}}
\pgfplotsset{major grid style={dotted,green!50!black}}
\usepackage{anysize}
\marginsize{30mm}{20mm}{15mm}{15mm}
\newcolumntype{Y}{>{\raggedleft\arraybackslash}X}
\renewcommand{\baselinestretch}{1.5}
\newcolumntype{P}[1]{>{\raggedright}p{#1}}
\newcolumntype{M}[1]{>{\raggedright}m{#1}}%%declaration de page de garde
\newcommand*\rfrac[2]{{}^{#1}\!/_{#2}}
\setcounter{secnumdepth}{3}
\setcounter{tocdepth}{3}
\newcommand{\cmark}{\ding{51}}%
\newcommand{\xmark}{\ding{59}}%

% Document
\begin{document}
    \sloppy
    \begin{titlepage}
        \renewcommand{\baselinestretch}{1}
        \begin{center}
            \begin{RLtext}
                AljmhwryT AljzA'iryT AldymqrA.tyT Al^s`byT
            \end{RLtext}
            {PEOPLE'S DEMOCRATIC REPUBLIC OF ALGERIA}
            \begin{RLtext}
            {wzArT Alt`lym Al`Aly w Alb.h_t Al`lmy}
            \end{RLtext}
            {DEPARTMENT OF HIGHER EDUCATION AND SCIENTIFIC RESEARCH}
            \begin{RLtext}
                jAm`T mus.taf_A s.tmbOly bim`skar
            \end{RLtext}
            {UNIVERSITY OF MUSTAPHA STAMBOULI MASCARA}
            \begin{figure}[h]
            	\centering
            		\includegraphics[width=4cm]{figurs/logouniv.jpeg}
            \end{figure}
            \begin{RLtext}
                klyT Al`lwm AldaqyqaT
            \end{RLtext}
            {FACULTY OF EXACT SCIENCES}\\
            {COMPUTER SCIENCE DEPARTMENT}\\
            \textsc{\textbf{ \large Master memory} }
        \end{center}
        {\bfseries field :} Mathematics and Computer Science\\
        {\bfseries Faculty  {\hspace*{0.34cm}} :} Computer Science\\
        {\bfseries Option {\hspace*{0.26cm}} :} Networks and distributed system
        \vspace{0.5cm}
        \begin{center}
            \textsc{\textbf{ \large Realized by :}}\\
            {\bfseries Elkaim Moulay Abdellah }{\hspace*{5cm}}{\bfseries Mazouz  Nail}   \\
            \vspace{1cm}
            \Large {\bfseries Theme }
        \end{center}
        \begin{center}
            \hrule width 460pt
            \bigskip
            \Large  \centering \textbf{ \textsc{ ................. } }
            \bigskip
            \hrule width 460pt
        \end{center}
        \raggedright
        \bigskip
        \vspace{1cm}
        \textit{\bfseries Proposed by : Dr. Madber Hayat }
        \vspace{2cm}
        \begin{center}
            \textit{\bfseries College year 2019/2020}
        \end{center}
    \end{titlepage}

    %%Dedicaces%%%%%%%%%%%%%%%%%%%%%%%%%%%%%%%%%%%%%%%%%%%%%%%%%%%%%%%%%%%%%%%%%%%%%%%%%%%%%%%%%%%%%%%%%%%%%%%%%%%%%%%%%%%%%%%%

    \thispagestyle{empty}
    \begin{figure}[h]
    	\centering
    		\includegraphics[width=14cm]{figurs/B.PNG}
    \end{figure}
    \newpage
    \thispagestyle{empty}
    \clearpage
    \newpage
    \pagenumbering{roman}
    \begin{center}
        \textbf{\\}
        \textbf{\\}
        \textbf{\huge \textsc{\itshape Dedications}}\\
        \textbf{\\}
        \textbf{\\}
        \begin{flushleft}
            \textsf{\qquad I dedicate this work to ............. }\\
        \end{flushleft}
    \end{center}
    \begin{flushright}
        \textbf{\textsc{\itshape Elkaim Moulay Abdellah}}
    \end{flushright}
    \newpage
    \begin{center}
        \textbf{\\}
        \textbf{\\}
        \textbf{\huge \textsc{\itshape Dedications}}\\
        \textbf{\\}
        \textbf{\\}

        \begin{flushleft}
            \textsf{\qquad I dedicate this work to ...................... }
        \end{flushleft}

        \normalsize{\itshape .....}
        \textbf{\\}
        \textbf{\\}
    \end{center}
    \begin{flushright}
        \textbf{\textsc{\itshape Mazouz Nail}}
    \end{flushright}

    %%Remerciement%%%%%%%%%%%%%%%%%%%%%%%%%%%%%%%%%%%%%%%%%%%%%%%%%%%%%%%%%%%%%%%%%%%%%%%%%%%%%%%%%%%%%%%%%%%%%%%%%%%%%%%%%%%%%%%%
    \newpage
    \begin{center}
        %\thispagestyle{myheadings}
        %\markboth{droite}{ }
        \textbf{\huge \textsc{\itshape thanks}}
    \end{center}
    \textsf{\qquad We thank .......... }

    %%Abstract %%%%%%%%%%%%%%%%%%%%%%%%%%%%%%%%%%%%%%%%%%%%%%%%%%%%%%%%%%%%%%%%%%%%%%%%%%%%%%%%%%%%%%%%%%%%%%%%%%%%%%%%%%%%%%%%
    \newpage
    \markboth{droite}{ }
    \begin{center}
    \textbf{\huge \textsc{\itshape \textit Abstract}}\\
    \end{center}
    \qquad ...........................

    %%resume %%%%%%%%%%%%%%%%%%%%%%%%%%%%%%%%%%%%%%%%%%%%%%%%%%%%%%%%%%%%%%%%%%%%%%%%%%%%%%%%%%%%%%%%%%%%%%%%%%%%%%%%%%%%%%%%
    \newpage
    \newcommand{\enteteresume}{\markboth{Resume}{Resume}} %il faut ajouter les commandes
    \begin{center}
    \textbf{\huge \textsc{\itshape \textit Résumé}}\\
    \end{center}
    \qquad ...............................

    % La table des matieres%%%%%%%%%%%%%%%%%%%%%%%%%%%%%%%%%%%%%%%%%%%%%%%%%%%%%%%%%%%%%%%%%%%%%%%%%%%%%%%%%%%%%%%%%%%%%%%%%%%%%%%%%%%%%%%%%%%%%%%%%%%%%%%%%%%%%
    \tableofcontents
    \listoffigures
    \listoftables

    %% introduction generale%%%%%%%%%%%%%%%%%%%%%%%%%%%%%%%%%%%%%%%%%%%%%%%%%%%%%%%%%%%%%%%%%%%%%%%%%%%%%%%%%%%%%%%%%%%%%%%%%%%%%%%%%%%%%%%%%%%%%%%%%%%%%%%%%%%%%%%%%%%
    \chapter*{General Introduction}
    \addcontentsline{toc}{chapter}{Introduction générale}
    \setcounter{page}{1}
    \lhead{}
    \cfoot{\bfseries \thepage}
    \rhead{Introduction générale}
    \pagenumbering{arabic}

    ..................................................

    %% chapitre 1%%%%%%%%%%%%%%%%%%%%%%%%%%%%%%%%%%%%%%%%%%%%%%%%%%%%%%%%%%%%%%%%%%%%%%%%%%%%%%%%%%%%%%%%%%%%%%%%%%%%%%%%%%%%%%%%%%%%%%%%%%%%%%%%%%%%%%%%%%%%%%%%%%%%%
    \include{chapiter1/main}

    %% chapitre 2%%%%%%%%%%%%%%%%%%%%%%%%%%%%%%%%%%%%%%%%%%%%%%%%%%%%%%%%%%%%%%%%%%%%%%%%%%%%%%%%%%%%%%%%%%%%%%%%%%%%%%%%%%%%%%%%%%%%%%%%%%%%%%%%%%%%%%%%%%%%%%%%%%%%%
    \include{chapiter2/main}

    %% conclusion generale%%%%%%%%%%%%%%%%%%%%%%%%%%%%%%%%%%%%%%%%%%%%%%%%%%%%%%%%%%%%%%%%%%%%%%%%%%%%%%%%%%%%%%%%%%%%%%%%%%%%%%%%%%%%%%%%%%%%%%%%%%%%%%%%%%%%%%%%%%%
    \chapter*{General conclusion}
    \addcontentsline{toc}{chapter}{General conclusion}
    \lhead{}
    \cfoot{\bfseries \thepage}
    \rhead{General conclusion}
    \markboth{droite}{General conclusion}

    .............................................

    %%References%%%%%%%%%%%%%%%%%%%%%%%%%%%%%%%%%%%%%%%%%%%%%%%%%%%%%%%%%%%%%%%%%%%%%%%%%%%%%%%%%%%%%%%%%%%%%%%%%%%%%%%%%%%%%%%%%%%%%%%%%%%%%%
    \clearpage
    \pagestyle{fancy}
    \addcontentsline{toc}{chapter}{Bibliography}
    \begin{thebibliography}{99}
        \lhead{}
        \cfoot{\bfseries \thepage}
        \rhead{Bibliography}

        \bibitem[1]{1}
        H. Altaama,
        Application Mobile Guide,
        Mémoire de Master en Informatique,
        Université Abou Bakr Belkaid de Tlemcen,
        2016.
        \bibitem[2]{2}
        \url{http://generationmobiles.net/2014/11/les-differents-types-dapps-mobiles/}, [consulté le 14/03/2018].

    \end{thebibliography}

\end{document}

    %% conclusion generale%%%%%%%%%%%%%%%%%%%%%%%%%%%%%%%%%%%%%%%%%%%%%%%%%%%%%%%%%%%%%%%%%%%%%%%%%%%%%%%%%%%%%%%%%%%%%%%%%%%%%%%%%%%%%%%%%%%%%%%%%%%%%%%%%%%%%%%%%%%
    \chapter*{General conclusion}
    \addcontentsline{toc}{chapter}{General conclusion}
    \lhead{}
    \cfoot{\bfseries \thepage}
    \rhead{General conclusion}
    \markboth{droite}{General conclusion}

    .............................................

    %%References%%%%%%%%%%%%%%%%%%%%%%%%%%%%%%%%%%%%%%%%%%%%%%%%%%%%%%%%%%%%%%%%%%%%%%%%%%%%%%%%%%%%%%%%%%%%%%%%%%%%%%%%%%%%%%%%%%%%%%%%%%%%%%
    \clearpage
    \pagestyle{fancy}
    \addcontentsline{toc}{chapter}{Bibliography}
    \begin{thebibliography}{99}
        \lhead{}
        \cfoot{\bfseries \thepage}
        \rhead{Bibliography}

        \bibitem[1]{1}
        H. Altaama,
        Application Mobile Guide,
        Mémoire de Master en Informatique,
        Université Abou Bakr Belkaid de Tlemcen,
        2016.
        \bibitem[2]{2}
        \url{http://generationmobiles.net/2014/11/les-differents-types-dapps-mobiles/}, [consulté le 14/03/2018].

    \end{thebibliography}

\end{document}

    %% conclusion generale%%%%%%%%%%%%%%%%%%%%%%%%%%%%%%%%%%%%%%%%%%%%%%%%%%%%%%%%%%%%%%%%%%%%%%%%%%%%%%%%%%%%%%%%%%%%%%%%%%%%%%%%%%%%%%%%%%%%%%%%%%%%%%%%%%%%%%%%%%%
    \chapter*{General conclusion}
    \addcontentsline{toc}{chapter}{General conclusion}
    \lhead{}
    \cfoot{\bfseries \thepage}
    \rhead{General conclusion}
    \markboth{droite}{General conclusion}

    .............................................

    %%References%%%%%%%%%%%%%%%%%%%%%%%%%%%%%%%%%%%%%%%%%%%%%%%%%%%%%%%%%%%%%%%%%%%%%%%%%%%%%%%%%%%%%%%%%%%%%%%%%%%%%%%%%%%%%%%%%%%%%%%%%%%%%%
    \clearpage
    \pagestyle{fancy}
    \addcontentsline{toc}{chapter}{Bibliography}
    \begin{thebibliography}{99}
        \lhead{}
        \cfoot{\bfseries \thepage}
        \rhead{Bibliography}

        \bibitem[1]{1}
        H. Altaama,
        Application Mobile Guide,
        Mémoire de Master en Informatique,
        Université Abou Bakr Belkaid de Tlemcen,
        2016.
        \bibitem[2]{2}
        \url{http://generationmobiles.net/2014/11/les-differents-types-dapps-mobiles/}, [consulté le 14/03/2018].

    \end{thebibliography}

\end{document}

    %% chapitre 2%%%%%%%%%%%%%%%%%%%%%%%%%%%%%%%%%%%%%%%%%%%%%%%%%%%%%%%%%%%%%%%%%%%%%%%%%%%%%%%%%%%%%%%%%%%%%%%%%%%%%%%%%%%%%%%%%%%%%%%%%%%%%%%%%%%%%%%%%%%%%%%%%%%%%
    %! Author = moulay
%! Date = 10/21/19

% Preamble
\documentclass[english,a4,12pt]{report}

% Packages
\usepackage{amsmath}
\usepackage[utf8]{inputenc}
\usepackage{xcolor}
\usepackage[T1]{fontenc}
\usepackage{arabtex}
\usepackage{sectsty}
\usepackage[cyr]{aeguill}
\usepackage{rotating}
\usepackage{multirow}
\usepackage{tabulary}
\usepackage{tabularht}
\usepackage{acronym}
\usepackage{fancyhdr}
\usepackage{lscape}
\usepackage{amssymb}
\usepackage{pifont}
\usepackage[most]{tcolorbox}
\usepackage{slashbox}
\usepackage{multido}
\usepackage{caption}
\usepackage{graphicx,wrapfig,lipsum}
\usepackage{enumitem}
\usepackage {fancybox}
\usepackage{array,tabularx}
\usepackage{colortbl}
\usepackage[noend]{algorithmic}
\usepackage[linesnumbered,ruled,vlined,boxed,commentsnumbered]{algorithm2e}
\DeclareGraphicsExtensions{.jpg,.pdf,.PNG,.gif}
\usepackage[pdftex,colorlinks=true,linkcolor=black,citecolor=black,urlcolor=black]{hyperref}
\usepackage{pgfplots}
\usepackage{pgfplotstable}
\usepackage{subcaption}
\usepackage[tikz]{bclogo}
\pgfplotsset{grid style={dashed,gray}}
\pgfplotsset{minor grid style={dotted,green!50!black}}
\pgfplotsset{major grid style={dotted,green!50!black}}
\usepackage{anysize}
\marginsize{30mm}{20mm}{15mm}{15mm}
\newcolumntype{Y}{>{\raggedleft\arraybackslash}X}
\renewcommand{\baselinestretch}{1.5}
\newcolumntype{P}[1]{>{\raggedright}p{#1}}
\newcolumntype{M}[1]{>{\raggedright}m{#1}}%%declaration de page de garde
\newcommand*\rfrac[2]{{}^{#1}\!/_{#2}}
\setcounter{secnumdepth}{3}
\setcounter{tocdepth}{3}
\newcommand{\cmark}{\ding{51}}%
\newcommand{\xmark}{\ding{59}}%

% Document
\begin{document}
    \sloppy
    \begin{titlepage}
        \renewcommand{\baselinestretch}{1}
        \begin{center}
            \begin{RLtext}
                AljmhwryT AljzA'iryT AldymqrA.tyT Al^s`byT
            \end{RLtext}
            {PEOPLE'S DEMOCRATIC REPUBLIC OF ALGERIA}
            \begin{RLtext}
            {wzArT Alt`lym Al`Aly w Alb.h_t Al`lmy}
            \end{RLtext}
            {DEPARTMENT OF HIGHER EDUCATION AND SCIENTIFIC RESEARCH}
            \begin{RLtext}
                jAm`T mus.taf_A s.tmbOly bim`skar
            \end{RLtext}
            {UNIVERSITY OF MUSTAPHA STAMBOULI MASCARA}
            \begin{figure}[h]
            	\centering
            		\includegraphics[width=4cm]{figurs/logouniv.jpeg}
            \end{figure}
            \begin{RLtext}
                klyT Al`lwm AldaqyqaT
            \end{RLtext}
            {FACULTY OF EXACT SCIENCES}\\
            {COMPUTER SCIENCE DEPARTMENT}\\
            \textsc{\textbf{ \large Master memory} }
        \end{center}
        {\bfseries field :} Mathematics and Computer Science\\
        {\bfseries Faculty  {\hspace*{0.34cm}} :} Computer Science\\
        {\bfseries Option {\hspace*{0.26cm}} :} Networks and distributed system
        \vspace{0.5cm}
        \begin{center}
            \textsc{\textbf{ \large Realized by :}}\\
            {\bfseries Elkaim Moulay Abdellah }{\hspace*{5cm}}{\bfseries Mazouz  Nail}   \\
            \vspace{1cm}
            \Large {\bfseries Theme }
        \end{center}
        \begin{center}
            \hrule width 460pt
            \bigskip
            \Large  \centering \textbf{ \textsc{ ................. } }
            \bigskip
            \hrule width 460pt
        \end{center}
        \raggedright
        \bigskip
        \vspace{1cm}
        \textit{\bfseries Proposed by : Dr. Madber Hayat }
        \vspace{2cm}
        \begin{center}
            \textit{\bfseries College year 2019/2020}
        \end{center}
    \end{titlepage}

    %%Dedicaces%%%%%%%%%%%%%%%%%%%%%%%%%%%%%%%%%%%%%%%%%%%%%%%%%%%%%%%%%%%%%%%%%%%%%%%%%%%%%%%%%%%%%%%%%%%%%%%%%%%%%%%%%%%%%%%%

    \thispagestyle{empty}
    \begin{figure}[h]
    	\centering
    		\includegraphics[width=14cm]{figurs/B.PNG}
    \end{figure}
    \newpage
    \thispagestyle{empty}
    \clearpage
    \newpage
    \pagenumbering{roman}
    \begin{center}
        \textbf{\\}
        \textbf{\\}
        \textbf{\huge \textsc{\itshape Dedications}}\\
        \textbf{\\}
        \textbf{\\}
        \begin{flushleft}
            \textsf{\qquad I dedicate this work to ............. }\\
        \end{flushleft}
    \end{center}
    \begin{flushright}
        \textbf{\textsc{\itshape Elkaim Moulay Abdellah}}
    \end{flushright}
    \newpage
    \begin{center}
        \textbf{\\}
        \textbf{\\}
        \textbf{\huge \textsc{\itshape Dedications}}\\
        \textbf{\\}
        \textbf{\\}

        \begin{flushleft}
            \textsf{\qquad I dedicate this work to ...................... }
        \end{flushleft}

        \normalsize{\itshape .....}
        \textbf{\\}
        \textbf{\\}
    \end{center}
    \begin{flushright}
        \textbf{\textsc{\itshape Mazouz Nail}}
    \end{flushright}

    %%Remerciement%%%%%%%%%%%%%%%%%%%%%%%%%%%%%%%%%%%%%%%%%%%%%%%%%%%%%%%%%%%%%%%%%%%%%%%%%%%%%%%%%%%%%%%%%%%%%%%%%%%%%%%%%%%%%%%%
    \newpage
    \begin{center}
        %\thispagestyle{myheadings}
        %\markboth{droite}{ }
        \textbf{\huge \textsc{\itshape thanks}}
    \end{center}
    \textsf{\qquad We thank .......... }

    %%Abstract %%%%%%%%%%%%%%%%%%%%%%%%%%%%%%%%%%%%%%%%%%%%%%%%%%%%%%%%%%%%%%%%%%%%%%%%%%%%%%%%%%%%%%%%%%%%%%%%%%%%%%%%%%%%%%%%
    \newpage
    \markboth{droite}{ }
    \begin{center}
    \textbf{\huge \textsc{\itshape \textit Abstract}}\\
    \end{center}
    \qquad ...........................

    %%resume %%%%%%%%%%%%%%%%%%%%%%%%%%%%%%%%%%%%%%%%%%%%%%%%%%%%%%%%%%%%%%%%%%%%%%%%%%%%%%%%%%%%%%%%%%%%%%%%%%%%%%%%%%%%%%%%
    \newpage
    \newcommand{\enteteresume}{\markboth{Resume}{Resume}} %il faut ajouter les commandes
    \begin{center}
    \textbf{\huge \textsc{\itshape \textit Résumé}}\\
    \end{center}
    \qquad ...............................

    % La table des matieres%%%%%%%%%%%%%%%%%%%%%%%%%%%%%%%%%%%%%%%%%%%%%%%%%%%%%%%%%%%%%%%%%%%%%%%%%%%%%%%%%%%%%%%%%%%%%%%%%%%%%%%%%%%%%%%%%%%%%%%%%%%%%%%%%%%%%
    \tableofcontents
    \listoffigures
    \listoftables

    %% introduction generale%%%%%%%%%%%%%%%%%%%%%%%%%%%%%%%%%%%%%%%%%%%%%%%%%%%%%%%%%%%%%%%%%%%%%%%%%%%%%%%%%%%%%%%%%%%%%%%%%%%%%%%%%%%%%%%%%%%%%%%%%%%%%%%%%%%%%%%%%%%
    \chapter*{General Introduction}
    \addcontentsline{toc}{chapter}{Introduction générale}
    \setcounter{page}{1}
    \lhead{}
    \cfoot{\bfseries \thepage}
    \rhead{Introduction générale}
    \pagenumbering{arabic}

    ..................................................

    %% chapitre 1%%%%%%%%%%%%%%%%%%%%%%%%%%%%%%%%%%%%%%%%%%%%%%%%%%%%%%%%%%%%%%%%%%%%%%%%%%%%%%%%%%%%%%%%%%%%%%%%%%%%%%%%%%%%%%%%%%%%%%%%%%%%%%%%%%%%%%%%%%%%%%%%%%%%%
    %! Author = moulay
%! Date = 10/21/19

% Preamble
\documentclass[english,a4,12pt]{report}

% Packages
\usepackage{amsmath}
\usepackage[utf8]{inputenc}
\usepackage{xcolor}
\usepackage[T1]{fontenc}
\usepackage{arabtex}
\usepackage{sectsty}
\usepackage[cyr]{aeguill}
\usepackage{rotating}
\usepackage{multirow}
\usepackage{tabulary}
\usepackage{tabularht}
\usepackage{acronym}
\usepackage{fancyhdr}
\usepackage{lscape}
\usepackage{amssymb}
\usepackage{pifont}
\usepackage[most]{tcolorbox}
\usepackage{slashbox}
\usepackage{multido}
\usepackage{caption}
\usepackage{graphicx,wrapfig,lipsum}
\usepackage{enumitem}
\usepackage {fancybox}
\usepackage{array,tabularx}
\usepackage{colortbl}
\usepackage[noend]{algorithmic}
\usepackage[linesnumbered,ruled,vlined,boxed,commentsnumbered]{algorithm2e}
\DeclareGraphicsExtensions{.jpg,.pdf,.PNG,.gif}
\usepackage[pdftex,colorlinks=true,linkcolor=black,citecolor=black,urlcolor=black]{hyperref}
\usepackage{pgfplots}
\usepackage{pgfplotstable}
\usepackage{subcaption}
\usepackage[tikz]{bclogo}
\pgfplotsset{grid style={dashed,gray}}
\pgfplotsset{minor grid style={dotted,green!50!black}}
\pgfplotsset{major grid style={dotted,green!50!black}}
\usepackage{anysize}
\marginsize{30mm}{20mm}{15mm}{15mm}
\newcolumntype{Y}{>{\raggedleft\arraybackslash}X}
\renewcommand{\baselinestretch}{1.5}
\newcolumntype{P}[1]{>{\raggedright}p{#1}}
\newcolumntype{M}[1]{>{\raggedright}m{#1}}%%declaration de page de garde
\newcommand*\rfrac[2]{{}^{#1}\!/_{#2}}
\setcounter{secnumdepth}{3}
\setcounter{tocdepth}{3}
\newcommand{\cmark}{\ding{51}}%
\newcommand{\xmark}{\ding{59}}%

% Document
\begin{document}
    \sloppy
    \begin{titlepage}
        \renewcommand{\baselinestretch}{1}
        \begin{center}
            \begin{RLtext}
                AljmhwryT AljzA'iryT AldymqrA.tyT Al^s`byT
            \end{RLtext}
            {PEOPLE'S DEMOCRATIC REPUBLIC OF ALGERIA}
            \begin{RLtext}
            {wzArT Alt`lym Al`Aly w Alb.h_t Al`lmy}
            \end{RLtext}
            {DEPARTMENT OF HIGHER EDUCATION AND SCIENTIFIC RESEARCH}
            \begin{RLtext}
                jAm`T mus.taf_A s.tmbOly bim`skar
            \end{RLtext}
            {UNIVERSITY OF MUSTAPHA STAMBOULI MASCARA}
            \begin{figure}[h]
            	\centering
            		\includegraphics[width=4cm]{figurs/logouniv.jpeg}
            \end{figure}
            \begin{RLtext}
                klyT Al`lwm AldaqyqaT
            \end{RLtext}
            {FACULTY OF EXACT SCIENCES}\\
            {COMPUTER SCIENCE DEPARTMENT}\\
            \textsc{\textbf{ \large Master memory} }
        \end{center}
        {\bfseries field :} Mathematics and Computer Science\\
        {\bfseries Faculty  {\hspace*{0.34cm}} :} Computer Science\\
        {\bfseries Option {\hspace*{0.26cm}} :} Networks and distributed system
        \vspace{0.5cm}
        \begin{center}
            \textsc{\textbf{ \large Realized by :}}\\
            {\bfseries Elkaim Moulay Abdellah }{\hspace*{5cm}}{\bfseries Mazouz  Nail}   \\
            \vspace{1cm}
            \Large {\bfseries Theme }
        \end{center}
        \begin{center}
            \hrule width 460pt
            \bigskip
            \Large  \centering \textbf{ \textsc{ ................. } }
            \bigskip
            \hrule width 460pt
        \end{center}
        \raggedright
        \bigskip
        \vspace{1cm}
        \textit{\bfseries Proposed by : Dr. Madber Hayat }
        \vspace{2cm}
        \begin{center}
            \textit{\bfseries College year 2019/2020}
        \end{center}
    \end{titlepage}

    %%Dedicaces%%%%%%%%%%%%%%%%%%%%%%%%%%%%%%%%%%%%%%%%%%%%%%%%%%%%%%%%%%%%%%%%%%%%%%%%%%%%%%%%%%%%%%%%%%%%%%%%%%%%%%%%%%%%%%%%

    \thispagestyle{empty}
    \begin{figure}[h]
    	\centering
    		\includegraphics[width=14cm]{figurs/B.PNG}
    \end{figure}
    \newpage
    \thispagestyle{empty}
    \clearpage
    \newpage
    \pagenumbering{roman}
    \begin{center}
        \textbf{\\}
        \textbf{\\}
        \textbf{\huge \textsc{\itshape Dedications}}\\
        \textbf{\\}
        \textbf{\\}
        \begin{flushleft}
            \textsf{\qquad I dedicate this work to ............. }\\
        \end{flushleft}
    \end{center}
    \begin{flushright}
        \textbf{\textsc{\itshape Elkaim Moulay Abdellah}}
    \end{flushright}
    \newpage
    \begin{center}
        \textbf{\\}
        \textbf{\\}
        \textbf{\huge \textsc{\itshape Dedications}}\\
        \textbf{\\}
        \textbf{\\}

        \begin{flushleft}
            \textsf{\qquad I dedicate this work to ...................... }
        \end{flushleft}

        \normalsize{\itshape .....}
        \textbf{\\}
        \textbf{\\}
    \end{center}
    \begin{flushright}
        \textbf{\textsc{\itshape Mazouz Nail}}
    \end{flushright}

    %%Remerciement%%%%%%%%%%%%%%%%%%%%%%%%%%%%%%%%%%%%%%%%%%%%%%%%%%%%%%%%%%%%%%%%%%%%%%%%%%%%%%%%%%%%%%%%%%%%%%%%%%%%%%%%%%%%%%%%
    \newpage
    \begin{center}
        %\thispagestyle{myheadings}
        %\markboth{droite}{ }
        \textbf{\huge \textsc{\itshape thanks}}
    \end{center}
    \textsf{\qquad We thank .......... }

    %%Abstract %%%%%%%%%%%%%%%%%%%%%%%%%%%%%%%%%%%%%%%%%%%%%%%%%%%%%%%%%%%%%%%%%%%%%%%%%%%%%%%%%%%%%%%%%%%%%%%%%%%%%%%%%%%%%%%%
    \newpage
    \markboth{droite}{ }
    \begin{center}
    \textbf{\huge \textsc{\itshape \textit Abstract}}\\
    \end{center}
    \qquad ...........................

    %%resume %%%%%%%%%%%%%%%%%%%%%%%%%%%%%%%%%%%%%%%%%%%%%%%%%%%%%%%%%%%%%%%%%%%%%%%%%%%%%%%%%%%%%%%%%%%%%%%%%%%%%%%%%%%%%%%%
    \newpage
    \newcommand{\enteteresume}{\markboth{Resume}{Resume}} %il faut ajouter les commandes
    \begin{center}
    \textbf{\huge \textsc{\itshape \textit Résumé}}\\
    \end{center}
    \qquad ...............................

    % La table des matieres%%%%%%%%%%%%%%%%%%%%%%%%%%%%%%%%%%%%%%%%%%%%%%%%%%%%%%%%%%%%%%%%%%%%%%%%%%%%%%%%%%%%%%%%%%%%%%%%%%%%%%%%%%%%%%%%%%%%%%%%%%%%%%%%%%%%%
    \tableofcontents
    \listoffigures
    \listoftables

    %% introduction generale%%%%%%%%%%%%%%%%%%%%%%%%%%%%%%%%%%%%%%%%%%%%%%%%%%%%%%%%%%%%%%%%%%%%%%%%%%%%%%%%%%%%%%%%%%%%%%%%%%%%%%%%%%%%%%%%%%%%%%%%%%%%%%%%%%%%%%%%%%%
    \chapter*{General Introduction}
    \addcontentsline{toc}{chapter}{Introduction générale}
    \setcounter{page}{1}
    \lhead{}
    \cfoot{\bfseries \thepage}
    \rhead{Introduction générale}
    \pagenumbering{arabic}

    ..................................................

    %% chapitre 1%%%%%%%%%%%%%%%%%%%%%%%%%%%%%%%%%%%%%%%%%%%%%%%%%%%%%%%%%%%%%%%%%%%%%%%%%%%%%%%%%%%%%%%%%%%%%%%%%%%%%%%%%%%%%%%%%%%%%%%%%%%%%%%%%%%%%%%%%%%%%%%%%%%%%
    %! Author = moulay
%! Date = 10/21/19

% Preamble
\documentclass[english,a4,12pt]{report}

% Packages
\usepackage{amsmath}
\usepackage[utf8]{inputenc}
\usepackage{xcolor}
\usepackage[T1]{fontenc}
\usepackage{arabtex}
\usepackage{sectsty}
\usepackage[cyr]{aeguill}
\usepackage{rotating}
\usepackage{multirow}
\usepackage{tabulary}
\usepackage{tabularht}
\usepackage{acronym}
\usepackage{fancyhdr}
\usepackage{lscape}
\usepackage{amssymb}
\usepackage{pifont}
\usepackage[most]{tcolorbox}
\usepackage{slashbox}
\usepackage{multido}
\usepackage{caption}
\usepackage{graphicx,wrapfig,lipsum}
\usepackage{enumitem}
\usepackage {fancybox}
\usepackage{array,tabularx}
\usepackage{colortbl}
\usepackage[noend]{algorithmic}
\usepackage[linesnumbered,ruled,vlined,boxed,commentsnumbered]{algorithm2e}
\DeclareGraphicsExtensions{.jpg,.pdf,.PNG,.gif}
\usepackage[pdftex,colorlinks=true,linkcolor=black,citecolor=black,urlcolor=black]{hyperref}
\usepackage{pgfplots}
\usepackage{pgfplotstable}
\usepackage{subcaption}
\usepackage[tikz]{bclogo}
\pgfplotsset{grid style={dashed,gray}}
\pgfplotsset{minor grid style={dotted,green!50!black}}
\pgfplotsset{major grid style={dotted,green!50!black}}
\usepackage{anysize}
\marginsize{30mm}{20mm}{15mm}{15mm}
\newcolumntype{Y}{>{\raggedleft\arraybackslash}X}
\renewcommand{\baselinestretch}{1.5}
\newcolumntype{P}[1]{>{\raggedright}p{#1}}
\newcolumntype{M}[1]{>{\raggedright}m{#1}}%%declaration de page de garde
\newcommand*\rfrac[2]{{}^{#1}\!/_{#2}}
\setcounter{secnumdepth}{3}
\setcounter{tocdepth}{3}
\newcommand{\cmark}{\ding{51}}%
\newcommand{\xmark}{\ding{59}}%

% Document
\begin{document}
    \sloppy
    \begin{titlepage}
        \renewcommand{\baselinestretch}{1}
        \begin{center}
            \begin{RLtext}
                AljmhwryT AljzA'iryT AldymqrA.tyT Al^s`byT
            \end{RLtext}
            {PEOPLE'S DEMOCRATIC REPUBLIC OF ALGERIA}
            \begin{RLtext}
            {wzArT Alt`lym Al`Aly w Alb.h_t Al`lmy}
            \end{RLtext}
            {DEPARTMENT OF HIGHER EDUCATION AND SCIENTIFIC RESEARCH}
            \begin{RLtext}
                jAm`T mus.taf_A s.tmbOly bim`skar
            \end{RLtext}
            {UNIVERSITY OF MUSTAPHA STAMBOULI MASCARA}
            \begin{figure}[h]
            	\centering
            		\includegraphics[width=4cm]{figurs/logouniv.jpeg}
            \end{figure}
            \begin{RLtext}
                klyT Al`lwm AldaqyqaT
            \end{RLtext}
            {FACULTY OF EXACT SCIENCES}\\
            {COMPUTER SCIENCE DEPARTMENT}\\
            \textsc{\textbf{ \large Master memory} }
        \end{center}
        {\bfseries field :} Mathematics and Computer Science\\
        {\bfseries Faculty  {\hspace*{0.34cm}} :} Computer Science\\
        {\bfseries Option {\hspace*{0.26cm}} :} Networks and distributed system
        \vspace{0.5cm}
        \begin{center}
            \textsc{\textbf{ \large Realized by :}}\\
            {\bfseries Elkaim Moulay Abdellah }{\hspace*{5cm}}{\bfseries Mazouz  Nail}   \\
            \vspace{1cm}
            \Large {\bfseries Theme }
        \end{center}
        \begin{center}
            \hrule width 460pt
            \bigskip
            \Large  \centering \textbf{ \textsc{ ................. } }
            \bigskip
            \hrule width 460pt
        \end{center}
        \raggedright
        \bigskip
        \vspace{1cm}
        \textit{\bfseries Proposed by : Dr. Madber Hayat }
        \vspace{2cm}
        \begin{center}
            \textit{\bfseries College year 2019/2020}
        \end{center}
    \end{titlepage}

    %%Dedicaces%%%%%%%%%%%%%%%%%%%%%%%%%%%%%%%%%%%%%%%%%%%%%%%%%%%%%%%%%%%%%%%%%%%%%%%%%%%%%%%%%%%%%%%%%%%%%%%%%%%%%%%%%%%%%%%%

    \thispagestyle{empty}
    \begin{figure}[h]
    	\centering
    		\includegraphics[width=14cm]{figurs/B.PNG}
    \end{figure}
    \newpage
    \thispagestyle{empty}
    \clearpage
    \newpage
    \pagenumbering{roman}
    \begin{center}
        \textbf{\\}
        \textbf{\\}
        \textbf{\huge \textsc{\itshape Dedications}}\\
        \textbf{\\}
        \textbf{\\}
        \begin{flushleft}
            \textsf{\qquad I dedicate this work to ............. }\\
        \end{flushleft}
    \end{center}
    \begin{flushright}
        \textbf{\textsc{\itshape Elkaim Moulay Abdellah}}
    \end{flushright}
    \newpage
    \begin{center}
        \textbf{\\}
        \textbf{\\}
        \textbf{\huge \textsc{\itshape Dedications}}\\
        \textbf{\\}
        \textbf{\\}

        \begin{flushleft}
            \textsf{\qquad I dedicate this work to ...................... }
        \end{flushleft}

        \normalsize{\itshape .....}
        \textbf{\\}
        \textbf{\\}
    \end{center}
    \begin{flushright}
        \textbf{\textsc{\itshape Mazouz Nail}}
    \end{flushright}

    %%Remerciement%%%%%%%%%%%%%%%%%%%%%%%%%%%%%%%%%%%%%%%%%%%%%%%%%%%%%%%%%%%%%%%%%%%%%%%%%%%%%%%%%%%%%%%%%%%%%%%%%%%%%%%%%%%%%%%%
    \newpage
    \begin{center}
        %\thispagestyle{myheadings}
        %\markboth{droite}{ }
        \textbf{\huge \textsc{\itshape thanks}}
    \end{center}
    \textsf{\qquad We thank .......... }

    %%Abstract %%%%%%%%%%%%%%%%%%%%%%%%%%%%%%%%%%%%%%%%%%%%%%%%%%%%%%%%%%%%%%%%%%%%%%%%%%%%%%%%%%%%%%%%%%%%%%%%%%%%%%%%%%%%%%%%
    \newpage
    \markboth{droite}{ }
    \begin{center}
    \textbf{\huge \textsc{\itshape \textit Abstract}}\\
    \end{center}
    \qquad ...........................

    %%resume %%%%%%%%%%%%%%%%%%%%%%%%%%%%%%%%%%%%%%%%%%%%%%%%%%%%%%%%%%%%%%%%%%%%%%%%%%%%%%%%%%%%%%%%%%%%%%%%%%%%%%%%%%%%%%%%
    \newpage
    \newcommand{\enteteresume}{\markboth{Resume}{Resume}} %il faut ajouter les commandes
    \begin{center}
    \textbf{\huge \textsc{\itshape \textit Résumé}}\\
    \end{center}
    \qquad ...............................

    % La table des matieres%%%%%%%%%%%%%%%%%%%%%%%%%%%%%%%%%%%%%%%%%%%%%%%%%%%%%%%%%%%%%%%%%%%%%%%%%%%%%%%%%%%%%%%%%%%%%%%%%%%%%%%%%%%%%%%%%%%%%%%%%%%%%%%%%%%%%
    \tableofcontents
    \listoffigures
    \listoftables

    %% introduction generale%%%%%%%%%%%%%%%%%%%%%%%%%%%%%%%%%%%%%%%%%%%%%%%%%%%%%%%%%%%%%%%%%%%%%%%%%%%%%%%%%%%%%%%%%%%%%%%%%%%%%%%%%%%%%%%%%%%%%%%%%%%%%%%%%%%%%%%%%%%
    \chapter*{General Introduction}
    \addcontentsline{toc}{chapter}{Introduction générale}
    \setcounter{page}{1}
    \lhead{}
    \cfoot{\bfseries \thepage}
    \rhead{Introduction générale}
    \pagenumbering{arabic}

    ..................................................

    %% chapitre 1%%%%%%%%%%%%%%%%%%%%%%%%%%%%%%%%%%%%%%%%%%%%%%%%%%%%%%%%%%%%%%%%%%%%%%%%%%%%%%%%%%%%%%%%%%%%%%%%%%%%%%%%%%%%%%%%%%%%%%%%%%%%%%%%%%%%%%%%%%%%%%%%%%%%%
    \include{chapiter1/main}

    %% chapitre 2%%%%%%%%%%%%%%%%%%%%%%%%%%%%%%%%%%%%%%%%%%%%%%%%%%%%%%%%%%%%%%%%%%%%%%%%%%%%%%%%%%%%%%%%%%%%%%%%%%%%%%%%%%%%%%%%%%%%%%%%%%%%%%%%%%%%%%%%%%%%%%%%%%%%%
    \include{chapiter2/main}

    %% conclusion generale%%%%%%%%%%%%%%%%%%%%%%%%%%%%%%%%%%%%%%%%%%%%%%%%%%%%%%%%%%%%%%%%%%%%%%%%%%%%%%%%%%%%%%%%%%%%%%%%%%%%%%%%%%%%%%%%%%%%%%%%%%%%%%%%%%%%%%%%%%%
    \chapter*{General conclusion}
    \addcontentsline{toc}{chapter}{General conclusion}
    \lhead{}
    \cfoot{\bfseries \thepage}
    \rhead{General conclusion}
    \markboth{droite}{General conclusion}

    .............................................

    %%References%%%%%%%%%%%%%%%%%%%%%%%%%%%%%%%%%%%%%%%%%%%%%%%%%%%%%%%%%%%%%%%%%%%%%%%%%%%%%%%%%%%%%%%%%%%%%%%%%%%%%%%%%%%%%%%%%%%%%%%%%%%%%%
    \clearpage
    \pagestyle{fancy}
    \addcontentsline{toc}{chapter}{Bibliography}
    \begin{thebibliography}{99}
        \lhead{}
        \cfoot{\bfseries \thepage}
        \rhead{Bibliography}

        \bibitem[1]{1}
        H. Altaama,
        Application Mobile Guide,
        Mémoire de Master en Informatique,
        Université Abou Bakr Belkaid de Tlemcen,
        2016.
        \bibitem[2]{2}
        \url{http://generationmobiles.net/2014/11/les-differents-types-dapps-mobiles/}, [consulté le 14/03/2018].

    \end{thebibliography}

\end{document}

    %% chapitre 2%%%%%%%%%%%%%%%%%%%%%%%%%%%%%%%%%%%%%%%%%%%%%%%%%%%%%%%%%%%%%%%%%%%%%%%%%%%%%%%%%%%%%%%%%%%%%%%%%%%%%%%%%%%%%%%%%%%%%%%%%%%%%%%%%%%%%%%%%%%%%%%%%%%%%
    %! Author = moulay
%! Date = 10/21/19

% Preamble
\documentclass[english,a4,12pt]{report}

% Packages
\usepackage{amsmath}
\usepackage[utf8]{inputenc}
\usepackage{xcolor}
\usepackage[T1]{fontenc}
\usepackage{arabtex}
\usepackage{sectsty}
\usepackage[cyr]{aeguill}
\usepackage{rotating}
\usepackage{multirow}
\usepackage{tabulary}
\usepackage{tabularht}
\usepackage{acronym}
\usepackage{fancyhdr}
\usepackage{lscape}
\usepackage{amssymb}
\usepackage{pifont}
\usepackage[most]{tcolorbox}
\usepackage{slashbox}
\usepackage{multido}
\usepackage{caption}
\usepackage{graphicx,wrapfig,lipsum}
\usepackage{enumitem}
\usepackage {fancybox}
\usepackage{array,tabularx}
\usepackage{colortbl}
\usepackage[noend]{algorithmic}
\usepackage[linesnumbered,ruled,vlined,boxed,commentsnumbered]{algorithm2e}
\DeclareGraphicsExtensions{.jpg,.pdf,.PNG,.gif}
\usepackage[pdftex,colorlinks=true,linkcolor=black,citecolor=black,urlcolor=black]{hyperref}
\usepackage{pgfplots}
\usepackage{pgfplotstable}
\usepackage{subcaption}
\usepackage[tikz]{bclogo}
\pgfplotsset{grid style={dashed,gray}}
\pgfplotsset{minor grid style={dotted,green!50!black}}
\pgfplotsset{major grid style={dotted,green!50!black}}
\usepackage{anysize}
\marginsize{30mm}{20mm}{15mm}{15mm}
\newcolumntype{Y}{>{\raggedleft\arraybackslash}X}
\renewcommand{\baselinestretch}{1.5}
\newcolumntype{P}[1]{>{\raggedright}p{#1}}
\newcolumntype{M}[1]{>{\raggedright}m{#1}}%%declaration de page de garde
\newcommand*\rfrac[2]{{}^{#1}\!/_{#2}}
\setcounter{secnumdepth}{3}
\setcounter{tocdepth}{3}
\newcommand{\cmark}{\ding{51}}%
\newcommand{\xmark}{\ding{59}}%

% Document
\begin{document}
    \sloppy
    \begin{titlepage}
        \renewcommand{\baselinestretch}{1}
        \begin{center}
            \begin{RLtext}
                AljmhwryT AljzA'iryT AldymqrA.tyT Al^s`byT
            \end{RLtext}
            {PEOPLE'S DEMOCRATIC REPUBLIC OF ALGERIA}
            \begin{RLtext}
            {wzArT Alt`lym Al`Aly w Alb.h_t Al`lmy}
            \end{RLtext}
            {DEPARTMENT OF HIGHER EDUCATION AND SCIENTIFIC RESEARCH}
            \begin{RLtext}
                jAm`T mus.taf_A s.tmbOly bim`skar
            \end{RLtext}
            {UNIVERSITY OF MUSTAPHA STAMBOULI MASCARA}
            \begin{figure}[h]
            	\centering
            		\includegraphics[width=4cm]{figurs/logouniv.jpeg}
            \end{figure}
            \begin{RLtext}
                klyT Al`lwm AldaqyqaT
            \end{RLtext}
            {FACULTY OF EXACT SCIENCES}\\
            {COMPUTER SCIENCE DEPARTMENT}\\
            \textsc{\textbf{ \large Master memory} }
        \end{center}
        {\bfseries field :} Mathematics and Computer Science\\
        {\bfseries Faculty  {\hspace*{0.34cm}} :} Computer Science\\
        {\bfseries Option {\hspace*{0.26cm}} :} Networks and distributed system
        \vspace{0.5cm}
        \begin{center}
            \textsc{\textbf{ \large Realized by :}}\\
            {\bfseries Elkaim Moulay Abdellah }{\hspace*{5cm}}{\bfseries Mazouz  Nail}   \\
            \vspace{1cm}
            \Large {\bfseries Theme }
        \end{center}
        \begin{center}
            \hrule width 460pt
            \bigskip
            \Large  \centering \textbf{ \textsc{ ................. } }
            \bigskip
            \hrule width 460pt
        \end{center}
        \raggedright
        \bigskip
        \vspace{1cm}
        \textit{\bfseries Proposed by : Dr. Madber Hayat }
        \vspace{2cm}
        \begin{center}
            \textit{\bfseries College year 2019/2020}
        \end{center}
    \end{titlepage}

    %%Dedicaces%%%%%%%%%%%%%%%%%%%%%%%%%%%%%%%%%%%%%%%%%%%%%%%%%%%%%%%%%%%%%%%%%%%%%%%%%%%%%%%%%%%%%%%%%%%%%%%%%%%%%%%%%%%%%%%%

    \thispagestyle{empty}
    \begin{figure}[h]
    	\centering
    		\includegraphics[width=14cm]{figurs/B.PNG}
    \end{figure}
    \newpage
    \thispagestyle{empty}
    \clearpage
    \newpage
    \pagenumbering{roman}
    \begin{center}
        \textbf{\\}
        \textbf{\\}
        \textbf{\huge \textsc{\itshape Dedications}}\\
        \textbf{\\}
        \textbf{\\}
        \begin{flushleft}
            \textsf{\qquad I dedicate this work to ............. }\\
        \end{flushleft}
    \end{center}
    \begin{flushright}
        \textbf{\textsc{\itshape Elkaim Moulay Abdellah}}
    \end{flushright}
    \newpage
    \begin{center}
        \textbf{\\}
        \textbf{\\}
        \textbf{\huge \textsc{\itshape Dedications}}\\
        \textbf{\\}
        \textbf{\\}

        \begin{flushleft}
            \textsf{\qquad I dedicate this work to ...................... }
        \end{flushleft}

        \normalsize{\itshape .....}
        \textbf{\\}
        \textbf{\\}
    \end{center}
    \begin{flushright}
        \textbf{\textsc{\itshape Mazouz Nail}}
    \end{flushright}

    %%Remerciement%%%%%%%%%%%%%%%%%%%%%%%%%%%%%%%%%%%%%%%%%%%%%%%%%%%%%%%%%%%%%%%%%%%%%%%%%%%%%%%%%%%%%%%%%%%%%%%%%%%%%%%%%%%%%%%%
    \newpage
    \begin{center}
        %\thispagestyle{myheadings}
        %\markboth{droite}{ }
        \textbf{\huge \textsc{\itshape thanks}}
    \end{center}
    \textsf{\qquad We thank .......... }

    %%Abstract %%%%%%%%%%%%%%%%%%%%%%%%%%%%%%%%%%%%%%%%%%%%%%%%%%%%%%%%%%%%%%%%%%%%%%%%%%%%%%%%%%%%%%%%%%%%%%%%%%%%%%%%%%%%%%%%
    \newpage
    \markboth{droite}{ }
    \begin{center}
    \textbf{\huge \textsc{\itshape \textit Abstract}}\\
    \end{center}
    \qquad ...........................

    %%resume %%%%%%%%%%%%%%%%%%%%%%%%%%%%%%%%%%%%%%%%%%%%%%%%%%%%%%%%%%%%%%%%%%%%%%%%%%%%%%%%%%%%%%%%%%%%%%%%%%%%%%%%%%%%%%%%
    \newpage
    \newcommand{\enteteresume}{\markboth{Resume}{Resume}} %il faut ajouter les commandes
    \begin{center}
    \textbf{\huge \textsc{\itshape \textit Résumé}}\\
    \end{center}
    \qquad ...............................

    % La table des matieres%%%%%%%%%%%%%%%%%%%%%%%%%%%%%%%%%%%%%%%%%%%%%%%%%%%%%%%%%%%%%%%%%%%%%%%%%%%%%%%%%%%%%%%%%%%%%%%%%%%%%%%%%%%%%%%%%%%%%%%%%%%%%%%%%%%%%
    \tableofcontents
    \listoffigures
    \listoftables

    %% introduction generale%%%%%%%%%%%%%%%%%%%%%%%%%%%%%%%%%%%%%%%%%%%%%%%%%%%%%%%%%%%%%%%%%%%%%%%%%%%%%%%%%%%%%%%%%%%%%%%%%%%%%%%%%%%%%%%%%%%%%%%%%%%%%%%%%%%%%%%%%%%
    \chapter*{General Introduction}
    \addcontentsline{toc}{chapter}{Introduction générale}
    \setcounter{page}{1}
    \lhead{}
    \cfoot{\bfseries \thepage}
    \rhead{Introduction générale}
    \pagenumbering{arabic}

    ..................................................

    %% chapitre 1%%%%%%%%%%%%%%%%%%%%%%%%%%%%%%%%%%%%%%%%%%%%%%%%%%%%%%%%%%%%%%%%%%%%%%%%%%%%%%%%%%%%%%%%%%%%%%%%%%%%%%%%%%%%%%%%%%%%%%%%%%%%%%%%%%%%%%%%%%%%%%%%%%%%%
    \include{chapiter1/main}

    %% chapitre 2%%%%%%%%%%%%%%%%%%%%%%%%%%%%%%%%%%%%%%%%%%%%%%%%%%%%%%%%%%%%%%%%%%%%%%%%%%%%%%%%%%%%%%%%%%%%%%%%%%%%%%%%%%%%%%%%%%%%%%%%%%%%%%%%%%%%%%%%%%%%%%%%%%%%%
    \include{chapiter2/main}

    %% conclusion generale%%%%%%%%%%%%%%%%%%%%%%%%%%%%%%%%%%%%%%%%%%%%%%%%%%%%%%%%%%%%%%%%%%%%%%%%%%%%%%%%%%%%%%%%%%%%%%%%%%%%%%%%%%%%%%%%%%%%%%%%%%%%%%%%%%%%%%%%%%%
    \chapter*{General conclusion}
    \addcontentsline{toc}{chapter}{General conclusion}
    \lhead{}
    \cfoot{\bfseries \thepage}
    \rhead{General conclusion}
    \markboth{droite}{General conclusion}

    .............................................

    %%References%%%%%%%%%%%%%%%%%%%%%%%%%%%%%%%%%%%%%%%%%%%%%%%%%%%%%%%%%%%%%%%%%%%%%%%%%%%%%%%%%%%%%%%%%%%%%%%%%%%%%%%%%%%%%%%%%%%%%%%%%%%%%%
    \clearpage
    \pagestyle{fancy}
    \addcontentsline{toc}{chapter}{Bibliography}
    \begin{thebibliography}{99}
        \lhead{}
        \cfoot{\bfseries \thepage}
        \rhead{Bibliography}

        \bibitem[1]{1}
        H. Altaama,
        Application Mobile Guide,
        Mémoire de Master en Informatique,
        Université Abou Bakr Belkaid de Tlemcen,
        2016.
        \bibitem[2]{2}
        \url{http://generationmobiles.net/2014/11/les-differents-types-dapps-mobiles/}, [consulté le 14/03/2018].

    \end{thebibliography}

\end{document}

    %% conclusion generale%%%%%%%%%%%%%%%%%%%%%%%%%%%%%%%%%%%%%%%%%%%%%%%%%%%%%%%%%%%%%%%%%%%%%%%%%%%%%%%%%%%%%%%%%%%%%%%%%%%%%%%%%%%%%%%%%%%%%%%%%%%%%%%%%%%%%%%%%%%
    \chapter*{General conclusion}
    \addcontentsline{toc}{chapter}{General conclusion}
    \lhead{}
    \cfoot{\bfseries \thepage}
    \rhead{General conclusion}
    \markboth{droite}{General conclusion}

    .............................................

    %%References%%%%%%%%%%%%%%%%%%%%%%%%%%%%%%%%%%%%%%%%%%%%%%%%%%%%%%%%%%%%%%%%%%%%%%%%%%%%%%%%%%%%%%%%%%%%%%%%%%%%%%%%%%%%%%%%%%%%%%%%%%%%%%
    \clearpage
    \pagestyle{fancy}
    \addcontentsline{toc}{chapter}{Bibliography}
    \begin{thebibliography}{99}
        \lhead{}
        \cfoot{\bfseries \thepage}
        \rhead{Bibliography}

        \bibitem[1]{1}
        H. Altaama,
        Application Mobile Guide,
        Mémoire de Master en Informatique,
        Université Abou Bakr Belkaid de Tlemcen,
        2016.
        \bibitem[2]{2}
        \url{http://generationmobiles.net/2014/11/les-differents-types-dapps-mobiles/}, [consulté le 14/03/2018].

    \end{thebibliography}

\end{document}

    %% chapitre 2%%%%%%%%%%%%%%%%%%%%%%%%%%%%%%%%%%%%%%%%%%%%%%%%%%%%%%%%%%%%%%%%%%%%%%%%%%%%%%%%%%%%%%%%%%%%%%%%%%%%%%%%%%%%%%%%%%%%%%%%%%%%%%%%%%%%%%%%%%%%%%%%%%%%%
    %! Author = moulay
%! Date = 10/21/19

% Preamble
\documentclass[english,a4,12pt]{report}

% Packages
\usepackage{amsmath}
\usepackage[utf8]{inputenc}
\usepackage{xcolor}
\usepackage[T1]{fontenc}
\usepackage{arabtex}
\usepackage{sectsty}
\usepackage[cyr]{aeguill}
\usepackage{rotating}
\usepackage{multirow}
\usepackage{tabulary}
\usepackage{tabularht}
\usepackage{acronym}
\usepackage{fancyhdr}
\usepackage{lscape}
\usepackage{amssymb}
\usepackage{pifont}
\usepackage[most]{tcolorbox}
\usepackage{slashbox}
\usepackage{multido}
\usepackage{caption}
\usepackage{graphicx,wrapfig,lipsum}
\usepackage{enumitem}
\usepackage {fancybox}
\usepackage{array,tabularx}
\usepackage{colortbl}
\usepackage[noend]{algorithmic}
\usepackage[linesnumbered,ruled,vlined,boxed,commentsnumbered]{algorithm2e}
\DeclareGraphicsExtensions{.jpg,.pdf,.PNG,.gif}
\usepackage[pdftex,colorlinks=true,linkcolor=black,citecolor=black,urlcolor=black]{hyperref}
\usepackage{pgfplots}
\usepackage{pgfplotstable}
\usepackage{subcaption}
\usepackage[tikz]{bclogo}
\pgfplotsset{grid style={dashed,gray}}
\pgfplotsset{minor grid style={dotted,green!50!black}}
\pgfplotsset{major grid style={dotted,green!50!black}}
\usepackage{anysize}
\marginsize{30mm}{20mm}{15mm}{15mm}
\newcolumntype{Y}{>{\raggedleft\arraybackslash}X}
\renewcommand{\baselinestretch}{1.5}
\newcolumntype{P}[1]{>{\raggedright}p{#1}}
\newcolumntype{M}[1]{>{\raggedright}m{#1}}%%declaration de page de garde
\newcommand*\rfrac[2]{{}^{#1}\!/_{#2}}
\setcounter{secnumdepth}{3}
\setcounter{tocdepth}{3}
\newcommand{\cmark}{\ding{51}}%
\newcommand{\xmark}{\ding{59}}%

% Document
\begin{document}
    \sloppy
    \begin{titlepage}
        \renewcommand{\baselinestretch}{1}
        \begin{center}
            \begin{RLtext}
                AljmhwryT AljzA'iryT AldymqrA.tyT Al^s`byT
            \end{RLtext}
            {PEOPLE'S DEMOCRATIC REPUBLIC OF ALGERIA}
            \begin{RLtext}
            {wzArT Alt`lym Al`Aly w Alb.h_t Al`lmy}
            \end{RLtext}
            {DEPARTMENT OF HIGHER EDUCATION AND SCIENTIFIC RESEARCH}
            \begin{RLtext}
                jAm`T mus.taf_A s.tmbOly bim`skar
            \end{RLtext}
            {UNIVERSITY OF MUSTAPHA STAMBOULI MASCARA}
            \begin{figure}[h]
            	\centering
            		\includegraphics[width=4cm]{figurs/logouniv.jpeg}
            \end{figure}
            \begin{RLtext}
                klyT Al`lwm AldaqyqaT
            \end{RLtext}
            {FACULTY OF EXACT SCIENCES}\\
            {COMPUTER SCIENCE DEPARTMENT}\\
            \textsc{\textbf{ \large Master memory} }
        \end{center}
        {\bfseries field :} Mathematics and Computer Science\\
        {\bfseries Faculty  {\hspace*{0.34cm}} :} Computer Science\\
        {\bfseries Option {\hspace*{0.26cm}} :} Networks and distributed system
        \vspace{0.5cm}
        \begin{center}
            \textsc{\textbf{ \large Realized by :}}\\
            {\bfseries Elkaim Moulay Abdellah }{\hspace*{5cm}}{\bfseries Mazouz  Nail}   \\
            \vspace{1cm}
            \Large {\bfseries Theme }
        \end{center}
        \begin{center}
            \hrule width 460pt
            \bigskip
            \Large  \centering \textbf{ \textsc{ ................. } }
            \bigskip
            \hrule width 460pt
        \end{center}
        \raggedright
        \bigskip
        \vspace{1cm}
        \textit{\bfseries Proposed by : Dr. Madber Hayat }
        \vspace{2cm}
        \begin{center}
            \textit{\bfseries College year 2019/2020}
        \end{center}
    \end{titlepage}

    %%Dedicaces%%%%%%%%%%%%%%%%%%%%%%%%%%%%%%%%%%%%%%%%%%%%%%%%%%%%%%%%%%%%%%%%%%%%%%%%%%%%%%%%%%%%%%%%%%%%%%%%%%%%%%%%%%%%%%%%

    \thispagestyle{empty}
    \begin{figure}[h]
    	\centering
    		\includegraphics[width=14cm]{figurs/B.PNG}
    \end{figure}
    \newpage
    \thispagestyle{empty}
    \clearpage
    \newpage
    \pagenumbering{roman}
    \begin{center}
        \textbf{\\}
        \textbf{\\}
        \textbf{\huge \textsc{\itshape Dedications}}\\
        \textbf{\\}
        \textbf{\\}
        \begin{flushleft}
            \textsf{\qquad I dedicate this work to ............. }\\
        \end{flushleft}
    \end{center}
    \begin{flushright}
        \textbf{\textsc{\itshape Elkaim Moulay Abdellah}}
    \end{flushright}
    \newpage
    \begin{center}
        \textbf{\\}
        \textbf{\\}
        \textbf{\huge \textsc{\itshape Dedications}}\\
        \textbf{\\}
        \textbf{\\}

        \begin{flushleft}
            \textsf{\qquad I dedicate this work to ...................... }
        \end{flushleft}

        \normalsize{\itshape .....}
        \textbf{\\}
        \textbf{\\}
    \end{center}
    \begin{flushright}
        \textbf{\textsc{\itshape Mazouz Nail}}
    \end{flushright}

    %%Remerciement%%%%%%%%%%%%%%%%%%%%%%%%%%%%%%%%%%%%%%%%%%%%%%%%%%%%%%%%%%%%%%%%%%%%%%%%%%%%%%%%%%%%%%%%%%%%%%%%%%%%%%%%%%%%%%%%
    \newpage
    \begin{center}
        %\thispagestyle{myheadings}
        %\markboth{droite}{ }
        \textbf{\huge \textsc{\itshape thanks}}
    \end{center}
    \textsf{\qquad We thank .......... }

    %%Abstract %%%%%%%%%%%%%%%%%%%%%%%%%%%%%%%%%%%%%%%%%%%%%%%%%%%%%%%%%%%%%%%%%%%%%%%%%%%%%%%%%%%%%%%%%%%%%%%%%%%%%%%%%%%%%%%%
    \newpage
    \markboth{droite}{ }
    \begin{center}
    \textbf{\huge \textsc{\itshape \textit Abstract}}\\
    \end{center}
    \qquad ...........................

    %%resume %%%%%%%%%%%%%%%%%%%%%%%%%%%%%%%%%%%%%%%%%%%%%%%%%%%%%%%%%%%%%%%%%%%%%%%%%%%%%%%%%%%%%%%%%%%%%%%%%%%%%%%%%%%%%%%%
    \newpage
    \newcommand{\enteteresume}{\markboth{Resume}{Resume}} %il faut ajouter les commandes
    \begin{center}
    \textbf{\huge \textsc{\itshape \textit Résumé}}\\
    \end{center}
    \qquad ...............................

    % La table des matieres%%%%%%%%%%%%%%%%%%%%%%%%%%%%%%%%%%%%%%%%%%%%%%%%%%%%%%%%%%%%%%%%%%%%%%%%%%%%%%%%%%%%%%%%%%%%%%%%%%%%%%%%%%%%%%%%%%%%%%%%%%%%%%%%%%%%%
    \tableofcontents
    \listoffigures
    \listoftables

    %% introduction generale%%%%%%%%%%%%%%%%%%%%%%%%%%%%%%%%%%%%%%%%%%%%%%%%%%%%%%%%%%%%%%%%%%%%%%%%%%%%%%%%%%%%%%%%%%%%%%%%%%%%%%%%%%%%%%%%%%%%%%%%%%%%%%%%%%%%%%%%%%%
    \chapter*{General Introduction}
    \addcontentsline{toc}{chapter}{Introduction générale}
    \setcounter{page}{1}
    \lhead{}
    \cfoot{\bfseries \thepage}
    \rhead{Introduction générale}
    \pagenumbering{arabic}

    ..................................................

    %% chapitre 1%%%%%%%%%%%%%%%%%%%%%%%%%%%%%%%%%%%%%%%%%%%%%%%%%%%%%%%%%%%%%%%%%%%%%%%%%%%%%%%%%%%%%%%%%%%%%%%%%%%%%%%%%%%%%%%%%%%%%%%%%%%%%%%%%%%%%%%%%%%%%%%%%%%%%
    %! Author = moulay
%! Date = 10/21/19

% Preamble
\documentclass[english,a4,12pt]{report}

% Packages
\usepackage{amsmath}
\usepackage[utf8]{inputenc}
\usepackage{xcolor}
\usepackage[T1]{fontenc}
\usepackage{arabtex}
\usepackage{sectsty}
\usepackage[cyr]{aeguill}
\usepackage{rotating}
\usepackage{multirow}
\usepackage{tabulary}
\usepackage{tabularht}
\usepackage{acronym}
\usepackage{fancyhdr}
\usepackage{lscape}
\usepackage{amssymb}
\usepackage{pifont}
\usepackage[most]{tcolorbox}
\usepackage{slashbox}
\usepackage{multido}
\usepackage{caption}
\usepackage{graphicx,wrapfig,lipsum}
\usepackage{enumitem}
\usepackage {fancybox}
\usepackage{array,tabularx}
\usepackage{colortbl}
\usepackage[noend]{algorithmic}
\usepackage[linesnumbered,ruled,vlined,boxed,commentsnumbered]{algorithm2e}
\DeclareGraphicsExtensions{.jpg,.pdf,.PNG,.gif}
\usepackage[pdftex,colorlinks=true,linkcolor=black,citecolor=black,urlcolor=black]{hyperref}
\usepackage{pgfplots}
\usepackage{pgfplotstable}
\usepackage{subcaption}
\usepackage[tikz]{bclogo}
\pgfplotsset{grid style={dashed,gray}}
\pgfplotsset{minor grid style={dotted,green!50!black}}
\pgfplotsset{major grid style={dotted,green!50!black}}
\usepackage{anysize}
\marginsize{30mm}{20mm}{15mm}{15mm}
\newcolumntype{Y}{>{\raggedleft\arraybackslash}X}
\renewcommand{\baselinestretch}{1.5}
\newcolumntype{P}[1]{>{\raggedright}p{#1}}
\newcolumntype{M}[1]{>{\raggedright}m{#1}}%%declaration de page de garde
\newcommand*\rfrac[2]{{}^{#1}\!/_{#2}}
\setcounter{secnumdepth}{3}
\setcounter{tocdepth}{3}
\newcommand{\cmark}{\ding{51}}%
\newcommand{\xmark}{\ding{59}}%

% Document
\begin{document}
    \sloppy
    \begin{titlepage}
        \renewcommand{\baselinestretch}{1}
        \begin{center}
            \begin{RLtext}
                AljmhwryT AljzA'iryT AldymqrA.tyT Al^s`byT
            \end{RLtext}
            {PEOPLE'S DEMOCRATIC REPUBLIC OF ALGERIA}
            \begin{RLtext}
            {wzArT Alt`lym Al`Aly w Alb.h_t Al`lmy}
            \end{RLtext}
            {DEPARTMENT OF HIGHER EDUCATION AND SCIENTIFIC RESEARCH}
            \begin{RLtext}
                jAm`T mus.taf_A s.tmbOly bim`skar
            \end{RLtext}
            {UNIVERSITY OF MUSTAPHA STAMBOULI MASCARA}
            \begin{figure}[h]
            	\centering
            		\includegraphics[width=4cm]{figurs/logouniv.jpeg}
            \end{figure}
            \begin{RLtext}
                klyT Al`lwm AldaqyqaT
            \end{RLtext}
            {FACULTY OF EXACT SCIENCES}\\
            {COMPUTER SCIENCE DEPARTMENT}\\
            \textsc{\textbf{ \large Master memory} }
        \end{center}
        {\bfseries field :} Mathematics and Computer Science\\
        {\bfseries Faculty  {\hspace*{0.34cm}} :} Computer Science\\
        {\bfseries Option {\hspace*{0.26cm}} :} Networks and distributed system
        \vspace{0.5cm}
        \begin{center}
            \textsc{\textbf{ \large Realized by :}}\\
            {\bfseries Elkaim Moulay Abdellah }{\hspace*{5cm}}{\bfseries Mazouz  Nail}   \\
            \vspace{1cm}
            \Large {\bfseries Theme }
        \end{center}
        \begin{center}
            \hrule width 460pt
            \bigskip
            \Large  \centering \textbf{ \textsc{ ................. } }
            \bigskip
            \hrule width 460pt
        \end{center}
        \raggedright
        \bigskip
        \vspace{1cm}
        \textit{\bfseries Proposed by : Dr. Madber Hayat }
        \vspace{2cm}
        \begin{center}
            \textit{\bfseries College year 2019/2020}
        \end{center}
    \end{titlepage}

    %%Dedicaces%%%%%%%%%%%%%%%%%%%%%%%%%%%%%%%%%%%%%%%%%%%%%%%%%%%%%%%%%%%%%%%%%%%%%%%%%%%%%%%%%%%%%%%%%%%%%%%%%%%%%%%%%%%%%%%%

    \thispagestyle{empty}
    \begin{figure}[h]
    	\centering
    		\includegraphics[width=14cm]{figurs/B.PNG}
    \end{figure}
    \newpage
    \thispagestyle{empty}
    \clearpage
    \newpage
    \pagenumbering{roman}
    \begin{center}
        \textbf{\\}
        \textbf{\\}
        \textbf{\huge \textsc{\itshape Dedications}}\\
        \textbf{\\}
        \textbf{\\}
        \begin{flushleft}
            \textsf{\qquad I dedicate this work to ............. }\\
        \end{flushleft}
    \end{center}
    \begin{flushright}
        \textbf{\textsc{\itshape Elkaim Moulay Abdellah}}
    \end{flushright}
    \newpage
    \begin{center}
        \textbf{\\}
        \textbf{\\}
        \textbf{\huge \textsc{\itshape Dedications}}\\
        \textbf{\\}
        \textbf{\\}

        \begin{flushleft}
            \textsf{\qquad I dedicate this work to ...................... }
        \end{flushleft}

        \normalsize{\itshape .....}
        \textbf{\\}
        \textbf{\\}
    \end{center}
    \begin{flushright}
        \textbf{\textsc{\itshape Mazouz Nail}}
    \end{flushright}

    %%Remerciement%%%%%%%%%%%%%%%%%%%%%%%%%%%%%%%%%%%%%%%%%%%%%%%%%%%%%%%%%%%%%%%%%%%%%%%%%%%%%%%%%%%%%%%%%%%%%%%%%%%%%%%%%%%%%%%%
    \newpage
    \begin{center}
        %\thispagestyle{myheadings}
        %\markboth{droite}{ }
        \textbf{\huge \textsc{\itshape thanks}}
    \end{center}
    \textsf{\qquad We thank .......... }

    %%Abstract %%%%%%%%%%%%%%%%%%%%%%%%%%%%%%%%%%%%%%%%%%%%%%%%%%%%%%%%%%%%%%%%%%%%%%%%%%%%%%%%%%%%%%%%%%%%%%%%%%%%%%%%%%%%%%%%
    \newpage
    \markboth{droite}{ }
    \begin{center}
    \textbf{\huge \textsc{\itshape \textit Abstract}}\\
    \end{center}
    \qquad ...........................

    %%resume %%%%%%%%%%%%%%%%%%%%%%%%%%%%%%%%%%%%%%%%%%%%%%%%%%%%%%%%%%%%%%%%%%%%%%%%%%%%%%%%%%%%%%%%%%%%%%%%%%%%%%%%%%%%%%%%
    \newpage
    \newcommand{\enteteresume}{\markboth{Resume}{Resume}} %il faut ajouter les commandes
    \begin{center}
    \textbf{\huge \textsc{\itshape \textit Résumé}}\\
    \end{center}
    \qquad ...............................

    % La table des matieres%%%%%%%%%%%%%%%%%%%%%%%%%%%%%%%%%%%%%%%%%%%%%%%%%%%%%%%%%%%%%%%%%%%%%%%%%%%%%%%%%%%%%%%%%%%%%%%%%%%%%%%%%%%%%%%%%%%%%%%%%%%%%%%%%%%%%
    \tableofcontents
    \listoffigures
    \listoftables

    %% introduction generale%%%%%%%%%%%%%%%%%%%%%%%%%%%%%%%%%%%%%%%%%%%%%%%%%%%%%%%%%%%%%%%%%%%%%%%%%%%%%%%%%%%%%%%%%%%%%%%%%%%%%%%%%%%%%%%%%%%%%%%%%%%%%%%%%%%%%%%%%%%
    \chapter*{General Introduction}
    \addcontentsline{toc}{chapter}{Introduction générale}
    \setcounter{page}{1}
    \lhead{}
    \cfoot{\bfseries \thepage}
    \rhead{Introduction générale}
    \pagenumbering{arabic}

    ..................................................

    %% chapitre 1%%%%%%%%%%%%%%%%%%%%%%%%%%%%%%%%%%%%%%%%%%%%%%%%%%%%%%%%%%%%%%%%%%%%%%%%%%%%%%%%%%%%%%%%%%%%%%%%%%%%%%%%%%%%%%%%%%%%%%%%%%%%%%%%%%%%%%%%%%%%%%%%%%%%%
    \include{chapiter1/main}

    %% chapitre 2%%%%%%%%%%%%%%%%%%%%%%%%%%%%%%%%%%%%%%%%%%%%%%%%%%%%%%%%%%%%%%%%%%%%%%%%%%%%%%%%%%%%%%%%%%%%%%%%%%%%%%%%%%%%%%%%%%%%%%%%%%%%%%%%%%%%%%%%%%%%%%%%%%%%%
    \include{chapiter2/main}

    %% conclusion generale%%%%%%%%%%%%%%%%%%%%%%%%%%%%%%%%%%%%%%%%%%%%%%%%%%%%%%%%%%%%%%%%%%%%%%%%%%%%%%%%%%%%%%%%%%%%%%%%%%%%%%%%%%%%%%%%%%%%%%%%%%%%%%%%%%%%%%%%%%%
    \chapter*{General conclusion}
    \addcontentsline{toc}{chapter}{General conclusion}
    \lhead{}
    \cfoot{\bfseries \thepage}
    \rhead{General conclusion}
    \markboth{droite}{General conclusion}

    .............................................

    %%References%%%%%%%%%%%%%%%%%%%%%%%%%%%%%%%%%%%%%%%%%%%%%%%%%%%%%%%%%%%%%%%%%%%%%%%%%%%%%%%%%%%%%%%%%%%%%%%%%%%%%%%%%%%%%%%%%%%%%%%%%%%%%%
    \clearpage
    \pagestyle{fancy}
    \addcontentsline{toc}{chapter}{Bibliography}
    \begin{thebibliography}{99}
        \lhead{}
        \cfoot{\bfseries \thepage}
        \rhead{Bibliography}

        \bibitem[1]{1}
        H. Altaama,
        Application Mobile Guide,
        Mémoire de Master en Informatique,
        Université Abou Bakr Belkaid de Tlemcen,
        2016.
        \bibitem[2]{2}
        \url{http://generationmobiles.net/2014/11/les-differents-types-dapps-mobiles/}, [consulté le 14/03/2018].

    \end{thebibliography}

\end{document}

    %% chapitre 2%%%%%%%%%%%%%%%%%%%%%%%%%%%%%%%%%%%%%%%%%%%%%%%%%%%%%%%%%%%%%%%%%%%%%%%%%%%%%%%%%%%%%%%%%%%%%%%%%%%%%%%%%%%%%%%%%%%%%%%%%%%%%%%%%%%%%%%%%%%%%%%%%%%%%
    %! Author = moulay
%! Date = 10/21/19

% Preamble
\documentclass[english,a4,12pt]{report}

% Packages
\usepackage{amsmath}
\usepackage[utf8]{inputenc}
\usepackage{xcolor}
\usepackage[T1]{fontenc}
\usepackage{arabtex}
\usepackage{sectsty}
\usepackage[cyr]{aeguill}
\usepackage{rotating}
\usepackage{multirow}
\usepackage{tabulary}
\usepackage{tabularht}
\usepackage{acronym}
\usepackage{fancyhdr}
\usepackage{lscape}
\usepackage{amssymb}
\usepackage{pifont}
\usepackage[most]{tcolorbox}
\usepackage{slashbox}
\usepackage{multido}
\usepackage{caption}
\usepackage{graphicx,wrapfig,lipsum}
\usepackage{enumitem}
\usepackage {fancybox}
\usepackage{array,tabularx}
\usepackage{colortbl}
\usepackage[noend]{algorithmic}
\usepackage[linesnumbered,ruled,vlined,boxed,commentsnumbered]{algorithm2e}
\DeclareGraphicsExtensions{.jpg,.pdf,.PNG,.gif}
\usepackage[pdftex,colorlinks=true,linkcolor=black,citecolor=black,urlcolor=black]{hyperref}
\usepackage{pgfplots}
\usepackage{pgfplotstable}
\usepackage{subcaption}
\usepackage[tikz]{bclogo}
\pgfplotsset{grid style={dashed,gray}}
\pgfplotsset{minor grid style={dotted,green!50!black}}
\pgfplotsset{major grid style={dotted,green!50!black}}
\usepackage{anysize}
\marginsize{30mm}{20mm}{15mm}{15mm}
\newcolumntype{Y}{>{\raggedleft\arraybackslash}X}
\renewcommand{\baselinestretch}{1.5}
\newcolumntype{P}[1]{>{\raggedright}p{#1}}
\newcolumntype{M}[1]{>{\raggedright}m{#1}}%%declaration de page de garde
\newcommand*\rfrac[2]{{}^{#1}\!/_{#2}}
\setcounter{secnumdepth}{3}
\setcounter{tocdepth}{3}
\newcommand{\cmark}{\ding{51}}%
\newcommand{\xmark}{\ding{59}}%

% Document
\begin{document}
    \sloppy
    \begin{titlepage}
        \renewcommand{\baselinestretch}{1}
        \begin{center}
            \begin{RLtext}
                AljmhwryT AljzA'iryT AldymqrA.tyT Al^s`byT
            \end{RLtext}
            {PEOPLE'S DEMOCRATIC REPUBLIC OF ALGERIA}
            \begin{RLtext}
            {wzArT Alt`lym Al`Aly w Alb.h_t Al`lmy}
            \end{RLtext}
            {DEPARTMENT OF HIGHER EDUCATION AND SCIENTIFIC RESEARCH}
            \begin{RLtext}
                jAm`T mus.taf_A s.tmbOly bim`skar
            \end{RLtext}
            {UNIVERSITY OF MUSTAPHA STAMBOULI MASCARA}
            \begin{figure}[h]
            	\centering
            		\includegraphics[width=4cm]{figurs/logouniv.jpeg}
            \end{figure}
            \begin{RLtext}
                klyT Al`lwm AldaqyqaT
            \end{RLtext}
            {FACULTY OF EXACT SCIENCES}\\
            {COMPUTER SCIENCE DEPARTMENT}\\
            \textsc{\textbf{ \large Master memory} }
        \end{center}
        {\bfseries field :} Mathematics and Computer Science\\
        {\bfseries Faculty  {\hspace*{0.34cm}} :} Computer Science\\
        {\bfseries Option {\hspace*{0.26cm}} :} Networks and distributed system
        \vspace{0.5cm}
        \begin{center}
            \textsc{\textbf{ \large Realized by :}}\\
            {\bfseries Elkaim Moulay Abdellah }{\hspace*{5cm}}{\bfseries Mazouz  Nail}   \\
            \vspace{1cm}
            \Large {\bfseries Theme }
        \end{center}
        \begin{center}
            \hrule width 460pt
            \bigskip
            \Large  \centering \textbf{ \textsc{ ................. } }
            \bigskip
            \hrule width 460pt
        \end{center}
        \raggedright
        \bigskip
        \vspace{1cm}
        \textit{\bfseries Proposed by : Dr. Madber Hayat }
        \vspace{2cm}
        \begin{center}
            \textit{\bfseries College year 2019/2020}
        \end{center}
    \end{titlepage}

    %%Dedicaces%%%%%%%%%%%%%%%%%%%%%%%%%%%%%%%%%%%%%%%%%%%%%%%%%%%%%%%%%%%%%%%%%%%%%%%%%%%%%%%%%%%%%%%%%%%%%%%%%%%%%%%%%%%%%%%%

    \thispagestyle{empty}
    \begin{figure}[h]
    	\centering
    		\includegraphics[width=14cm]{figurs/B.PNG}
    \end{figure}
    \newpage
    \thispagestyle{empty}
    \clearpage
    \newpage
    \pagenumbering{roman}
    \begin{center}
        \textbf{\\}
        \textbf{\\}
        \textbf{\huge \textsc{\itshape Dedications}}\\
        \textbf{\\}
        \textbf{\\}
        \begin{flushleft}
            \textsf{\qquad I dedicate this work to ............. }\\
        \end{flushleft}
    \end{center}
    \begin{flushright}
        \textbf{\textsc{\itshape Elkaim Moulay Abdellah}}
    \end{flushright}
    \newpage
    \begin{center}
        \textbf{\\}
        \textbf{\\}
        \textbf{\huge \textsc{\itshape Dedications}}\\
        \textbf{\\}
        \textbf{\\}

        \begin{flushleft}
            \textsf{\qquad I dedicate this work to ...................... }
        \end{flushleft}

        \normalsize{\itshape .....}
        \textbf{\\}
        \textbf{\\}
    \end{center}
    \begin{flushright}
        \textbf{\textsc{\itshape Mazouz Nail}}
    \end{flushright}

    %%Remerciement%%%%%%%%%%%%%%%%%%%%%%%%%%%%%%%%%%%%%%%%%%%%%%%%%%%%%%%%%%%%%%%%%%%%%%%%%%%%%%%%%%%%%%%%%%%%%%%%%%%%%%%%%%%%%%%%
    \newpage
    \begin{center}
        %\thispagestyle{myheadings}
        %\markboth{droite}{ }
        \textbf{\huge \textsc{\itshape thanks}}
    \end{center}
    \textsf{\qquad We thank .......... }

    %%Abstract %%%%%%%%%%%%%%%%%%%%%%%%%%%%%%%%%%%%%%%%%%%%%%%%%%%%%%%%%%%%%%%%%%%%%%%%%%%%%%%%%%%%%%%%%%%%%%%%%%%%%%%%%%%%%%%%
    \newpage
    \markboth{droite}{ }
    \begin{center}
    \textbf{\huge \textsc{\itshape \textit Abstract}}\\
    \end{center}
    \qquad ...........................

    %%resume %%%%%%%%%%%%%%%%%%%%%%%%%%%%%%%%%%%%%%%%%%%%%%%%%%%%%%%%%%%%%%%%%%%%%%%%%%%%%%%%%%%%%%%%%%%%%%%%%%%%%%%%%%%%%%%%
    \newpage
    \newcommand{\enteteresume}{\markboth{Resume}{Resume}} %il faut ajouter les commandes
    \begin{center}
    \textbf{\huge \textsc{\itshape \textit Résumé}}\\
    \end{center}
    \qquad ...............................

    % La table des matieres%%%%%%%%%%%%%%%%%%%%%%%%%%%%%%%%%%%%%%%%%%%%%%%%%%%%%%%%%%%%%%%%%%%%%%%%%%%%%%%%%%%%%%%%%%%%%%%%%%%%%%%%%%%%%%%%%%%%%%%%%%%%%%%%%%%%%
    \tableofcontents
    \listoffigures
    \listoftables

    %% introduction generale%%%%%%%%%%%%%%%%%%%%%%%%%%%%%%%%%%%%%%%%%%%%%%%%%%%%%%%%%%%%%%%%%%%%%%%%%%%%%%%%%%%%%%%%%%%%%%%%%%%%%%%%%%%%%%%%%%%%%%%%%%%%%%%%%%%%%%%%%%%
    \chapter*{General Introduction}
    \addcontentsline{toc}{chapter}{Introduction générale}
    \setcounter{page}{1}
    \lhead{}
    \cfoot{\bfseries \thepage}
    \rhead{Introduction générale}
    \pagenumbering{arabic}

    ..................................................

    %% chapitre 1%%%%%%%%%%%%%%%%%%%%%%%%%%%%%%%%%%%%%%%%%%%%%%%%%%%%%%%%%%%%%%%%%%%%%%%%%%%%%%%%%%%%%%%%%%%%%%%%%%%%%%%%%%%%%%%%%%%%%%%%%%%%%%%%%%%%%%%%%%%%%%%%%%%%%
    \include{chapiter1/main}

    %% chapitre 2%%%%%%%%%%%%%%%%%%%%%%%%%%%%%%%%%%%%%%%%%%%%%%%%%%%%%%%%%%%%%%%%%%%%%%%%%%%%%%%%%%%%%%%%%%%%%%%%%%%%%%%%%%%%%%%%%%%%%%%%%%%%%%%%%%%%%%%%%%%%%%%%%%%%%
    \include{chapiter2/main}

    %% conclusion generale%%%%%%%%%%%%%%%%%%%%%%%%%%%%%%%%%%%%%%%%%%%%%%%%%%%%%%%%%%%%%%%%%%%%%%%%%%%%%%%%%%%%%%%%%%%%%%%%%%%%%%%%%%%%%%%%%%%%%%%%%%%%%%%%%%%%%%%%%%%
    \chapter*{General conclusion}
    \addcontentsline{toc}{chapter}{General conclusion}
    \lhead{}
    \cfoot{\bfseries \thepage}
    \rhead{General conclusion}
    \markboth{droite}{General conclusion}

    .............................................

    %%References%%%%%%%%%%%%%%%%%%%%%%%%%%%%%%%%%%%%%%%%%%%%%%%%%%%%%%%%%%%%%%%%%%%%%%%%%%%%%%%%%%%%%%%%%%%%%%%%%%%%%%%%%%%%%%%%%%%%%%%%%%%%%%
    \clearpage
    \pagestyle{fancy}
    \addcontentsline{toc}{chapter}{Bibliography}
    \begin{thebibliography}{99}
        \lhead{}
        \cfoot{\bfseries \thepage}
        \rhead{Bibliography}

        \bibitem[1]{1}
        H. Altaama,
        Application Mobile Guide,
        Mémoire de Master en Informatique,
        Université Abou Bakr Belkaid de Tlemcen,
        2016.
        \bibitem[2]{2}
        \url{http://generationmobiles.net/2014/11/les-differents-types-dapps-mobiles/}, [consulté le 14/03/2018].

    \end{thebibliography}

\end{document}

    %% conclusion generale%%%%%%%%%%%%%%%%%%%%%%%%%%%%%%%%%%%%%%%%%%%%%%%%%%%%%%%%%%%%%%%%%%%%%%%%%%%%%%%%%%%%%%%%%%%%%%%%%%%%%%%%%%%%%%%%%%%%%%%%%%%%%%%%%%%%%%%%%%%
    \chapter*{General conclusion}
    \addcontentsline{toc}{chapter}{General conclusion}
    \lhead{}
    \cfoot{\bfseries \thepage}
    \rhead{General conclusion}
    \markboth{droite}{General conclusion}

    .............................................

    %%References%%%%%%%%%%%%%%%%%%%%%%%%%%%%%%%%%%%%%%%%%%%%%%%%%%%%%%%%%%%%%%%%%%%%%%%%%%%%%%%%%%%%%%%%%%%%%%%%%%%%%%%%%%%%%%%%%%%%%%%%%%%%%%
    \clearpage
    \pagestyle{fancy}
    \addcontentsline{toc}{chapter}{Bibliography}
    \begin{thebibliography}{99}
        \lhead{}
        \cfoot{\bfseries \thepage}
        \rhead{Bibliography}

        \bibitem[1]{1}
        H. Altaama,
        Application Mobile Guide,
        Mémoire de Master en Informatique,
        Université Abou Bakr Belkaid de Tlemcen,
        2016.
        \bibitem[2]{2}
        \url{http://generationmobiles.net/2014/11/les-differents-types-dapps-mobiles/}, [consulté le 14/03/2018].

    \end{thebibliography}

\end{document}

    %% conclusion generale%%%%%%%%%%%%%%%%%%%%%%%%%%%%%%%%%%%%%%%%%%%%%%%%%%%%%%%%%%%%%%%%%%%%%%%%%%%%%%%%%%%%%%%%%%%%%%%%%%%%%%%%%%%%%%%%%%%%%%%%%%%%%%%%%%%%%%%%%%%
    \chapter*{General conclusion}
    \addcontentsline{toc}{chapter}{General conclusion}
    \lhead{}
    \cfoot{\bfseries \thepage}
    \rhead{General conclusion}
    \markboth{droite}{General conclusion}

    .............................................

    %%References%%%%%%%%%%%%%%%%%%%%%%%%%%%%%%%%%%%%%%%%%%%%%%%%%%%%%%%%%%%%%%%%%%%%%%%%%%%%%%%%%%%%%%%%%%%%%%%%%%%%%%%%%%%%%%%%%%%%%%%%%%%%%%
    \clearpage
    \pagestyle{fancy}
    \addcontentsline{toc}{chapter}{Bibliography}
    \begin{thebibliography}{99}
        \lhead{}
        \cfoot{\bfseries \thepage}
        \rhead{Bibliography}

        \bibitem[1]{1}
        H. Altaama,
        Application Mobile Guide,
        Mémoire de Master en Informatique,
        Université Abou Bakr Belkaid de Tlemcen,
        2016.
        \bibitem[2]{2}
        \url{http://generationmobiles.net/2014/11/les-differents-types-dapps-mobiles/}, [consulté le 14/03/2018].

    \end{thebibliography}

\end{document}

     %% chapitre 3%%%%%%%%%%%%%%%%%%%%%%%%%%%%%%%%%%%%%%%%%%%%%%%%%%%%%%%%%%%%%%%%%%%%%%%%%%%%%%%%%%%%%%%%%%%%%%%%%%%%%%%%%%%%%%%%%%%%%%%%%%%%%%%%%%%%%%%%%%%%%%%%%%%%%
    %! Author = moulay
%! Date = 10/21/19

% Preamble
\documentclass[english,a4,12pt]{report}

% Packages
\usepackage{amsmath}
\usepackage[utf8]{inputenc}
\usepackage{xcolor}
\usepackage[T1]{fontenc}
\usepackage{arabtex}
\usepackage{sectsty}
\usepackage[cyr]{aeguill}
\usepackage{rotating}
\usepackage{multirow}
\usepackage{tabulary}
\usepackage{tabularht}
\usepackage{acronym}
\usepackage{fancyhdr}
\usepackage{lscape}
\usepackage{amssymb}
\usepackage{pifont}
\usepackage[most]{tcolorbox}
\usepackage{slashbox}
\usepackage{multido}
\usepackage{caption}
\usepackage{graphicx,wrapfig,lipsum}
\usepackage{enumitem}
\usepackage {fancybox}
\usepackage{array,tabularx}
\usepackage{colortbl}
\usepackage[noend]{algorithmic}
\usepackage[linesnumbered,ruled,vlined,boxed,commentsnumbered]{algorithm2e}
\DeclareGraphicsExtensions{.jpg,.pdf,.PNG,.gif}
\usepackage[pdftex,colorlinks=true,linkcolor=black,citecolor=black,urlcolor=black]{hyperref}
\usepackage{pgfplots}
\usepackage{pgfplotstable}
\usepackage{subcaption}
\usepackage[tikz]{bclogo}
\pgfplotsset{grid style={dashed,gray}}
\pgfplotsset{minor grid style={dotted,green!50!black}}
\pgfplotsset{major grid style={dotted,green!50!black}}
\usepackage{anysize}
\marginsize{30mm}{20mm}{15mm}{15mm}
\newcolumntype{Y}{>{\raggedleft\arraybackslash}X}
\renewcommand{\baselinestretch}{1.5}
\newcolumntype{P}[1]{>{\raggedright}p{#1}}
\newcolumntype{M}[1]{>{\raggedright}m{#1}}%%declaration de page de garde
\newcommand*\rfrac[2]{{}^{#1}\!/_{#2}}
\setcounter{secnumdepth}{3}
\setcounter{tocdepth}{3}
\newcommand{\cmark}{\ding{51}}%
\newcommand{\xmark}{\ding{59}}%

% Document
\begin{document}
    \sloppy
    \begin{titlepage}
        \renewcommand{\baselinestretch}{1}
        \begin{center}
            \begin{RLtext}
                AljmhwryT AljzA'iryT AldymqrA.tyT Al^s`byT
            \end{RLtext}
            {PEOPLE'S DEMOCRATIC REPUBLIC OF ALGERIA}
            \begin{RLtext}
            {wzArT Alt`lym Al`Aly w Alb.h_t Al`lmy}
            \end{RLtext}
            {DEPARTMENT OF HIGHER EDUCATION AND SCIENTIFIC RESEARCH}
            \begin{RLtext}
                jAm`T mus.taf_A s.tmbOly bim`skar
            \end{RLtext}
            {UNIVERSITY OF MUSTAPHA STAMBOULI MASCARA}
            \begin{figure}[h]
            	\centering
            		\includegraphics[width=4cm]{figurs/logouniv.jpeg}
            \end{figure}
            \begin{RLtext}
                klyT Al`lwm AldaqyqaT
            \end{RLtext}
            {FACULTY OF EXACT SCIENCES}\\
            {COMPUTER SCIENCE DEPARTMENT}\\
            \textsc{\textbf{ \large Master memory} }
        \end{center}
        {\bfseries field :} Mathematics and Computer Science\\
        {\bfseries Faculty  {\hspace*{0.34cm}} :} Computer Science\\
        {\bfseries Option {\hspace*{0.26cm}} :} Networks and distributed system
        \vspace{0.5cm}
        \begin{center}
            \textsc{\textbf{ \large Realized by :}}\\
            {\bfseries Elkaim Moulay Abdellah }{\hspace*{5cm}}{\bfseries Mazouz  Nail}   \\
            \vspace{1cm}
            \Large {\bfseries Theme }
        \end{center}
        \begin{center}
            \hrule width 460pt
            \bigskip
            \Large  \centering \textbf{ \textsc{ ................. } }
            \bigskip
            \hrule width 460pt
        \end{center}
        \raggedright
        \bigskip
        \vspace{1cm}
        \textit{\bfseries Proposed by : Dr. Madber Hayat }
        \vspace{2cm}
        \begin{center}
            \textit{\bfseries College year 2019/2020}
        \end{center}
    \end{titlepage}

    %%Dedicaces%%%%%%%%%%%%%%%%%%%%%%%%%%%%%%%%%%%%%%%%%%%%%%%%%%%%%%%%%%%%%%%%%%%%%%%%%%%%%%%%%%%%%%%%%%%%%%%%%%%%%%%%%%%%%%%%

    \thispagestyle{empty}
    \begin{figure}[h]
    	\centering
    		\includegraphics[width=14cm]{figurs/B.PNG}
    \end{figure}
    \newpage
    \thispagestyle{empty}
    \clearpage
    \newpage
    \pagenumbering{roman}
    \begin{center}
        \textbf{\\}
        \textbf{\\}
        \textbf{\huge \textsc{\itshape Dedications}}\\
        \textbf{\\}
        \textbf{\\}
        \begin{flushleft}
            \textsf{\qquad I dedicate this work to ............. }\\
        \end{flushleft}
    \end{center}
    \begin{flushright}
        \textbf{\textsc{\itshape Elkaim Moulay Abdellah}}
    \end{flushright}
    \newpage
    \begin{center}
        \textbf{\\}
        \textbf{\\}
        \textbf{\huge \textsc{\itshape Dedications}}\\
        \textbf{\\}
        \textbf{\\}

        \begin{flushleft}
            \textsf{\qquad I dedicate this work to ...................... }
        \end{flushleft}

        \normalsize{\itshape .....}
        \textbf{\\}
        \textbf{\\}
    \end{center}
    \begin{flushright}
        \textbf{\textsc{\itshape Mazouz Nail}}
    \end{flushright}

    %%Remerciement%%%%%%%%%%%%%%%%%%%%%%%%%%%%%%%%%%%%%%%%%%%%%%%%%%%%%%%%%%%%%%%%%%%%%%%%%%%%%%%%%%%%%%%%%%%%%%%%%%%%%%%%%%%%%%%%
    \newpage
    \begin{center}
        %\thispagestyle{myheadings}
        %\markboth{droite}{ }
        \textbf{\huge \textsc{\itshape thanks}}
    \end{center}
    \textsf{\qquad We thank .......... }

    %%Abstract %%%%%%%%%%%%%%%%%%%%%%%%%%%%%%%%%%%%%%%%%%%%%%%%%%%%%%%%%%%%%%%%%%%%%%%%%%%%%%%%%%%%%%%%%%%%%%%%%%%%%%%%%%%%%%%%
    \newpage
    \markboth{droite}{ }
    \begin{center}
    \textbf{\huge \textsc{\itshape \textit Abstract}}\\
    \end{center}
    \qquad ...........................

    %%resume %%%%%%%%%%%%%%%%%%%%%%%%%%%%%%%%%%%%%%%%%%%%%%%%%%%%%%%%%%%%%%%%%%%%%%%%%%%%%%%%%%%%%%%%%%%%%%%%%%%%%%%%%%%%%%%%
    \newpage
    \newcommand{\enteteresume}{\markboth{Resume}{Resume}} %il faut ajouter les commandes
    \begin{center}
    \textbf{\huge \textsc{\itshape \textit Résumé}}\\
    \end{center}
    \qquad ...............................

    % La table des matieres%%%%%%%%%%%%%%%%%%%%%%%%%%%%%%%%%%%%%%%%%%%%%%%%%%%%%%%%%%%%%%%%%%%%%%%%%%%%%%%%%%%%%%%%%%%%%%%%%%%%%%%%%%%%%%%%%%%%%%%%%%%%%%%%%%%%%
    \tableofcontents
    \listoffigures
    \listoftables

    %% introduction generale%%%%%%%%%%%%%%%%%%%%%%%%%%%%%%%%%%%%%%%%%%%%%%%%%%%%%%%%%%%%%%%%%%%%%%%%%%%%%%%%%%%%%%%%%%%%%%%%%%%%%%%%%%%%%%%%%%%%%%%%%%%%%%%%%%%%%%%%%%%
    \chapter*{General Introduction}
    \addcontentsline{toc}{chapter}{Introduction générale}
    \setcounter{page}{1}
    \lhead{}
    \cfoot{\bfseries \thepage}
    \rhead{Introduction générale}
    \pagenumbering{arabic}

    ..................................................

    %% chapitre 1%%%%%%%%%%%%%%%%%%%%%%%%%%%%%%%%%%%%%%%%%%%%%%%%%%%%%%%%%%%%%%%%%%%%%%%%%%%%%%%%%%%%%%%%%%%%%%%%%%%%%%%%%%%%%%%%%%%%%%%%%%%%%%%%%%%%%%%%%%%%%%%%%%%%%
    %! Author = moulay
%! Date = 10/21/19

% Preamble
\documentclass[english,a4,12pt]{report}

% Packages
\usepackage{amsmath}
\usepackage[utf8]{inputenc}
\usepackage{xcolor}
\usepackage[T1]{fontenc}
\usepackage{arabtex}
\usepackage{sectsty}
\usepackage[cyr]{aeguill}
\usepackage{rotating}
\usepackage{multirow}
\usepackage{tabulary}
\usepackage{tabularht}
\usepackage{acronym}
\usepackage{fancyhdr}
\usepackage{lscape}
\usepackage{amssymb}
\usepackage{pifont}
\usepackage[most]{tcolorbox}
\usepackage{slashbox}
\usepackage{multido}
\usepackage{caption}
\usepackage{graphicx,wrapfig,lipsum}
\usepackage{enumitem}
\usepackage {fancybox}
\usepackage{array,tabularx}
\usepackage{colortbl}
\usepackage[noend]{algorithmic}
\usepackage[linesnumbered,ruled,vlined,boxed,commentsnumbered]{algorithm2e}
\DeclareGraphicsExtensions{.jpg,.pdf,.PNG,.gif}
\usepackage[pdftex,colorlinks=true,linkcolor=black,citecolor=black,urlcolor=black]{hyperref}
\usepackage{pgfplots}
\usepackage{pgfplotstable}
\usepackage{subcaption}
\usepackage[tikz]{bclogo}
\pgfplotsset{grid style={dashed,gray}}
\pgfplotsset{minor grid style={dotted,green!50!black}}
\pgfplotsset{major grid style={dotted,green!50!black}}
\usepackage{anysize}
\marginsize{30mm}{20mm}{15mm}{15mm}
\newcolumntype{Y}{>{\raggedleft\arraybackslash}X}
\renewcommand{\baselinestretch}{1.5}
\newcolumntype{P}[1]{>{\raggedright}p{#1}}
\newcolumntype{M}[1]{>{\raggedright}m{#1}}%%declaration de page de garde
\newcommand*\rfrac[2]{{}^{#1}\!/_{#2}}
\setcounter{secnumdepth}{3}
\setcounter{tocdepth}{3}
\newcommand{\cmark}{\ding{51}}%
\newcommand{\xmark}{\ding{59}}%

% Document
\begin{document}
    \sloppy
    \begin{titlepage}
        \renewcommand{\baselinestretch}{1}
        \begin{center}
            \begin{RLtext}
                AljmhwryT AljzA'iryT AldymqrA.tyT Al^s`byT
            \end{RLtext}
            {PEOPLE'S DEMOCRATIC REPUBLIC OF ALGERIA}
            \begin{RLtext}
            {wzArT Alt`lym Al`Aly w Alb.h_t Al`lmy}
            \end{RLtext}
            {DEPARTMENT OF HIGHER EDUCATION AND SCIENTIFIC RESEARCH}
            \begin{RLtext}
                jAm`T mus.taf_A s.tmbOly bim`skar
            \end{RLtext}
            {UNIVERSITY OF MUSTAPHA STAMBOULI MASCARA}
            \begin{figure}[h]
            	\centering
            		\includegraphics[width=4cm]{figurs/logouniv.jpeg}
            \end{figure}
            \begin{RLtext}
                klyT Al`lwm AldaqyqaT
            \end{RLtext}
            {FACULTY OF EXACT SCIENCES}\\
            {COMPUTER SCIENCE DEPARTMENT}\\
            \textsc{\textbf{ \large Master memory} }
        \end{center}
        {\bfseries field :} Mathematics and Computer Science\\
        {\bfseries Faculty  {\hspace*{0.34cm}} :} Computer Science\\
        {\bfseries Option {\hspace*{0.26cm}} :} Networks and distributed system
        \vspace{0.5cm}
        \begin{center}
            \textsc{\textbf{ \large Realized by :}}\\
            {\bfseries Elkaim Moulay Abdellah }{\hspace*{5cm}}{\bfseries Mazouz  Nail}   \\
            \vspace{1cm}
            \Large {\bfseries Theme }
        \end{center}
        \begin{center}
            \hrule width 460pt
            \bigskip
            \Large  \centering \textbf{ \textsc{ ................. } }
            \bigskip
            \hrule width 460pt
        \end{center}
        \raggedright
        \bigskip
        \vspace{1cm}
        \textit{\bfseries Proposed by : Dr. Madber Hayat }
        \vspace{2cm}
        \begin{center}
            \textit{\bfseries College year 2019/2020}
        \end{center}
    \end{titlepage}

    %%Dedicaces%%%%%%%%%%%%%%%%%%%%%%%%%%%%%%%%%%%%%%%%%%%%%%%%%%%%%%%%%%%%%%%%%%%%%%%%%%%%%%%%%%%%%%%%%%%%%%%%%%%%%%%%%%%%%%%%

    \thispagestyle{empty}
    \begin{figure}[h]
    	\centering
    		\includegraphics[width=14cm]{figurs/B.PNG}
    \end{figure}
    \newpage
    \thispagestyle{empty}
    \clearpage
    \newpage
    \pagenumbering{roman}
    \begin{center}
        \textbf{\\}
        \textbf{\\}
        \textbf{\huge \textsc{\itshape Dedications}}\\
        \textbf{\\}
        \textbf{\\}
        \begin{flushleft}
            \textsf{\qquad I dedicate this work to ............. }\\
        \end{flushleft}
    \end{center}
    \begin{flushright}
        \textbf{\textsc{\itshape Elkaim Moulay Abdellah}}
    \end{flushright}
    \newpage
    \begin{center}
        \textbf{\\}
        \textbf{\\}
        \textbf{\huge \textsc{\itshape Dedications}}\\
        \textbf{\\}
        \textbf{\\}

        \begin{flushleft}
            \textsf{\qquad I dedicate this work to ...................... }
        \end{flushleft}

        \normalsize{\itshape .....}
        \textbf{\\}
        \textbf{\\}
    \end{center}
    \begin{flushright}
        \textbf{\textsc{\itshape Mazouz Nail}}
    \end{flushright}

    %%Remerciement%%%%%%%%%%%%%%%%%%%%%%%%%%%%%%%%%%%%%%%%%%%%%%%%%%%%%%%%%%%%%%%%%%%%%%%%%%%%%%%%%%%%%%%%%%%%%%%%%%%%%%%%%%%%%%%%
    \newpage
    \begin{center}
        %\thispagestyle{myheadings}
        %\markboth{droite}{ }
        \textbf{\huge \textsc{\itshape thanks}}
    \end{center}
    \textsf{\qquad We thank .......... }

    %%Abstract %%%%%%%%%%%%%%%%%%%%%%%%%%%%%%%%%%%%%%%%%%%%%%%%%%%%%%%%%%%%%%%%%%%%%%%%%%%%%%%%%%%%%%%%%%%%%%%%%%%%%%%%%%%%%%%%
    \newpage
    \markboth{droite}{ }
    \begin{center}
    \textbf{\huge \textsc{\itshape \textit Abstract}}\\
    \end{center}
    \qquad ...........................

    %%resume %%%%%%%%%%%%%%%%%%%%%%%%%%%%%%%%%%%%%%%%%%%%%%%%%%%%%%%%%%%%%%%%%%%%%%%%%%%%%%%%%%%%%%%%%%%%%%%%%%%%%%%%%%%%%%%%
    \newpage
    \newcommand{\enteteresume}{\markboth{Resume}{Resume}} %il faut ajouter les commandes
    \begin{center}
    \textbf{\huge \textsc{\itshape \textit Résumé}}\\
    \end{center}
    \qquad ...............................

    % La table des matieres%%%%%%%%%%%%%%%%%%%%%%%%%%%%%%%%%%%%%%%%%%%%%%%%%%%%%%%%%%%%%%%%%%%%%%%%%%%%%%%%%%%%%%%%%%%%%%%%%%%%%%%%%%%%%%%%%%%%%%%%%%%%%%%%%%%%%
    \tableofcontents
    \listoffigures
    \listoftables

    %% introduction generale%%%%%%%%%%%%%%%%%%%%%%%%%%%%%%%%%%%%%%%%%%%%%%%%%%%%%%%%%%%%%%%%%%%%%%%%%%%%%%%%%%%%%%%%%%%%%%%%%%%%%%%%%%%%%%%%%%%%%%%%%%%%%%%%%%%%%%%%%%%
    \chapter*{General Introduction}
    \addcontentsline{toc}{chapter}{Introduction générale}
    \setcounter{page}{1}
    \lhead{}
    \cfoot{\bfseries \thepage}
    \rhead{Introduction générale}
    \pagenumbering{arabic}

    ..................................................

    %% chapitre 1%%%%%%%%%%%%%%%%%%%%%%%%%%%%%%%%%%%%%%%%%%%%%%%%%%%%%%%%%%%%%%%%%%%%%%%%%%%%%%%%%%%%%%%%%%%%%%%%%%%%%%%%%%%%%%%%%%%%%%%%%%%%%%%%%%%%%%%%%%%%%%%%%%%%%
    %! Author = moulay
%! Date = 10/21/19

% Preamble
\documentclass[english,a4,12pt]{report}

% Packages
\usepackage{amsmath}
\usepackage[utf8]{inputenc}
\usepackage{xcolor}
\usepackage[T1]{fontenc}
\usepackage{arabtex}
\usepackage{sectsty}
\usepackage[cyr]{aeguill}
\usepackage{rotating}
\usepackage{multirow}
\usepackage{tabulary}
\usepackage{tabularht}
\usepackage{acronym}
\usepackage{fancyhdr}
\usepackage{lscape}
\usepackage{amssymb}
\usepackage{pifont}
\usepackage[most]{tcolorbox}
\usepackage{slashbox}
\usepackage{multido}
\usepackage{caption}
\usepackage{graphicx,wrapfig,lipsum}
\usepackage{enumitem}
\usepackage {fancybox}
\usepackage{array,tabularx}
\usepackage{colortbl}
\usepackage[noend]{algorithmic}
\usepackage[linesnumbered,ruled,vlined,boxed,commentsnumbered]{algorithm2e}
\DeclareGraphicsExtensions{.jpg,.pdf,.PNG,.gif}
\usepackage[pdftex,colorlinks=true,linkcolor=black,citecolor=black,urlcolor=black]{hyperref}
\usepackage{pgfplots}
\usepackage{pgfplotstable}
\usepackage{subcaption}
\usepackage[tikz]{bclogo}
\pgfplotsset{grid style={dashed,gray}}
\pgfplotsset{minor grid style={dotted,green!50!black}}
\pgfplotsset{major grid style={dotted,green!50!black}}
\usepackage{anysize}
\marginsize{30mm}{20mm}{15mm}{15mm}
\newcolumntype{Y}{>{\raggedleft\arraybackslash}X}
\renewcommand{\baselinestretch}{1.5}
\newcolumntype{P}[1]{>{\raggedright}p{#1}}
\newcolumntype{M}[1]{>{\raggedright}m{#1}}%%declaration de page de garde
\newcommand*\rfrac[2]{{}^{#1}\!/_{#2}}
\setcounter{secnumdepth}{3}
\setcounter{tocdepth}{3}
\newcommand{\cmark}{\ding{51}}%
\newcommand{\xmark}{\ding{59}}%

% Document
\begin{document}
    \sloppy
    \begin{titlepage}
        \renewcommand{\baselinestretch}{1}
        \begin{center}
            \begin{RLtext}
                AljmhwryT AljzA'iryT AldymqrA.tyT Al^s`byT
            \end{RLtext}
            {PEOPLE'S DEMOCRATIC REPUBLIC OF ALGERIA}
            \begin{RLtext}
            {wzArT Alt`lym Al`Aly w Alb.h_t Al`lmy}
            \end{RLtext}
            {DEPARTMENT OF HIGHER EDUCATION AND SCIENTIFIC RESEARCH}
            \begin{RLtext}
                jAm`T mus.taf_A s.tmbOly bim`skar
            \end{RLtext}
            {UNIVERSITY OF MUSTAPHA STAMBOULI MASCARA}
            \begin{figure}[h]
            	\centering
            		\includegraphics[width=4cm]{figurs/logouniv.jpeg}
            \end{figure}
            \begin{RLtext}
                klyT Al`lwm AldaqyqaT
            \end{RLtext}
            {FACULTY OF EXACT SCIENCES}\\
            {COMPUTER SCIENCE DEPARTMENT}\\
            \textsc{\textbf{ \large Master memory} }
        \end{center}
        {\bfseries field :} Mathematics and Computer Science\\
        {\bfseries Faculty  {\hspace*{0.34cm}} :} Computer Science\\
        {\bfseries Option {\hspace*{0.26cm}} :} Networks and distributed system
        \vspace{0.5cm}
        \begin{center}
            \textsc{\textbf{ \large Realized by :}}\\
            {\bfseries Elkaim Moulay Abdellah }{\hspace*{5cm}}{\bfseries Mazouz  Nail}   \\
            \vspace{1cm}
            \Large {\bfseries Theme }
        \end{center}
        \begin{center}
            \hrule width 460pt
            \bigskip
            \Large  \centering \textbf{ \textsc{ ................. } }
            \bigskip
            \hrule width 460pt
        \end{center}
        \raggedright
        \bigskip
        \vspace{1cm}
        \textit{\bfseries Proposed by : Dr. Madber Hayat }
        \vspace{2cm}
        \begin{center}
            \textit{\bfseries College year 2019/2020}
        \end{center}
    \end{titlepage}

    %%Dedicaces%%%%%%%%%%%%%%%%%%%%%%%%%%%%%%%%%%%%%%%%%%%%%%%%%%%%%%%%%%%%%%%%%%%%%%%%%%%%%%%%%%%%%%%%%%%%%%%%%%%%%%%%%%%%%%%%

    \thispagestyle{empty}
    \begin{figure}[h]
    	\centering
    		\includegraphics[width=14cm]{figurs/B.PNG}
    \end{figure}
    \newpage
    \thispagestyle{empty}
    \clearpage
    \newpage
    \pagenumbering{roman}
    \begin{center}
        \textbf{\\}
        \textbf{\\}
        \textbf{\huge \textsc{\itshape Dedications}}\\
        \textbf{\\}
        \textbf{\\}
        \begin{flushleft}
            \textsf{\qquad I dedicate this work to ............. }\\
        \end{flushleft}
    \end{center}
    \begin{flushright}
        \textbf{\textsc{\itshape Elkaim Moulay Abdellah}}
    \end{flushright}
    \newpage
    \begin{center}
        \textbf{\\}
        \textbf{\\}
        \textbf{\huge \textsc{\itshape Dedications}}\\
        \textbf{\\}
        \textbf{\\}

        \begin{flushleft}
            \textsf{\qquad I dedicate this work to ...................... }
        \end{flushleft}

        \normalsize{\itshape .....}
        \textbf{\\}
        \textbf{\\}
    \end{center}
    \begin{flushright}
        \textbf{\textsc{\itshape Mazouz Nail}}
    \end{flushright}

    %%Remerciement%%%%%%%%%%%%%%%%%%%%%%%%%%%%%%%%%%%%%%%%%%%%%%%%%%%%%%%%%%%%%%%%%%%%%%%%%%%%%%%%%%%%%%%%%%%%%%%%%%%%%%%%%%%%%%%%
    \newpage
    \begin{center}
        %\thispagestyle{myheadings}
        %\markboth{droite}{ }
        \textbf{\huge \textsc{\itshape thanks}}
    \end{center}
    \textsf{\qquad We thank .......... }

    %%Abstract %%%%%%%%%%%%%%%%%%%%%%%%%%%%%%%%%%%%%%%%%%%%%%%%%%%%%%%%%%%%%%%%%%%%%%%%%%%%%%%%%%%%%%%%%%%%%%%%%%%%%%%%%%%%%%%%
    \newpage
    \markboth{droite}{ }
    \begin{center}
    \textbf{\huge \textsc{\itshape \textit Abstract}}\\
    \end{center}
    \qquad ...........................

    %%resume %%%%%%%%%%%%%%%%%%%%%%%%%%%%%%%%%%%%%%%%%%%%%%%%%%%%%%%%%%%%%%%%%%%%%%%%%%%%%%%%%%%%%%%%%%%%%%%%%%%%%%%%%%%%%%%%
    \newpage
    \newcommand{\enteteresume}{\markboth{Resume}{Resume}} %il faut ajouter les commandes
    \begin{center}
    \textbf{\huge \textsc{\itshape \textit Résumé}}\\
    \end{center}
    \qquad ...............................

    % La table des matieres%%%%%%%%%%%%%%%%%%%%%%%%%%%%%%%%%%%%%%%%%%%%%%%%%%%%%%%%%%%%%%%%%%%%%%%%%%%%%%%%%%%%%%%%%%%%%%%%%%%%%%%%%%%%%%%%%%%%%%%%%%%%%%%%%%%%%
    \tableofcontents
    \listoffigures
    \listoftables

    %% introduction generale%%%%%%%%%%%%%%%%%%%%%%%%%%%%%%%%%%%%%%%%%%%%%%%%%%%%%%%%%%%%%%%%%%%%%%%%%%%%%%%%%%%%%%%%%%%%%%%%%%%%%%%%%%%%%%%%%%%%%%%%%%%%%%%%%%%%%%%%%%%
    \chapter*{General Introduction}
    \addcontentsline{toc}{chapter}{Introduction générale}
    \setcounter{page}{1}
    \lhead{}
    \cfoot{\bfseries \thepage}
    \rhead{Introduction générale}
    \pagenumbering{arabic}

    ..................................................

    %% chapitre 1%%%%%%%%%%%%%%%%%%%%%%%%%%%%%%%%%%%%%%%%%%%%%%%%%%%%%%%%%%%%%%%%%%%%%%%%%%%%%%%%%%%%%%%%%%%%%%%%%%%%%%%%%%%%%%%%%%%%%%%%%%%%%%%%%%%%%%%%%%%%%%%%%%%%%
    \include{chapiter1/main}

    %% chapitre 2%%%%%%%%%%%%%%%%%%%%%%%%%%%%%%%%%%%%%%%%%%%%%%%%%%%%%%%%%%%%%%%%%%%%%%%%%%%%%%%%%%%%%%%%%%%%%%%%%%%%%%%%%%%%%%%%%%%%%%%%%%%%%%%%%%%%%%%%%%%%%%%%%%%%%
    \include{chapiter2/main}

    %% conclusion generale%%%%%%%%%%%%%%%%%%%%%%%%%%%%%%%%%%%%%%%%%%%%%%%%%%%%%%%%%%%%%%%%%%%%%%%%%%%%%%%%%%%%%%%%%%%%%%%%%%%%%%%%%%%%%%%%%%%%%%%%%%%%%%%%%%%%%%%%%%%
    \chapter*{General conclusion}
    \addcontentsline{toc}{chapter}{General conclusion}
    \lhead{}
    \cfoot{\bfseries \thepage}
    \rhead{General conclusion}
    \markboth{droite}{General conclusion}

    .............................................

    %%References%%%%%%%%%%%%%%%%%%%%%%%%%%%%%%%%%%%%%%%%%%%%%%%%%%%%%%%%%%%%%%%%%%%%%%%%%%%%%%%%%%%%%%%%%%%%%%%%%%%%%%%%%%%%%%%%%%%%%%%%%%%%%%
    \clearpage
    \pagestyle{fancy}
    \addcontentsline{toc}{chapter}{Bibliography}
    \begin{thebibliography}{99}
        \lhead{}
        \cfoot{\bfseries \thepage}
        \rhead{Bibliography}

        \bibitem[1]{1}
        H. Altaama,
        Application Mobile Guide,
        Mémoire de Master en Informatique,
        Université Abou Bakr Belkaid de Tlemcen,
        2016.
        \bibitem[2]{2}
        \url{http://generationmobiles.net/2014/11/les-differents-types-dapps-mobiles/}, [consulté le 14/03/2018].

    \end{thebibliography}

\end{document}

    %% chapitre 2%%%%%%%%%%%%%%%%%%%%%%%%%%%%%%%%%%%%%%%%%%%%%%%%%%%%%%%%%%%%%%%%%%%%%%%%%%%%%%%%%%%%%%%%%%%%%%%%%%%%%%%%%%%%%%%%%%%%%%%%%%%%%%%%%%%%%%%%%%%%%%%%%%%%%
    %! Author = moulay
%! Date = 10/21/19

% Preamble
\documentclass[english,a4,12pt]{report}

% Packages
\usepackage{amsmath}
\usepackage[utf8]{inputenc}
\usepackage{xcolor}
\usepackage[T1]{fontenc}
\usepackage{arabtex}
\usepackage{sectsty}
\usepackage[cyr]{aeguill}
\usepackage{rotating}
\usepackage{multirow}
\usepackage{tabulary}
\usepackage{tabularht}
\usepackage{acronym}
\usepackage{fancyhdr}
\usepackage{lscape}
\usepackage{amssymb}
\usepackage{pifont}
\usepackage[most]{tcolorbox}
\usepackage{slashbox}
\usepackage{multido}
\usepackage{caption}
\usepackage{graphicx,wrapfig,lipsum}
\usepackage{enumitem}
\usepackage {fancybox}
\usepackage{array,tabularx}
\usepackage{colortbl}
\usepackage[noend]{algorithmic}
\usepackage[linesnumbered,ruled,vlined,boxed,commentsnumbered]{algorithm2e}
\DeclareGraphicsExtensions{.jpg,.pdf,.PNG,.gif}
\usepackage[pdftex,colorlinks=true,linkcolor=black,citecolor=black,urlcolor=black]{hyperref}
\usepackage{pgfplots}
\usepackage{pgfplotstable}
\usepackage{subcaption}
\usepackage[tikz]{bclogo}
\pgfplotsset{grid style={dashed,gray}}
\pgfplotsset{minor grid style={dotted,green!50!black}}
\pgfplotsset{major grid style={dotted,green!50!black}}
\usepackage{anysize}
\marginsize{30mm}{20mm}{15mm}{15mm}
\newcolumntype{Y}{>{\raggedleft\arraybackslash}X}
\renewcommand{\baselinestretch}{1.5}
\newcolumntype{P}[1]{>{\raggedright}p{#1}}
\newcolumntype{M}[1]{>{\raggedright}m{#1}}%%declaration de page de garde
\newcommand*\rfrac[2]{{}^{#1}\!/_{#2}}
\setcounter{secnumdepth}{3}
\setcounter{tocdepth}{3}
\newcommand{\cmark}{\ding{51}}%
\newcommand{\xmark}{\ding{59}}%

% Document
\begin{document}
    \sloppy
    \begin{titlepage}
        \renewcommand{\baselinestretch}{1}
        \begin{center}
            \begin{RLtext}
                AljmhwryT AljzA'iryT AldymqrA.tyT Al^s`byT
            \end{RLtext}
            {PEOPLE'S DEMOCRATIC REPUBLIC OF ALGERIA}
            \begin{RLtext}
            {wzArT Alt`lym Al`Aly w Alb.h_t Al`lmy}
            \end{RLtext}
            {DEPARTMENT OF HIGHER EDUCATION AND SCIENTIFIC RESEARCH}
            \begin{RLtext}
                jAm`T mus.taf_A s.tmbOly bim`skar
            \end{RLtext}
            {UNIVERSITY OF MUSTAPHA STAMBOULI MASCARA}
            \begin{figure}[h]
            	\centering
            		\includegraphics[width=4cm]{figurs/logouniv.jpeg}
            \end{figure}
            \begin{RLtext}
                klyT Al`lwm AldaqyqaT
            \end{RLtext}
            {FACULTY OF EXACT SCIENCES}\\
            {COMPUTER SCIENCE DEPARTMENT}\\
            \textsc{\textbf{ \large Master memory} }
        \end{center}
        {\bfseries field :} Mathematics and Computer Science\\
        {\bfseries Faculty  {\hspace*{0.34cm}} :} Computer Science\\
        {\bfseries Option {\hspace*{0.26cm}} :} Networks and distributed system
        \vspace{0.5cm}
        \begin{center}
            \textsc{\textbf{ \large Realized by :}}\\
            {\bfseries Elkaim Moulay Abdellah }{\hspace*{5cm}}{\bfseries Mazouz  Nail}   \\
            \vspace{1cm}
            \Large {\bfseries Theme }
        \end{center}
        \begin{center}
            \hrule width 460pt
            \bigskip
            \Large  \centering \textbf{ \textsc{ ................. } }
            \bigskip
            \hrule width 460pt
        \end{center}
        \raggedright
        \bigskip
        \vspace{1cm}
        \textit{\bfseries Proposed by : Dr. Madber Hayat }
        \vspace{2cm}
        \begin{center}
            \textit{\bfseries College year 2019/2020}
        \end{center}
    \end{titlepage}

    %%Dedicaces%%%%%%%%%%%%%%%%%%%%%%%%%%%%%%%%%%%%%%%%%%%%%%%%%%%%%%%%%%%%%%%%%%%%%%%%%%%%%%%%%%%%%%%%%%%%%%%%%%%%%%%%%%%%%%%%

    \thispagestyle{empty}
    \begin{figure}[h]
    	\centering
    		\includegraphics[width=14cm]{figurs/B.PNG}
    \end{figure}
    \newpage
    \thispagestyle{empty}
    \clearpage
    \newpage
    \pagenumbering{roman}
    \begin{center}
        \textbf{\\}
        \textbf{\\}
        \textbf{\huge \textsc{\itshape Dedications}}\\
        \textbf{\\}
        \textbf{\\}
        \begin{flushleft}
            \textsf{\qquad I dedicate this work to ............. }\\
        \end{flushleft}
    \end{center}
    \begin{flushright}
        \textbf{\textsc{\itshape Elkaim Moulay Abdellah}}
    \end{flushright}
    \newpage
    \begin{center}
        \textbf{\\}
        \textbf{\\}
        \textbf{\huge \textsc{\itshape Dedications}}\\
        \textbf{\\}
        \textbf{\\}

        \begin{flushleft}
            \textsf{\qquad I dedicate this work to ...................... }
        \end{flushleft}

        \normalsize{\itshape .....}
        \textbf{\\}
        \textbf{\\}
    \end{center}
    \begin{flushright}
        \textbf{\textsc{\itshape Mazouz Nail}}
    \end{flushright}

    %%Remerciement%%%%%%%%%%%%%%%%%%%%%%%%%%%%%%%%%%%%%%%%%%%%%%%%%%%%%%%%%%%%%%%%%%%%%%%%%%%%%%%%%%%%%%%%%%%%%%%%%%%%%%%%%%%%%%%%
    \newpage
    \begin{center}
        %\thispagestyle{myheadings}
        %\markboth{droite}{ }
        \textbf{\huge \textsc{\itshape thanks}}
    \end{center}
    \textsf{\qquad We thank .......... }

    %%Abstract %%%%%%%%%%%%%%%%%%%%%%%%%%%%%%%%%%%%%%%%%%%%%%%%%%%%%%%%%%%%%%%%%%%%%%%%%%%%%%%%%%%%%%%%%%%%%%%%%%%%%%%%%%%%%%%%
    \newpage
    \markboth{droite}{ }
    \begin{center}
    \textbf{\huge \textsc{\itshape \textit Abstract}}\\
    \end{center}
    \qquad ...........................

    %%resume %%%%%%%%%%%%%%%%%%%%%%%%%%%%%%%%%%%%%%%%%%%%%%%%%%%%%%%%%%%%%%%%%%%%%%%%%%%%%%%%%%%%%%%%%%%%%%%%%%%%%%%%%%%%%%%%
    \newpage
    \newcommand{\enteteresume}{\markboth{Resume}{Resume}} %il faut ajouter les commandes
    \begin{center}
    \textbf{\huge \textsc{\itshape \textit Résumé}}\\
    \end{center}
    \qquad ...............................

    % La table des matieres%%%%%%%%%%%%%%%%%%%%%%%%%%%%%%%%%%%%%%%%%%%%%%%%%%%%%%%%%%%%%%%%%%%%%%%%%%%%%%%%%%%%%%%%%%%%%%%%%%%%%%%%%%%%%%%%%%%%%%%%%%%%%%%%%%%%%
    \tableofcontents
    \listoffigures
    \listoftables

    %% introduction generale%%%%%%%%%%%%%%%%%%%%%%%%%%%%%%%%%%%%%%%%%%%%%%%%%%%%%%%%%%%%%%%%%%%%%%%%%%%%%%%%%%%%%%%%%%%%%%%%%%%%%%%%%%%%%%%%%%%%%%%%%%%%%%%%%%%%%%%%%%%
    \chapter*{General Introduction}
    \addcontentsline{toc}{chapter}{Introduction générale}
    \setcounter{page}{1}
    \lhead{}
    \cfoot{\bfseries \thepage}
    \rhead{Introduction générale}
    \pagenumbering{arabic}

    ..................................................

    %% chapitre 1%%%%%%%%%%%%%%%%%%%%%%%%%%%%%%%%%%%%%%%%%%%%%%%%%%%%%%%%%%%%%%%%%%%%%%%%%%%%%%%%%%%%%%%%%%%%%%%%%%%%%%%%%%%%%%%%%%%%%%%%%%%%%%%%%%%%%%%%%%%%%%%%%%%%%
    \include{chapiter1/main}

    %% chapitre 2%%%%%%%%%%%%%%%%%%%%%%%%%%%%%%%%%%%%%%%%%%%%%%%%%%%%%%%%%%%%%%%%%%%%%%%%%%%%%%%%%%%%%%%%%%%%%%%%%%%%%%%%%%%%%%%%%%%%%%%%%%%%%%%%%%%%%%%%%%%%%%%%%%%%%
    \include{chapiter2/main}

    %% conclusion generale%%%%%%%%%%%%%%%%%%%%%%%%%%%%%%%%%%%%%%%%%%%%%%%%%%%%%%%%%%%%%%%%%%%%%%%%%%%%%%%%%%%%%%%%%%%%%%%%%%%%%%%%%%%%%%%%%%%%%%%%%%%%%%%%%%%%%%%%%%%
    \chapter*{General conclusion}
    \addcontentsline{toc}{chapter}{General conclusion}
    \lhead{}
    \cfoot{\bfseries \thepage}
    \rhead{General conclusion}
    \markboth{droite}{General conclusion}

    .............................................

    %%References%%%%%%%%%%%%%%%%%%%%%%%%%%%%%%%%%%%%%%%%%%%%%%%%%%%%%%%%%%%%%%%%%%%%%%%%%%%%%%%%%%%%%%%%%%%%%%%%%%%%%%%%%%%%%%%%%%%%%%%%%%%%%%
    \clearpage
    \pagestyle{fancy}
    \addcontentsline{toc}{chapter}{Bibliography}
    \begin{thebibliography}{99}
        \lhead{}
        \cfoot{\bfseries \thepage}
        \rhead{Bibliography}

        \bibitem[1]{1}
        H. Altaama,
        Application Mobile Guide,
        Mémoire de Master en Informatique,
        Université Abou Bakr Belkaid de Tlemcen,
        2016.
        \bibitem[2]{2}
        \url{http://generationmobiles.net/2014/11/les-differents-types-dapps-mobiles/}, [consulté le 14/03/2018].

    \end{thebibliography}

\end{document}

    %% conclusion generale%%%%%%%%%%%%%%%%%%%%%%%%%%%%%%%%%%%%%%%%%%%%%%%%%%%%%%%%%%%%%%%%%%%%%%%%%%%%%%%%%%%%%%%%%%%%%%%%%%%%%%%%%%%%%%%%%%%%%%%%%%%%%%%%%%%%%%%%%%%
    \chapter*{General conclusion}
    \addcontentsline{toc}{chapter}{General conclusion}
    \lhead{}
    \cfoot{\bfseries \thepage}
    \rhead{General conclusion}
    \markboth{droite}{General conclusion}

    .............................................

    %%References%%%%%%%%%%%%%%%%%%%%%%%%%%%%%%%%%%%%%%%%%%%%%%%%%%%%%%%%%%%%%%%%%%%%%%%%%%%%%%%%%%%%%%%%%%%%%%%%%%%%%%%%%%%%%%%%%%%%%%%%%%%%%%
    \clearpage
    \pagestyle{fancy}
    \addcontentsline{toc}{chapter}{Bibliography}
    \begin{thebibliography}{99}
        \lhead{}
        \cfoot{\bfseries \thepage}
        \rhead{Bibliography}

        \bibitem[1]{1}
        H. Altaama,
        Application Mobile Guide,
        Mémoire de Master en Informatique,
        Université Abou Bakr Belkaid de Tlemcen,
        2016.
        \bibitem[2]{2}
        \url{http://generationmobiles.net/2014/11/les-differents-types-dapps-mobiles/}, [consulté le 14/03/2018].

    \end{thebibliography}

\end{document}

    %% chapitre 2%%%%%%%%%%%%%%%%%%%%%%%%%%%%%%%%%%%%%%%%%%%%%%%%%%%%%%%%%%%%%%%%%%%%%%%%%%%%%%%%%%%%%%%%%%%%%%%%%%%%%%%%%%%%%%%%%%%%%%%%%%%%%%%%%%%%%%%%%%%%%%%%%%%%%
    %! Author = moulay
%! Date = 10/21/19

% Preamble
\documentclass[english,a4,12pt]{report}

% Packages
\usepackage{amsmath}
\usepackage[utf8]{inputenc}
\usepackage{xcolor}
\usepackage[T1]{fontenc}
\usepackage{arabtex}
\usepackage{sectsty}
\usepackage[cyr]{aeguill}
\usepackage{rotating}
\usepackage{multirow}
\usepackage{tabulary}
\usepackage{tabularht}
\usepackage{acronym}
\usepackage{fancyhdr}
\usepackage{lscape}
\usepackage{amssymb}
\usepackage{pifont}
\usepackage[most]{tcolorbox}
\usepackage{slashbox}
\usepackage{multido}
\usepackage{caption}
\usepackage{graphicx,wrapfig,lipsum}
\usepackage{enumitem}
\usepackage {fancybox}
\usepackage{array,tabularx}
\usepackage{colortbl}
\usepackage[noend]{algorithmic}
\usepackage[linesnumbered,ruled,vlined,boxed,commentsnumbered]{algorithm2e}
\DeclareGraphicsExtensions{.jpg,.pdf,.PNG,.gif}
\usepackage[pdftex,colorlinks=true,linkcolor=black,citecolor=black,urlcolor=black]{hyperref}
\usepackage{pgfplots}
\usepackage{pgfplotstable}
\usepackage{subcaption}
\usepackage[tikz]{bclogo}
\pgfplotsset{grid style={dashed,gray}}
\pgfplotsset{minor grid style={dotted,green!50!black}}
\pgfplotsset{major grid style={dotted,green!50!black}}
\usepackage{anysize}
\marginsize{30mm}{20mm}{15mm}{15mm}
\newcolumntype{Y}{>{\raggedleft\arraybackslash}X}
\renewcommand{\baselinestretch}{1.5}
\newcolumntype{P}[1]{>{\raggedright}p{#1}}
\newcolumntype{M}[1]{>{\raggedright}m{#1}}%%declaration de page de garde
\newcommand*\rfrac[2]{{}^{#1}\!/_{#2}}
\setcounter{secnumdepth}{3}
\setcounter{tocdepth}{3}
\newcommand{\cmark}{\ding{51}}%
\newcommand{\xmark}{\ding{59}}%

% Document
\begin{document}
    \sloppy
    \begin{titlepage}
        \renewcommand{\baselinestretch}{1}
        \begin{center}
            \begin{RLtext}
                AljmhwryT AljzA'iryT AldymqrA.tyT Al^s`byT
            \end{RLtext}
            {PEOPLE'S DEMOCRATIC REPUBLIC OF ALGERIA}
            \begin{RLtext}
            {wzArT Alt`lym Al`Aly w Alb.h_t Al`lmy}
            \end{RLtext}
            {DEPARTMENT OF HIGHER EDUCATION AND SCIENTIFIC RESEARCH}
            \begin{RLtext}
                jAm`T mus.taf_A s.tmbOly bim`skar
            \end{RLtext}
            {UNIVERSITY OF MUSTAPHA STAMBOULI MASCARA}
            \begin{figure}[h]
            	\centering
            		\includegraphics[width=4cm]{figurs/logouniv.jpeg}
            \end{figure}
            \begin{RLtext}
                klyT Al`lwm AldaqyqaT
            \end{RLtext}
            {FACULTY OF EXACT SCIENCES}\\
            {COMPUTER SCIENCE DEPARTMENT}\\
            \textsc{\textbf{ \large Master memory} }
        \end{center}
        {\bfseries field :} Mathematics and Computer Science\\
        {\bfseries Faculty  {\hspace*{0.34cm}} :} Computer Science\\
        {\bfseries Option {\hspace*{0.26cm}} :} Networks and distributed system
        \vspace{0.5cm}
        \begin{center}
            \textsc{\textbf{ \large Realized by :}}\\
            {\bfseries Elkaim Moulay Abdellah }{\hspace*{5cm}}{\bfseries Mazouz  Nail}   \\
            \vspace{1cm}
            \Large {\bfseries Theme }
        \end{center}
        \begin{center}
            \hrule width 460pt
            \bigskip
            \Large  \centering \textbf{ \textsc{ ................. } }
            \bigskip
            \hrule width 460pt
        \end{center}
        \raggedright
        \bigskip
        \vspace{1cm}
        \textit{\bfseries Proposed by : Dr. Madber Hayat }
        \vspace{2cm}
        \begin{center}
            \textit{\bfseries College year 2019/2020}
        \end{center}
    \end{titlepage}

    %%Dedicaces%%%%%%%%%%%%%%%%%%%%%%%%%%%%%%%%%%%%%%%%%%%%%%%%%%%%%%%%%%%%%%%%%%%%%%%%%%%%%%%%%%%%%%%%%%%%%%%%%%%%%%%%%%%%%%%%

    \thispagestyle{empty}
    \begin{figure}[h]
    	\centering
    		\includegraphics[width=14cm]{figurs/B.PNG}
    \end{figure}
    \newpage
    \thispagestyle{empty}
    \clearpage
    \newpage
    \pagenumbering{roman}
    \begin{center}
        \textbf{\\}
        \textbf{\\}
        \textbf{\huge \textsc{\itshape Dedications}}\\
        \textbf{\\}
        \textbf{\\}
        \begin{flushleft}
            \textsf{\qquad I dedicate this work to ............. }\\
        \end{flushleft}
    \end{center}
    \begin{flushright}
        \textbf{\textsc{\itshape Elkaim Moulay Abdellah}}
    \end{flushright}
    \newpage
    \begin{center}
        \textbf{\\}
        \textbf{\\}
        \textbf{\huge \textsc{\itshape Dedications}}\\
        \textbf{\\}
        \textbf{\\}

        \begin{flushleft}
            \textsf{\qquad I dedicate this work to ...................... }
        \end{flushleft}

        \normalsize{\itshape .....}
        \textbf{\\}
        \textbf{\\}
    \end{center}
    \begin{flushright}
        \textbf{\textsc{\itshape Mazouz Nail}}
    \end{flushright}

    %%Remerciement%%%%%%%%%%%%%%%%%%%%%%%%%%%%%%%%%%%%%%%%%%%%%%%%%%%%%%%%%%%%%%%%%%%%%%%%%%%%%%%%%%%%%%%%%%%%%%%%%%%%%%%%%%%%%%%%
    \newpage
    \begin{center}
        %\thispagestyle{myheadings}
        %\markboth{droite}{ }
        \textbf{\huge \textsc{\itshape thanks}}
    \end{center}
    \textsf{\qquad We thank .......... }

    %%Abstract %%%%%%%%%%%%%%%%%%%%%%%%%%%%%%%%%%%%%%%%%%%%%%%%%%%%%%%%%%%%%%%%%%%%%%%%%%%%%%%%%%%%%%%%%%%%%%%%%%%%%%%%%%%%%%%%
    \newpage
    \markboth{droite}{ }
    \begin{center}
    \textbf{\huge \textsc{\itshape \textit Abstract}}\\
    \end{center}
    \qquad ...........................

    %%resume %%%%%%%%%%%%%%%%%%%%%%%%%%%%%%%%%%%%%%%%%%%%%%%%%%%%%%%%%%%%%%%%%%%%%%%%%%%%%%%%%%%%%%%%%%%%%%%%%%%%%%%%%%%%%%%%
    \newpage
    \newcommand{\enteteresume}{\markboth{Resume}{Resume}} %il faut ajouter les commandes
    \begin{center}
    \textbf{\huge \textsc{\itshape \textit Résumé}}\\
    \end{center}
    \qquad ...............................

    % La table des matieres%%%%%%%%%%%%%%%%%%%%%%%%%%%%%%%%%%%%%%%%%%%%%%%%%%%%%%%%%%%%%%%%%%%%%%%%%%%%%%%%%%%%%%%%%%%%%%%%%%%%%%%%%%%%%%%%%%%%%%%%%%%%%%%%%%%%%
    \tableofcontents
    \listoffigures
    \listoftables

    %% introduction generale%%%%%%%%%%%%%%%%%%%%%%%%%%%%%%%%%%%%%%%%%%%%%%%%%%%%%%%%%%%%%%%%%%%%%%%%%%%%%%%%%%%%%%%%%%%%%%%%%%%%%%%%%%%%%%%%%%%%%%%%%%%%%%%%%%%%%%%%%%%
    \chapter*{General Introduction}
    \addcontentsline{toc}{chapter}{Introduction générale}
    \setcounter{page}{1}
    \lhead{}
    \cfoot{\bfseries \thepage}
    \rhead{Introduction générale}
    \pagenumbering{arabic}

    ..................................................

    %% chapitre 1%%%%%%%%%%%%%%%%%%%%%%%%%%%%%%%%%%%%%%%%%%%%%%%%%%%%%%%%%%%%%%%%%%%%%%%%%%%%%%%%%%%%%%%%%%%%%%%%%%%%%%%%%%%%%%%%%%%%%%%%%%%%%%%%%%%%%%%%%%%%%%%%%%%%%
    %! Author = moulay
%! Date = 10/21/19

% Preamble
\documentclass[english,a4,12pt]{report}

% Packages
\usepackage{amsmath}
\usepackage[utf8]{inputenc}
\usepackage{xcolor}
\usepackage[T1]{fontenc}
\usepackage{arabtex}
\usepackage{sectsty}
\usepackage[cyr]{aeguill}
\usepackage{rotating}
\usepackage{multirow}
\usepackage{tabulary}
\usepackage{tabularht}
\usepackage{acronym}
\usepackage{fancyhdr}
\usepackage{lscape}
\usepackage{amssymb}
\usepackage{pifont}
\usepackage[most]{tcolorbox}
\usepackage{slashbox}
\usepackage{multido}
\usepackage{caption}
\usepackage{graphicx,wrapfig,lipsum}
\usepackage{enumitem}
\usepackage {fancybox}
\usepackage{array,tabularx}
\usepackage{colortbl}
\usepackage[noend]{algorithmic}
\usepackage[linesnumbered,ruled,vlined,boxed,commentsnumbered]{algorithm2e}
\DeclareGraphicsExtensions{.jpg,.pdf,.PNG,.gif}
\usepackage[pdftex,colorlinks=true,linkcolor=black,citecolor=black,urlcolor=black]{hyperref}
\usepackage{pgfplots}
\usepackage{pgfplotstable}
\usepackage{subcaption}
\usepackage[tikz]{bclogo}
\pgfplotsset{grid style={dashed,gray}}
\pgfplotsset{minor grid style={dotted,green!50!black}}
\pgfplotsset{major grid style={dotted,green!50!black}}
\usepackage{anysize}
\marginsize{30mm}{20mm}{15mm}{15mm}
\newcolumntype{Y}{>{\raggedleft\arraybackslash}X}
\renewcommand{\baselinestretch}{1.5}
\newcolumntype{P}[1]{>{\raggedright}p{#1}}
\newcolumntype{M}[1]{>{\raggedright}m{#1}}%%declaration de page de garde
\newcommand*\rfrac[2]{{}^{#1}\!/_{#2}}
\setcounter{secnumdepth}{3}
\setcounter{tocdepth}{3}
\newcommand{\cmark}{\ding{51}}%
\newcommand{\xmark}{\ding{59}}%

% Document
\begin{document}
    \sloppy
    \begin{titlepage}
        \renewcommand{\baselinestretch}{1}
        \begin{center}
            \begin{RLtext}
                AljmhwryT AljzA'iryT AldymqrA.tyT Al^s`byT
            \end{RLtext}
            {PEOPLE'S DEMOCRATIC REPUBLIC OF ALGERIA}
            \begin{RLtext}
            {wzArT Alt`lym Al`Aly w Alb.h_t Al`lmy}
            \end{RLtext}
            {DEPARTMENT OF HIGHER EDUCATION AND SCIENTIFIC RESEARCH}
            \begin{RLtext}
                jAm`T mus.taf_A s.tmbOly bim`skar
            \end{RLtext}
            {UNIVERSITY OF MUSTAPHA STAMBOULI MASCARA}
            \begin{figure}[h]
            	\centering
            		\includegraphics[width=4cm]{figurs/logouniv.jpeg}
            \end{figure}
            \begin{RLtext}
                klyT Al`lwm AldaqyqaT
            \end{RLtext}
            {FACULTY OF EXACT SCIENCES}\\
            {COMPUTER SCIENCE DEPARTMENT}\\
            \textsc{\textbf{ \large Master memory} }
        \end{center}
        {\bfseries field :} Mathematics and Computer Science\\
        {\bfseries Faculty  {\hspace*{0.34cm}} :} Computer Science\\
        {\bfseries Option {\hspace*{0.26cm}} :} Networks and distributed system
        \vspace{0.5cm}
        \begin{center}
            \textsc{\textbf{ \large Realized by :}}\\
            {\bfseries Elkaim Moulay Abdellah }{\hspace*{5cm}}{\bfseries Mazouz  Nail}   \\
            \vspace{1cm}
            \Large {\bfseries Theme }
        \end{center}
        \begin{center}
            \hrule width 460pt
            \bigskip
            \Large  \centering \textbf{ \textsc{ ................. } }
            \bigskip
            \hrule width 460pt
        \end{center}
        \raggedright
        \bigskip
        \vspace{1cm}
        \textit{\bfseries Proposed by : Dr. Madber Hayat }
        \vspace{2cm}
        \begin{center}
            \textit{\bfseries College year 2019/2020}
        \end{center}
    \end{titlepage}

    %%Dedicaces%%%%%%%%%%%%%%%%%%%%%%%%%%%%%%%%%%%%%%%%%%%%%%%%%%%%%%%%%%%%%%%%%%%%%%%%%%%%%%%%%%%%%%%%%%%%%%%%%%%%%%%%%%%%%%%%

    \thispagestyle{empty}
    \begin{figure}[h]
    	\centering
    		\includegraphics[width=14cm]{figurs/B.PNG}
    \end{figure}
    \newpage
    \thispagestyle{empty}
    \clearpage
    \newpage
    \pagenumbering{roman}
    \begin{center}
        \textbf{\\}
        \textbf{\\}
        \textbf{\huge \textsc{\itshape Dedications}}\\
        \textbf{\\}
        \textbf{\\}
        \begin{flushleft}
            \textsf{\qquad I dedicate this work to ............. }\\
        \end{flushleft}
    \end{center}
    \begin{flushright}
        \textbf{\textsc{\itshape Elkaim Moulay Abdellah}}
    \end{flushright}
    \newpage
    \begin{center}
        \textbf{\\}
        \textbf{\\}
        \textbf{\huge \textsc{\itshape Dedications}}\\
        \textbf{\\}
        \textbf{\\}

        \begin{flushleft}
            \textsf{\qquad I dedicate this work to ...................... }
        \end{flushleft}

        \normalsize{\itshape .....}
        \textbf{\\}
        \textbf{\\}
    \end{center}
    \begin{flushright}
        \textbf{\textsc{\itshape Mazouz Nail}}
    \end{flushright}

    %%Remerciement%%%%%%%%%%%%%%%%%%%%%%%%%%%%%%%%%%%%%%%%%%%%%%%%%%%%%%%%%%%%%%%%%%%%%%%%%%%%%%%%%%%%%%%%%%%%%%%%%%%%%%%%%%%%%%%%
    \newpage
    \begin{center}
        %\thispagestyle{myheadings}
        %\markboth{droite}{ }
        \textbf{\huge \textsc{\itshape thanks}}
    \end{center}
    \textsf{\qquad We thank .......... }

    %%Abstract %%%%%%%%%%%%%%%%%%%%%%%%%%%%%%%%%%%%%%%%%%%%%%%%%%%%%%%%%%%%%%%%%%%%%%%%%%%%%%%%%%%%%%%%%%%%%%%%%%%%%%%%%%%%%%%%
    \newpage
    \markboth{droite}{ }
    \begin{center}
    \textbf{\huge \textsc{\itshape \textit Abstract}}\\
    \end{center}
    \qquad ...........................

    %%resume %%%%%%%%%%%%%%%%%%%%%%%%%%%%%%%%%%%%%%%%%%%%%%%%%%%%%%%%%%%%%%%%%%%%%%%%%%%%%%%%%%%%%%%%%%%%%%%%%%%%%%%%%%%%%%%%
    \newpage
    \newcommand{\enteteresume}{\markboth{Resume}{Resume}} %il faut ajouter les commandes
    \begin{center}
    \textbf{\huge \textsc{\itshape \textit Résumé}}\\
    \end{center}
    \qquad ...............................

    % La table des matieres%%%%%%%%%%%%%%%%%%%%%%%%%%%%%%%%%%%%%%%%%%%%%%%%%%%%%%%%%%%%%%%%%%%%%%%%%%%%%%%%%%%%%%%%%%%%%%%%%%%%%%%%%%%%%%%%%%%%%%%%%%%%%%%%%%%%%
    \tableofcontents
    \listoffigures
    \listoftables

    %% introduction generale%%%%%%%%%%%%%%%%%%%%%%%%%%%%%%%%%%%%%%%%%%%%%%%%%%%%%%%%%%%%%%%%%%%%%%%%%%%%%%%%%%%%%%%%%%%%%%%%%%%%%%%%%%%%%%%%%%%%%%%%%%%%%%%%%%%%%%%%%%%
    \chapter*{General Introduction}
    \addcontentsline{toc}{chapter}{Introduction générale}
    \setcounter{page}{1}
    \lhead{}
    \cfoot{\bfseries \thepage}
    \rhead{Introduction générale}
    \pagenumbering{arabic}

    ..................................................

    %% chapitre 1%%%%%%%%%%%%%%%%%%%%%%%%%%%%%%%%%%%%%%%%%%%%%%%%%%%%%%%%%%%%%%%%%%%%%%%%%%%%%%%%%%%%%%%%%%%%%%%%%%%%%%%%%%%%%%%%%%%%%%%%%%%%%%%%%%%%%%%%%%%%%%%%%%%%%
    \include{chapiter1/main}

    %% chapitre 2%%%%%%%%%%%%%%%%%%%%%%%%%%%%%%%%%%%%%%%%%%%%%%%%%%%%%%%%%%%%%%%%%%%%%%%%%%%%%%%%%%%%%%%%%%%%%%%%%%%%%%%%%%%%%%%%%%%%%%%%%%%%%%%%%%%%%%%%%%%%%%%%%%%%%
    \include{chapiter2/main}

    %% conclusion generale%%%%%%%%%%%%%%%%%%%%%%%%%%%%%%%%%%%%%%%%%%%%%%%%%%%%%%%%%%%%%%%%%%%%%%%%%%%%%%%%%%%%%%%%%%%%%%%%%%%%%%%%%%%%%%%%%%%%%%%%%%%%%%%%%%%%%%%%%%%
    \chapter*{General conclusion}
    \addcontentsline{toc}{chapter}{General conclusion}
    \lhead{}
    \cfoot{\bfseries \thepage}
    \rhead{General conclusion}
    \markboth{droite}{General conclusion}

    .............................................

    %%References%%%%%%%%%%%%%%%%%%%%%%%%%%%%%%%%%%%%%%%%%%%%%%%%%%%%%%%%%%%%%%%%%%%%%%%%%%%%%%%%%%%%%%%%%%%%%%%%%%%%%%%%%%%%%%%%%%%%%%%%%%%%%%
    \clearpage
    \pagestyle{fancy}
    \addcontentsline{toc}{chapter}{Bibliography}
    \begin{thebibliography}{99}
        \lhead{}
        \cfoot{\bfseries \thepage}
        \rhead{Bibliography}

        \bibitem[1]{1}
        H. Altaama,
        Application Mobile Guide,
        Mémoire de Master en Informatique,
        Université Abou Bakr Belkaid de Tlemcen,
        2016.
        \bibitem[2]{2}
        \url{http://generationmobiles.net/2014/11/les-differents-types-dapps-mobiles/}, [consulté le 14/03/2018].

    \end{thebibliography}

\end{document}

    %% chapitre 2%%%%%%%%%%%%%%%%%%%%%%%%%%%%%%%%%%%%%%%%%%%%%%%%%%%%%%%%%%%%%%%%%%%%%%%%%%%%%%%%%%%%%%%%%%%%%%%%%%%%%%%%%%%%%%%%%%%%%%%%%%%%%%%%%%%%%%%%%%%%%%%%%%%%%
    %! Author = moulay
%! Date = 10/21/19

% Preamble
\documentclass[english,a4,12pt]{report}

% Packages
\usepackage{amsmath}
\usepackage[utf8]{inputenc}
\usepackage{xcolor}
\usepackage[T1]{fontenc}
\usepackage{arabtex}
\usepackage{sectsty}
\usepackage[cyr]{aeguill}
\usepackage{rotating}
\usepackage{multirow}
\usepackage{tabulary}
\usepackage{tabularht}
\usepackage{acronym}
\usepackage{fancyhdr}
\usepackage{lscape}
\usepackage{amssymb}
\usepackage{pifont}
\usepackage[most]{tcolorbox}
\usepackage{slashbox}
\usepackage{multido}
\usepackage{caption}
\usepackage{graphicx,wrapfig,lipsum}
\usepackage{enumitem}
\usepackage {fancybox}
\usepackage{array,tabularx}
\usepackage{colortbl}
\usepackage[noend]{algorithmic}
\usepackage[linesnumbered,ruled,vlined,boxed,commentsnumbered]{algorithm2e}
\DeclareGraphicsExtensions{.jpg,.pdf,.PNG,.gif}
\usepackage[pdftex,colorlinks=true,linkcolor=black,citecolor=black,urlcolor=black]{hyperref}
\usepackage{pgfplots}
\usepackage{pgfplotstable}
\usepackage{subcaption}
\usepackage[tikz]{bclogo}
\pgfplotsset{grid style={dashed,gray}}
\pgfplotsset{minor grid style={dotted,green!50!black}}
\pgfplotsset{major grid style={dotted,green!50!black}}
\usepackage{anysize}
\marginsize{30mm}{20mm}{15mm}{15mm}
\newcolumntype{Y}{>{\raggedleft\arraybackslash}X}
\renewcommand{\baselinestretch}{1.5}
\newcolumntype{P}[1]{>{\raggedright}p{#1}}
\newcolumntype{M}[1]{>{\raggedright}m{#1}}%%declaration de page de garde
\newcommand*\rfrac[2]{{}^{#1}\!/_{#2}}
\setcounter{secnumdepth}{3}
\setcounter{tocdepth}{3}
\newcommand{\cmark}{\ding{51}}%
\newcommand{\xmark}{\ding{59}}%

% Document
\begin{document}
    \sloppy
    \begin{titlepage}
        \renewcommand{\baselinestretch}{1}
        \begin{center}
            \begin{RLtext}
                AljmhwryT AljzA'iryT AldymqrA.tyT Al^s`byT
            \end{RLtext}
            {PEOPLE'S DEMOCRATIC REPUBLIC OF ALGERIA}
            \begin{RLtext}
            {wzArT Alt`lym Al`Aly w Alb.h_t Al`lmy}
            \end{RLtext}
            {DEPARTMENT OF HIGHER EDUCATION AND SCIENTIFIC RESEARCH}
            \begin{RLtext}
                jAm`T mus.taf_A s.tmbOly bim`skar
            \end{RLtext}
            {UNIVERSITY OF MUSTAPHA STAMBOULI MASCARA}
            \begin{figure}[h]
            	\centering
            		\includegraphics[width=4cm]{figurs/logouniv.jpeg}
            \end{figure}
            \begin{RLtext}
                klyT Al`lwm AldaqyqaT
            \end{RLtext}
            {FACULTY OF EXACT SCIENCES}\\
            {COMPUTER SCIENCE DEPARTMENT}\\
            \textsc{\textbf{ \large Master memory} }
        \end{center}
        {\bfseries field :} Mathematics and Computer Science\\
        {\bfseries Faculty  {\hspace*{0.34cm}} :} Computer Science\\
        {\bfseries Option {\hspace*{0.26cm}} :} Networks and distributed system
        \vspace{0.5cm}
        \begin{center}
            \textsc{\textbf{ \large Realized by :}}\\
            {\bfseries Elkaim Moulay Abdellah }{\hspace*{5cm}}{\bfseries Mazouz  Nail}   \\
            \vspace{1cm}
            \Large {\bfseries Theme }
        \end{center}
        \begin{center}
            \hrule width 460pt
            \bigskip
            \Large  \centering \textbf{ \textsc{ ................. } }
            \bigskip
            \hrule width 460pt
        \end{center}
        \raggedright
        \bigskip
        \vspace{1cm}
        \textit{\bfseries Proposed by : Dr. Madber Hayat }
        \vspace{2cm}
        \begin{center}
            \textit{\bfseries College year 2019/2020}
        \end{center}
    \end{titlepage}

    %%Dedicaces%%%%%%%%%%%%%%%%%%%%%%%%%%%%%%%%%%%%%%%%%%%%%%%%%%%%%%%%%%%%%%%%%%%%%%%%%%%%%%%%%%%%%%%%%%%%%%%%%%%%%%%%%%%%%%%%

    \thispagestyle{empty}
    \begin{figure}[h]
    	\centering
    		\includegraphics[width=14cm]{figurs/B.PNG}
    \end{figure}
    \newpage
    \thispagestyle{empty}
    \clearpage
    \newpage
    \pagenumbering{roman}
    \begin{center}
        \textbf{\\}
        \textbf{\\}
        \textbf{\huge \textsc{\itshape Dedications}}\\
        \textbf{\\}
        \textbf{\\}
        \begin{flushleft}
            \textsf{\qquad I dedicate this work to ............. }\\
        \end{flushleft}
    \end{center}
    \begin{flushright}
        \textbf{\textsc{\itshape Elkaim Moulay Abdellah}}
    \end{flushright}
    \newpage
    \begin{center}
        \textbf{\\}
        \textbf{\\}
        \textbf{\huge \textsc{\itshape Dedications}}\\
        \textbf{\\}
        \textbf{\\}

        \begin{flushleft}
            \textsf{\qquad I dedicate this work to ...................... }
        \end{flushleft}

        \normalsize{\itshape .....}
        \textbf{\\}
        \textbf{\\}
    \end{center}
    \begin{flushright}
        \textbf{\textsc{\itshape Mazouz Nail}}
    \end{flushright}

    %%Remerciement%%%%%%%%%%%%%%%%%%%%%%%%%%%%%%%%%%%%%%%%%%%%%%%%%%%%%%%%%%%%%%%%%%%%%%%%%%%%%%%%%%%%%%%%%%%%%%%%%%%%%%%%%%%%%%%%
    \newpage
    \begin{center}
        %\thispagestyle{myheadings}
        %\markboth{droite}{ }
        \textbf{\huge \textsc{\itshape thanks}}
    \end{center}
    \textsf{\qquad We thank .......... }

    %%Abstract %%%%%%%%%%%%%%%%%%%%%%%%%%%%%%%%%%%%%%%%%%%%%%%%%%%%%%%%%%%%%%%%%%%%%%%%%%%%%%%%%%%%%%%%%%%%%%%%%%%%%%%%%%%%%%%%
    \newpage
    \markboth{droite}{ }
    \begin{center}
    \textbf{\huge \textsc{\itshape \textit Abstract}}\\
    \end{center}
    \qquad ...........................

    %%resume %%%%%%%%%%%%%%%%%%%%%%%%%%%%%%%%%%%%%%%%%%%%%%%%%%%%%%%%%%%%%%%%%%%%%%%%%%%%%%%%%%%%%%%%%%%%%%%%%%%%%%%%%%%%%%%%
    \newpage
    \newcommand{\enteteresume}{\markboth{Resume}{Resume}} %il faut ajouter les commandes
    \begin{center}
    \textbf{\huge \textsc{\itshape \textit Résumé}}\\
    \end{center}
    \qquad ...............................

    % La table des matieres%%%%%%%%%%%%%%%%%%%%%%%%%%%%%%%%%%%%%%%%%%%%%%%%%%%%%%%%%%%%%%%%%%%%%%%%%%%%%%%%%%%%%%%%%%%%%%%%%%%%%%%%%%%%%%%%%%%%%%%%%%%%%%%%%%%%%
    \tableofcontents
    \listoffigures
    \listoftables

    %% introduction generale%%%%%%%%%%%%%%%%%%%%%%%%%%%%%%%%%%%%%%%%%%%%%%%%%%%%%%%%%%%%%%%%%%%%%%%%%%%%%%%%%%%%%%%%%%%%%%%%%%%%%%%%%%%%%%%%%%%%%%%%%%%%%%%%%%%%%%%%%%%
    \chapter*{General Introduction}
    \addcontentsline{toc}{chapter}{Introduction générale}
    \setcounter{page}{1}
    \lhead{}
    \cfoot{\bfseries \thepage}
    \rhead{Introduction générale}
    \pagenumbering{arabic}

    ..................................................

    %% chapitre 1%%%%%%%%%%%%%%%%%%%%%%%%%%%%%%%%%%%%%%%%%%%%%%%%%%%%%%%%%%%%%%%%%%%%%%%%%%%%%%%%%%%%%%%%%%%%%%%%%%%%%%%%%%%%%%%%%%%%%%%%%%%%%%%%%%%%%%%%%%%%%%%%%%%%%
    \include{chapiter1/main}

    %% chapitre 2%%%%%%%%%%%%%%%%%%%%%%%%%%%%%%%%%%%%%%%%%%%%%%%%%%%%%%%%%%%%%%%%%%%%%%%%%%%%%%%%%%%%%%%%%%%%%%%%%%%%%%%%%%%%%%%%%%%%%%%%%%%%%%%%%%%%%%%%%%%%%%%%%%%%%
    \include{chapiter2/main}

    %% conclusion generale%%%%%%%%%%%%%%%%%%%%%%%%%%%%%%%%%%%%%%%%%%%%%%%%%%%%%%%%%%%%%%%%%%%%%%%%%%%%%%%%%%%%%%%%%%%%%%%%%%%%%%%%%%%%%%%%%%%%%%%%%%%%%%%%%%%%%%%%%%%
    \chapter*{General conclusion}
    \addcontentsline{toc}{chapter}{General conclusion}
    \lhead{}
    \cfoot{\bfseries \thepage}
    \rhead{General conclusion}
    \markboth{droite}{General conclusion}

    .............................................

    %%References%%%%%%%%%%%%%%%%%%%%%%%%%%%%%%%%%%%%%%%%%%%%%%%%%%%%%%%%%%%%%%%%%%%%%%%%%%%%%%%%%%%%%%%%%%%%%%%%%%%%%%%%%%%%%%%%%%%%%%%%%%%%%%
    \clearpage
    \pagestyle{fancy}
    \addcontentsline{toc}{chapter}{Bibliography}
    \begin{thebibliography}{99}
        \lhead{}
        \cfoot{\bfseries \thepage}
        \rhead{Bibliography}

        \bibitem[1]{1}
        H. Altaama,
        Application Mobile Guide,
        Mémoire de Master en Informatique,
        Université Abou Bakr Belkaid de Tlemcen,
        2016.
        \bibitem[2]{2}
        \url{http://generationmobiles.net/2014/11/les-differents-types-dapps-mobiles/}, [consulté le 14/03/2018].

    \end{thebibliography}

\end{document}

    %% conclusion generale%%%%%%%%%%%%%%%%%%%%%%%%%%%%%%%%%%%%%%%%%%%%%%%%%%%%%%%%%%%%%%%%%%%%%%%%%%%%%%%%%%%%%%%%%%%%%%%%%%%%%%%%%%%%%%%%%%%%%%%%%%%%%%%%%%%%%%%%%%%
    \chapter*{General conclusion}
    \addcontentsline{toc}{chapter}{General conclusion}
    \lhead{}
    \cfoot{\bfseries \thepage}
    \rhead{General conclusion}
    \markboth{droite}{General conclusion}

    .............................................

    %%References%%%%%%%%%%%%%%%%%%%%%%%%%%%%%%%%%%%%%%%%%%%%%%%%%%%%%%%%%%%%%%%%%%%%%%%%%%%%%%%%%%%%%%%%%%%%%%%%%%%%%%%%%%%%%%%%%%%%%%%%%%%%%%
    \clearpage
    \pagestyle{fancy}
    \addcontentsline{toc}{chapter}{Bibliography}
    \begin{thebibliography}{99}
        \lhead{}
        \cfoot{\bfseries \thepage}
        \rhead{Bibliography}

        \bibitem[1]{1}
        H. Altaama,
        Application Mobile Guide,
        Mémoire de Master en Informatique,
        Université Abou Bakr Belkaid de Tlemcen,
        2016.
        \bibitem[2]{2}
        \url{http://generationmobiles.net/2014/11/les-differents-types-dapps-mobiles/}, [consulté le 14/03/2018].

    \end{thebibliography}

\end{document}

    %% conclusion generale%%%%%%%%%%%%%%%%%%%%%%%%%%%%%%%%%%%%%%%%%%%%%%%%%%%%%%%%%%%%%%%%%%%%%%%%%%%%%%%%%%%%%%%%%%%%%%%%%%%%%%%%%%%%%%%%%%%%%%%%%%%%%%%%%%%%%%%%%%%
    \chapter*{General conclusion}
    \addcontentsline{toc}{chapter}{General conclusion}
    \lhead{}
    \cfoot{\bfseries \thepage}
    \rhead{General conclusion}
    \markboth{droite}{General conclusion}

    .............................................

    %%References%%%%%%%%%%%%%%%%%%%%%%%%%%%%%%%%%%%%%%%%%%%%%%%%%%%%%%%%%%%%%%%%%%%%%%%%%%%%%%%%%%%%%%%%%%%%%%%%%%%%%%%%%%%%%%%%%%%%%%%%%%%%%%
    \clearpage
    \pagestyle{fancy}
    \addcontentsline{toc}{chapter}{Bibliography}
    \begin{thebibliography}{99}
        \lhead{}
        \cfoot{\bfseries \thepage}
        \rhead{Bibliography}

        \bibitem[1]{1}
        H. Altaama,
        Application Mobile Guide,
        Mémoire de Master en Informatique,
        Université Abou Bakr Belkaid de Tlemcen,
        2016.
        \bibitem[2]{2}
        \url{http://generationmobiles.net/2014/11/les-differents-types-dapps-mobiles/}, [consulté le 14/03/2018].

    \end{thebibliography}

\end{document}

    %% conclusion generale%%%%%%%%%%%%%%%%%%%%%%%%%%%%%%%%%%%%%%%%%%%%%%%%%%%%%%%%%%%%%%%%%%%%%%%%%%%%%%%%%%%%%%%%%%%%%%%%%%%%%%%%%%%%%%%%%%%%%%%%%%%%%%%%%%%%%%%%%%%
    %! Author = moulay
%! Date = 10/21/19

% Preamble
\documentclass[english,a4,12pt]{report}

% Packages
\usepackage{amsmath}
\usepackage[utf8]{inputenc}
\usepackage{xcolor}
\usepackage[T1]{fontenc}
\usepackage{arabtex}
\usepackage{sectsty}
\usepackage[cyr]{aeguill}
\usepackage{rotating}
\usepackage{multirow}
\usepackage{tabulary}
\usepackage{tabularht}
\usepackage{acronym}
\usepackage{fancyhdr}
\usepackage{lscape}
\usepackage{amssymb}
\usepackage{pifont}
\usepackage[most]{tcolorbox}
\usepackage{slashbox}
\usepackage{multido}
\usepackage{caption}
\usepackage{graphicx,wrapfig,lipsum}
\usepackage{enumitem}
\usepackage {fancybox}
\usepackage{array,tabularx}
\usepackage{colortbl}
\usepackage[noend]{algorithmic}
\usepackage[linesnumbered,ruled,vlined,boxed,commentsnumbered]{algorithm2e}
\DeclareGraphicsExtensions{.jpg,.pdf,.PNG,.gif}
\usepackage[pdftex,colorlinks=true,linkcolor=black,citecolor=black,urlcolor=black]{hyperref}
\usepackage{pgfplots}
\usepackage{pgfplotstable}
\usepackage{subcaption}
\usepackage[tikz]{bclogo}
\pgfplotsset{grid style={dashed,gray}}
\pgfplotsset{minor grid style={dotted,green!50!black}}
\pgfplotsset{major grid style={dotted,green!50!black}}
\usepackage{anysize}
\marginsize{30mm}{20mm}{15mm}{15mm}
\newcolumntype{Y}{>{\raggedleft\arraybackslash}X}
\renewcommand{\baselinestretch}{1.5}
\newcolumntype{P}[1]{>{\raggedright}p{#1}}
\newcolumntype{M}[1]{>{\raggedright}m{#1}}%%declaration de page de garde
\newcommand*\rfrac[2]{{}^{#1}\!/_{#2}}
\setcounter{secnumdepth}{3}
\setcounter{tocdepth}{3}
\newcommand{\cmark}{\ding{51}}%
\newcommand{\xmark}{\ding{59}}%

% Document
\begin{document}
    \sloppy
    \begin{titlepage}
        \renewcommand{\baselinestretch}{1}
        \begin{center}
            \begin{RLtext}
                AljmhwryT AljzA'iryT AldymqrA.tyT Al^s`byT
            \end{RLtext}
            {PEOPLE'S DEMOCRATIC REPUBLIC OF ALGERIA}
            \begin{RLtext}
            {wzArT Alt`lym Al`Aly w Alb.h_t Al`lmy}
            \end{RLtext}
            {DEPARTMENT OF HIGHER EDUCATION AND SCIENTIFIC RESEARCH}
            \begin{RLtext}
                jAm`T mus.taf_A s.tmbOly bim`skar
            \end{RLtext}
            {UNIVERSITY OF MUSTAPHA STAMBOULI MASCARA}
            \begin{figure}[h]
            	\centering
            		\includegraphics[width=4cm]{figurs/logouniv.jpeg}
            \end{figure}
            \begin{RLtext}
                klyT Al`lwm AldaqyqaT
            \end{RLtext}
            {FACULTY OF EXACT SCIENCES}\\
            {COMPUTER SCIENCE DEPARTMENT}\\
            \textsc{\textbf{ \large Master memory} }
        \end{center}
        {\bfseries field :} Mathematics and Computer Science\\
        {\bfseries Faculty  {\hspace*{0.34cm}} :} Computer Science\\
        {\bfseries Option {\hspace*{0.26cm}} :} Networks and distributed system
        \vspace{0.5cm}
        \begin{center}
            \textsc{\textbf{ \large Realized by :}}\\
            {\bfseries Elkaim Moulay Abdellah }{\hspace*{5cm}}{\bfseries Mazouz  Nail}   \\
            \vspace{1cm}
            \Large {\bfseries Theme }
        \end{center}
        \begin{center}
            \hrule width 460pt
            \bigskip
            \Large  \centering \textbf{ \textsc{ ................. } }
            \bigskip
            \hrule width 460pt
        \end{center}
        \raggedright
        \bigskip
        \vspace{1cm}
        \textit{\bfseries Proposed by : Dr. Madber Hayat }
        \vspace{2cm}
        \begin{center}
            \textit{\bfseries College year 2019/2020}
        \end{center}
    \end{titlepage}

    %%Dedicaces%%%%%%%%%%%%%%%%%%%%%%%%%%%%%%%%%%%%%%%%%%%%%%%%%%%%%%%%%%%%%%%%%%%%%%%%%%%%%%%%%%%%%%%%%%%%%%%%%%%%%%%%%%%%%%%%

    \thispagestyle{empty}
    \begin{figure}[h]
    	\centering
    		\includegraphics[width=14cm]{figurs/B.PNG}
    \end{figure}
    \newpage
    \thispagestyle{empty}
    \clearpage
    \newpage
    \pagenumbering{roman}
    \begin{center}
        \textbf{\\}
        \textbf{\\}
        \textbf{\huge \textsc{\itshape Dedications}}\\
        \textbf{\\}
        \textbf{\\}
        \begin{flushleft}
            \textsf{\qquad I dedicate this work to ............. }\\
        \end{flushleft}
    \end{center}
    \begin{flushright}
        \textbf{\textsc{\itshape Elkaim Moulay Abdellah}}
    \end{flushright}
    \newpage
    \begin{center}
        \textbf{\\}
        \textbf{\\}
        \textbf{\huge \textsc{\itshape Dedications}}\\
        \textbf{\\}
        \textbf{\\}

        \begin{flushleft}
            \textsf{\qquad I dedicate this work to ...................... }
        \end{flushleft}

        \normalsize{\itshape .....}
        \textbf{\\}
        \textbf{\\}
    \end{center}
    \begin{flushright}
        \textbf{\textsc{\itshape Mazouz Nail}}
    \end{flushright}

    %%Remerciement%%%%%%%%%%%%%%%%%%%%%%%%%%%%%%%%%%%%%%%%%%%%%%%%%%%%%%%%%%%%%%%%%%%%%%%%%%%%%%%%%%%%%%%%%%%%%%%%%%%%%%%%%%%%%%%%
    \newpage
    \begin{center}
        %\thispagestyle{myheadings}
        %\markboth{droite}{ }
        \textbf{\huge \textsc{\itshape thanks}}
    \end{center}
    \textsf{\qquad We thank .......... }

    %%Abstract %%%%%%%%%%%%%%%%%%%%%%%%%%%%%%%%%%%%%%%%%%%%%%%%%%%%%%%%%%%%%%%%%%%%%%%%%%%%%%%%%%%%%%%%%%%%%%%%%%%%%%%%%%%%%%%%
    \newpage
    \markboth{droite}{ }
    \begin{center}
    \textbf{\huge \textsc{\itshape \textit Abstract}}\\
    \end{center}
    \qquad ...........................

    %%resume %%%%%%%%%%%%%%%%%%%%%%%%%%%%%%%%%%%%%%%%%%%%%%%%%%%%%%%%%%%%%%%%%%%%%%%%%%%%%%%%%%%%%%%%%%%%%%%%%%%%%%%%%%%%%%%%
    \newpage
    \newcommand{\enteteresume}{\markboth{Resume}{Resume}} %il faut ajouter les commandes
    \begin{center}
    \textbf{\huge \textsc{\itshape \textit Résumé}}\\
    \end{center}
    \qquad ...............................

    % La table des matieres%%%%%%%%%%%%%%%%%%%%%%%%%%%%%%%%%%%%%%%%%%%%%%%%%%%%%%%%%%%%%%%%%%%%%%%%%%%%%%%%%%%%%%%%%%%%%%%%%%%%%%%%%%%%%%%%%%%%%%%%%%%%%%%%%%%%%
    \tableofcontents
    \listoffigures
    \listoftables

    %% introduction generale%%%%%%%%%%%%%%%%%%%%%%%%%%%%%%%%%%%%%%%%%%%%%%%%%%%%%%%%%%%%%%%%%%%%%%%%%%%%%%%%%%%%%%%%%%%%%%%%%%%%%%%%%%%%%%%%%%%%%%%%%%%%%%%%%%%%%%%%%%%
    \chapter*{General Introduction}
    \addcontentsline{toc}{chapter}{Introduction générale}
    \setcounter{page}{1}
    \lhead{}
    \cfoot{\bfseries \thepage}
    \rhead{Introduction générale}
    \pagenumbering{arabic}

    ..................................................

    %% chapitre 1%%%%%%%%%%%%%%%%%%%%%%%%%%%%%%%%%%%%%%%%%%%%%%%%%%%%%%%%%%%%%%%%%%%%%%%%%%%%%%%%%%%%%%%%%%%%%%%%%%%%%%%%%%%%%%%%%%%%%%%%%%%%%%%%%%%%%%%%%%%%%%%%%%%%%
    %! Author = moulay
%! Date = 10/21/19

% Preamble
\documentclass[english,a4,12pt]{report}

% Packages
\usepackage{amsmath}
\usepackage[utf8]{inputenc}
\usepackage{xcolor}
\usepackage[T1]{fontenc}
\usepackage{arabtex}
\usepackage{sectsty}
\usepackage[cyr]{aeguill}
\usepackage{rotating}
\usepackage{multirow}
\usepackage{tabulary}
\usepackage{tabularht}
\usepackage{acronym}
\usepackage{fancyhdr}
\usepackage{lscape}
\usepackage{amssymb}
\usepackage{pifont}
\usepackage[most]{tcolorbox}
\usepackage{slashbox}
\usepackage{multido}
\usepackage{caption}
\usepackage{graphicx,wrapfig,lipsum}
\usepackage{enumitem}
\usepackage {fancybox}
\usepackage{array,tabularx}
\usepackage{colortbl}
\usepackage[noend]{algorithmic}
\usepackage[linesnumbered,ruled,vlined,boxed,commentsnumbered]{algorithm2e}
\DeclareGraphicsExtensions{.jpg,.pdf,.PNG,.gif}
\usepackage[pdftex,colorlinks=true,linkcolor=black,citecolor=black,urlcolor=black]{hyperref}
\usepackage{pgfplots}
\usepackage{pgfplotstable}
\usepackage{subcaption}
\usepackage[tikz]{bclogo}
\pgfplotsset{grid style={dashed,gray}}
\pgfplotsset{minor grid style={dotted,green!50!black}}
\pgfplotsset{major grid style={dotted,green!50!black}}
\usepackage{anysize}
\marginsize{30mm}{20mm}{15mm}{15mm}
\newcolumntype{Y}{>{\raggedleft\arraybackslash}X}
\renewcommand{\baselinestretch}{1.5}
\newcolumntype{P}[1]{>{\raggedright}p{#1}}
\newcolumntype{M}[1]{>{\raggedright}m{#1}}%%declaration de page de garde
\newcommand*\rfrac[2]{{}^{#1}\!/_{#2}}
\setcounter{secnumdepth}{3}
\setcounter{tocdepth}{3}
\newcommand{\cmark}{\ding{51}}%
\newcommand{\xmark}{\ding{59}}%

% Document
\begin{document}
    \sloppy
    \begin{titlepage}
        \renewcommand{\baselinestretch}{1}
        \begin{center}
            \begin{RLtext}
                AljmhwryT AljzA'iryT AldymqrA.tyT Al^s`byT
            \end{RLtext}
            {PEOPLE'S DEMOCRATIC REPUBLIC OF ALGERIA}
            \begin{RLtext}
            {wzArT Alt`lym Al`Aly w Alb.h_t Al`lmy}
            \end{RLtext}
            {DEPARTMENT OF HIGHER EDUCATION AND SCIENTIFIC RESEARCH}
            \begin{RLtext}
                jAm`T mus.taf_A s.tmbOly bim`skar
            \end{RLtext}
            {UNIVERSITY OF MUSTAPHA STAMBOULI MASCARA}
            \begin{figure}[h]
            	\centering
            		\includegraphics[width=4cm]{figurs/logouniv.jpeg}
            \end{figure}
            \begin{RLtext}
                klyT Al`lwm AldaqyqaT
            \end{RLtext}
            {FACULTY OF EXACT SCIENCES}\\
            {COMPUTER SCIENCE DEPARTMENT}\\
            \textsc{\textbf{ \large Master memory} }
        \end{center}
        {\bfseries field :} Mathematics and Computer Science\\
        {\bfseries Faculty  {\hspace*{0.34cm}} :} Computer Science\\
        {\bfseries Option {\hspace*{0.26cm}} :} Networks and distributed system
        \vspace{0.5cm}
        \begin{center}
            \textsc{\textbf{ \large Realized by :}}\\
            {\bfseries Elkaim Moulay Abdellah }{\hspace*{5cm}}{\bfseries Mazouz  Nail}   \\
            \vspace{1cm}
            \Large {\bfseries Theme }
        \end{center}
        \begin{center}
            \hrule width 460pt
            \bigskip
            \Large  \centering \textbf{ \textsc{ ................. } }
            \bigskip
            \hrule width 460pt
        \end{center}
        \raggedright
        \bigskip
        \vspace{1cm}
        \textit{\bfseries Proposed by : Dr. Madber Hayat }
        \vspace{2cm}
        \begin{center}
            \textit{\bfseries College year 2019/2020}
        \end{center}
    \end{titlepage}

    %%Dedicaces%%%%%%%%%%%%%%%%%%%%%%%%%%%%%%%%%%%%%%%%%%%%%%%%%%%%%%%%%%%%%%%%%%%%%%%%%%%%%%%%%%%%%%%%%%%%%%%%%%%%%%%%%%%%%%%%

    \thispagestyle{empty}
    \begin{figure}[h]
    	\centering
    		\includegraphics[width=14cm]{figurs/B.PNG}
    \end{figure}
    \newpage
    \thispagestyle{empty}
    \clearpage
    \newpage
    \pagenumbering{roman}
    \begin{center}
        \textbf{\\}
        \textbf{\\}
        \textbf{\huge \textsc{\itshape Dedications}}\\
        \textbf{\\}
        \textbf{\\}
        \begin{flushleft}
            \textsf{\qquad I dedicate this work to ............. }\\
        \end{flushleft}
    \end{center}
    \begin{flushright}
        \textbf{\textsc{\itshape Elkaim Moulay Abdellah}}
    \end{flushright}
    \newpage
    \begin{center}
        \textbf{\\}
        \textbf{\\}
        \textbf{\huge \textsc{\itshape Dedications}}\\
        \textbf{\\}
        \textbf{\\}

        \begin{flushleft}
            \textsf{\qquad I dedicate this work to ...................... }
        \end{flushleft}

        \normalsize{\itshape .....}
        \textbf{\\}
        \textbf{\\}
    \end{center}
    \begin{flushright}
        \textbf{\textsc{\itshape Mazouz Nail}}
    \end{flushright}

    %%Remerciement%%%%%%%%%%%%%%%%%%%%%%%%%%%%%%%%%%%%%%%%%%%%%%%%%%%%%%%%%%%%%%%%%%%%%%%%%%%%%%%%%%%%%%%%%%%%%%%%%%%%%%%%%%%%%%%%
    \newpage
    \begin{center}
        %\thispagestyle{myheadings}
        %\markboth{droite}{ }
        \textbf{\huge \textsc{\itshape thanks}}
    \end{center}
    \textsf{\qquad We thank .......... }

    %%Abstract %%%%%%%%%%%%%%%%%%%%%%%%%%%%%%%%%%%%%%%%%%%%%%%%%%%%%%%%%%%%%%%%%%%%%%%%%%%%%%%%%%%%%%%%%%%%%%%%%%%%%%%%%%%%%%%%
    \newpage
    \markboth{droite}{ }
    \begin{center}
    \textbf{\huge \textsc{\itshape \textit Abstract}}\\
    \end{center}
    \qquad ...........................

    %%resume %%%%%%%%%%%%%%%%%%%%%%%%%%%%%%%%%%%%%%%%%%%%%%%%%%%%%%%%%%%%%%%%%%%%%%%%%%%%%%%%%%%%%%%%%%%%%%%%%%%%%%%%%%%%%%%%
    \newpage
    \newcommand{\enteteresume}{\markboth{Resume}{Resume}} %il faut ajouter les commandes
    \begin{center}
    \textbf{\huge \textsc{\itshape \textit Résumé}}\\
    \end{center}
    \qquad ...............................

    % La table des matieres%%%%%%%%%%%%%%%%%%%%%%%%%%%%%%%%%%%%%%%%%%%%%%%%%%%%%%%%%%%%%%%%%%%%%%%%%%%%%%%%%%%%%%%%%%%%%%%%%%%%%%%%%%%%%%%%%%%%%%%%%%%%%%%%%%%%%
    \tableofcontents
    \listoffigures
    \listoftables

    %% introduction generale%%%%%%%%%%%%%%%%%%%%%%%%%%%%%%%%%%%%%%%%%%%%%%%%%%%%%%%%%%%%%%%%%%%%%%%%%%%%%%%%%%%%%%%%%%%%%%%%%%%%%%%%%%%%%%%%%%%%%%%%%%%%%%%%%%%%%%%%%%%
    \chapter*{General Introduction}
    \addcontentsline{toc}{chapter}{Introduction générale}
    \setcounter{page}{1}
    \lhead{}
    \cfoot{\bfseries \thepage}
    \rhead{Introduction générale}
    \pagenumbering{arabic}

    ..................................................

    %% chapitre 1%%%%%%%%%%%%%%%%%%%%%%%%%%%%%%%%%%%%%%%%%%%%%%%%%%%%%%%%%%%%%%%%%%%%%%%%%%%%%%%%%%%%%%%%%%%%%%%%%%%%%%%%%%%%%%%%%%%%%%%%%%%%%%%%%%%%%%%%%%%%%%%%%%%%%
    %! Author = moulay
%! Date = 10/21/19

% Preamble
\documentclass[english,a4,12pt]{report}

% Packages
\usepackage{amsmath}
\usepackage[utf8]{inputenc}
\usepackage{xcolor}
\usepackage[T1]{fontenc}
\usepackage{arabtex}
\usepackage{sectsty}
\usepackage[cyr]{aeguill}
\usepackage{rotating}
\usepackage{multirow}
\usepackage{tabulary}
\usepackage{tabularht}
\usepackage{acronym}
\usepackage{fancyhdr}
\usepackage{lscape}
\usepackage{amssymb}
\usepackage{pifont}
\usepackage[most]{tcolorbox}
\usepackage{slashbox}
\usepackage{multido}
\usepackage{caption}
\usepackage{graphicx,wrapfig,lipsum}
\usepackage{enumitem}
\usepackage {fancybox}
\usepackage{array,tabularx}
\usepackage{colortbl}
\usepackage[noend]{algorithmic}
\usepackage[linesnumbered,ruled,vlined,boxed,commentsnumbered]{algorithm2e}
\DeclareGraphicsExtensions{.jpg,.pdf,.PNG,.gif}
\usepackage[pdftex,colorlinks=true,linkcolor=black,citecolor=black,urlcolor=black]{hyperref}
\usepackage{pgfplots}
\usepackage{pgfplotstable}
\usepackage{subcaption}
\usepackage[tikz]{bclogo}
\pgfplotsset{grid style={dashed,gray}}
\pgfplotsset{minor grid style={dotted,green!50!black}}
\pgfplotsset{major grid style={dotted,green!50!black}}
\usepackage{anysize}
\marginsize{30mm}{20mm}{15mm}{15mm}
\newcolumntype{Y}{>{\raggedleft\arraybackslash}X}
\renewcommand{\baselinestretch}{1.5}
\newcolumntype{P}[1]{>{\raggedright}p{#1}}
\newcolumntype{M}[1]{>{\raggedright}m{#1}}%%declaration de page de garde
\newcommand*\rfrac[2]{{}^{#1}\!/_{#2}}
\setcounter{secnumdepth}{3}
\setcounter{tocdepth}{3}
\newcommand{\cmark}{\ding{51}}%
\newcommand{\xmark}{\ding{59}}%

% Document
\begin{document}
    \sloppy
    \begin{titlepage}
        \renewcommand{\baselinestretch}{1}
        \begin{center}
            \begin{RLtext}
                AljmhwryT AljzA'iryT AldymqrA.tyT Al^s`byT
            \end{RLtext}
            {PEOPLE'S DEMOCRATIC REPUBLIC OF ALGERIA}
            \begin{RLtext}
            {wzArT Alt`lym Al`Aly w Alb.h_t Al`lmy}
            \end{RLtext}
            {DEPARTMENT OF HIGHER EDUCATION AND SCIENTIFIC RESEARCH}
            \begin{RLtext}
                jAm`T mus.taf_A s.tmbOly bim`skar
            \end{RLtext}
            {UNIVERSITY OF MUSTAPHA STAMBOULI MASCARA}
            \begin{figure}[h]
            	\centering
            		\includegraphics[width=4cm]{figurs/logouniv.jpeg}
            \end{figure}
            \begin{RLtext}
                klyT Al`lwm AldaqyqaT
            \end{RLtext}
            {FACULTY OF EXACT SCIENCES}\\
            {COMPUTER SCIENCE DEPARTMENT}\\
            \textsc{\textbf{ \large Master memory} }
        \end{center}
        {\bfseries field :} Mathematics and Computer Science\\
        {\bfseries Faculty  {\hspace*{0.34cm}} :} Computer Science\\
        {\bfseries Option {\hspace*{0.26cm}} :} Networks and distributed system
        \vspace{0.5cm}
        \begin{center}
            \textsc{\textbf{ \large Realized by :}}\\
            {\bfseries Elkaim Moulay Abdellah }{\hspace*{5cm}}{\bfseries Mazouz  Nail}   \\
            \vspace{1cm}
            \Large {\bfseries Theme }
        \end{center}
        \begin{center}
            \hrule width 460pt
            \bigskip
            \Large  \centering \textbf{ \textsc{ ................. } }
            \bigskip
            \hrule width 460pt
        \end{center}
        \raggedright
        \bigskip
        \vspace{1cm}
        \textit{\bfseries Proposed by : Dr. Madber Hayat }
        \vspace{2cm}
        \begin{center}
            \textit{\bfseries College year 2019/2020}
        \end{center}
    \end{titlepage}

    %%Dedicaces%%%%%%%%%%%%%%%%%%%%%%%%%%%%%%%%%%%%%%%%%%%%%%%%%%%%%%%%%%%%%%%%%%%%%%%%%%%%%%%%%%%%%%%%%%%%%%%%%%%%%%%%%%%%%%%%

    \thispagestyle{empty}
    \begin{figure}[h]
    	\centering
    		\includegraphics[width=14cm]{figurs/B.PNG}
    \end{figure}
    \newpage
    \thispagestyle{empty}
    \clearpage
    \newpage
    \pagenumbering{roman}
    \begin{center}
        \textbf{\\}
        \textbf{\\}
        \textbf{\huge \textsc{\itshape Dedications}}\\
        \textbf{\\}
        \textbf{\\}
        \begin{flushleft}
            \textsf{\qquad I dedicate this work to ............. }\\
        \end{flushleft}
    \end{center}
    \begin{flushright}
        \textbf{\textsc{\itshape Elkaim Moulay Abdellah}}
    \end{flushright}
    \newpage
    \begin{center}
        \textbf{\\}
        \textbf{\\}
        \textbf{\huge \textsc{\itshape Dedications}}\\
        \textbf{\\}
        \textbf{\\}

        \begin{flushleft}
            \textsf{\qquad I dedicate this work to ...................... }
        \end{flushleft}

        \normalsize{\itshape .....}
        \textbf{\\}
        \textbf{\\}
    \end{center}
    \begin{flushright}
        \textbf{\textsc{\itshape Mazouz Nail}}
    \end{flushright}

    %%Remerciement%%%%%%%%%%%%%%%%%%%%%%%%%%%%%%%%%%%%%%%%%%%%%%%%%%%%%%%%%%%%%%%%%%%%%%%%%%%%%%%%%%%%%%%%%%%%%%%%%%%%%%%%%%%%%%%%
    \newpage
    \begin{center}
        %\thispagestyle{myheadings}
        %\markboth{droite}{ }
        \textbf{\huge \textsc{\itshape thanks}}
    \end{center}
    \textsf{\qquad We thank .......... }

    %%Abstract %%%%%%%%%%%%%%%%%%%%%%%%%%%%%%%%%%%%%%%%%%%%%%%%%%%%%%%%%%%%%%%%%%%%%%%%%%%%%%%%%%%%%%%%%%%%%%%%%%%%%%%%%%%%%%%%
    \newpage
    \markboth{droite}{ }
    \begin{center}
    \textbf{\huge \textsc{\itshape \textit Abstract}}\\
    \end{center}
    \qquad ...........................

    %%resume %%%%%%%%%%%%%%%%%%%%%%%%%%%%%%%%%%%%%%%%%%%%%%%%%%%%%%%%%%%%%%%%%%%%%%%%%%%%%%%%%%%%%%%%%%%%%%%%%%%%%%%%%%%%%%%%
    \newpage
    \newcommand{\enteteresume}{\markboth{Resume}{Resume}} %il faut ajouter les commandes
    \begin{center}
    \textbf{\huge \textsc{\itshape \textit Résumé}}\\
    \end{center}
    \qquad ...............................

    % La table des matieres%%%%%%%%%%%%%%%%%%%%%%%%%%%%%%%%%%%%%%%%%%%%%%%%%%%%%%%%%%%%%%%%%%%%%%%%%%%%%%%%%%%%%%%%%%%%%%%%%%%%%%%%%%%%%%%%%%%%%%%%%%%%%%%%%%%%%
    \tableofcontents
    \listoffigures
    \listoftables

    %% introduction generale%%%%%%%%%%%%%%%%%%%%%%%%%%%%%%%%%%%%%%%%%%%%%%%%%%%%%%%%%%%%%%%%%%%%%%%%%%%%%%%%%%%%%%%%%%%%%%%%%%%%%%%%%%%%%%%%%%%%%%%%%%%%%%%%%%%%%%%%%%%
    \chapter*{General Introduction}
    \addcontentsline{toc}{chapter}{Introduction générale}
    \setcounter{page}{1}
    \lhead{}
    \cfoot{\bfseries \thepage}
    \rhead{Introduction générale}
    \pagenumbering{arabic}

    ..................................................

    %% chapitre 1%%%%%%%%%%%%%%%%%%%%%%%%%%%%%%%%%%%%%%%%%%%%%%%%%%%%%%%%%%%%%%%%%%%%%%%%%%%%%%%%%%%%%%%%%%%%%%%%%%%%%%%%%%%%%%%%%%%%%%%%%%%%%%%%%%%%%%%%%%%%%%%%%%%%%
    \include{chapiter1/main}

    %% chapitre 2%%%%%%%%%%%%%%%%%%%%%%%%%%%%%%%%%%%%%%%%%%%%%%%%%%%%%%%%%%%%%%%%%%%%%%%%%%%%%%%%%%%%%%%%%%%%%%%%%%%%%%%%%%%%%%%%%%%%%%%%%%%%%%%%%%%%%%%%%%%%%%%%%%%%%
    \include{chapiter2/main}

    %% conclusion generale%%%%%%%%%%%%%%%%%%%%%%%%%%%%%%%%%%%%%%%%%%%%%%%%%%%%%%%%%%%%%%%%%%%%%%%%%%%%%%%%%%%%%%%%%%%%%%%%%%%%%%%%%%%%%%%%%%%%%%%%%%%%%%%%%%%%%%%%%%%
    \chapter*{General conclusion}
    \addcontentsline{toc}{chapter}{General conclusion}
    \lhead{}
    \cfoot{\bfseries \thepage}
    \rhead{General conclusion}
    \markboth{droite}{General conclusion}

    .............................................

    %%References%%%%%%%%%%%%%%%%%%%%%%%%%%%%%%%%%%%%%%%%%%%%%%%%%%%%%%%%%%%%%%%%%%%%%%%%%%%%%%%%%%%%%%%%%%%%%%%%%%%%%%%%%%%%%%%%%%%%%%%%%%%%%%
    \clearpage
    \pagestyle{fancy}
    \addcontentsline{toc}{chapter}{Bibliography}
    \begin{thebibliography}{99}
        \lhead{}
        \cfoot{\bfseries \thepage}
        \rhead{Bibliography}

        \bibitem[1]{1}
        H. Altaama,
        Application Mobile Guide,
        Mémoire de Master en Informatique,
        Université Abou Bakr Belkaid de Tlemcen,
        2016.
        \bibitem[2]{2}
        \url{http://generationmobiles.net/2014/11/les-differents-types-dapps-mobiles/}, [consulté le 14/03/2018].

    \end{thebibliography}

\end{document}

    %% chapitre 2%%%%%%%%%%%%%%%%%%%%%%%%%%%%%%%%%%%%%%%%%%%%%%%%%%%%%%%%%%%%%%%%%%%%%%%%%%%%%%%%%%%%%%%%%%%%%%%%%%%%%%%%%%%%%%%%%%%%%%%%%%%%%%%%%%%%%%%%%%%%%%%%%%%%%
    %! Author = moulay
%! Date = 10/21/19

% Preamble
\documentclass[english,a4,12pt]{report}

% Packages
\usepackage{amsmath}
\usepackage[utf8]{inputenc}
\usepackage{xcolor}
\usepackage[T1]{fontenc}
\usepackage{arabtex}
\usepackage{sectsty}
\usepackage[cyr]{aeguill}
\usepackage{rotating}
\usepackage{multirow}
\usepackage{tabulary}
\usepackage{tabularht}
\usepackage{acronym}
\usepackage{fancyhdr}
\usepackage{lscape}
\usepackage{amssymb}
\usepackage{pifont}
\usepackage[most]{tcolorbox}
\usepackage{slashbox}
\usepackage{multido}
\usepackage{caption}
\usepackage{graphicx,wrapfig,lipsum}
\usepackage{enumitem}
\usepackage {fancybox}
\usepackage{array,tabularx}
\usepackage{colortbl}
\usepackage[noend]{algorithmic}
\usepackage[linesnumbered,ruled,vlined,boxed,commentsnumbered]{algorithm2e}
\DeclareGraphicsExtensions{.jpg,.pdf,.PNG,.gif}
\usepackage[pdftex,colorlinks=true,linkcolor=black,citecolor=black,urlcolor=black]{hyperref}
\usepackage{pgfplots}
\usepackage{pgfplotstable}
\usepackage{subcaption}
\usepackage[tikz]{bclogo}
\pgfplotsset{grid style={dashed,gray}}
\pgfplotsset{minor grid style={dotted,green!50!black}}
\pgfplotsset{major grid style={dotted,green!50!black}}
\usepackage{anysize}
\marginsize{30mm}{20mm}{15mm}{15mm}
\newcolumntype{Y}{>{\raggedleft\arraybackslash}X}
\renewcommand{\baselinestretch}{1.5}
\newcolumntype{P}[1]{>{\raggedright}p{#1}}
\newcolumntype{M}[1]{>{\raggedright}m{#1}}%%declaration de page de garde
\newcommand*\rfrac[2]{{}^{#1}\!/_{#2}}
\setcounter{secnumdepth}{3}
\setcounter{tocdepth}{3}
\newcommand{\cmark}{\ding{51}}%
\newcommand{\xmark}{\ding{59}}%

% Document
\begin{document}
    \sloppy
    \begin{titlepage}
        \renewcommand{\baselinestretch}{1}
        \begin{center}
            \begin{RLtext}
                AljmhwryT AljzA'iryT AldymqrA.tyT Al^s`byT
            \end{RLtext}
            {PEOPLE'S DEMOCRATIC REPUBLIC OF ALGERIA}
            \begin{RLtext}
            {wzArT Alt`lym Al`Aly w Alb.h_t Al`lmy}
            \end{RLtext}
            {DEPARTMENT OF HIGHER EDUCATION AND SCIENTIFIC RESEARCH}
            \begin{RLtext}
                jAm`T mus.taf_A s.tmbOly bim`skar
            \end{RLtext}
            {UNIVERSITY OF MUSTAPHA STAMBOULI MASCARA}
            \begin{figure}[h]
            	\centering
            		\includegraphics[width=4cm]{figurs/logouniv.jpeg}
            \end{figure}
            \begin{RLtext}
                klyT Al`lwm AldaqyqaT
            \end{RLtext}
            {FACULTY OF EXACT SCIENCES}\\
            {COMPUTER SCIENCE DEPARTMENT}\\
            \textsc{\textbf{ \large Master memory} }
        \end{center}
        {\bfseries field :} Mathematics and Computer Science\\
        {\bfseries Faculty  {\hspace*{0.34cm}} :} Computer Science\\
        {\bfseries Option {\hspace*{0.26cm}} :} Networks and distributed system
        \vspace{0.5cm}
        \begin{center}
            \textsc{\textbf{ \large Realized by :}}\\
            {\bfseries Elkaim Moulay Abdellah }{\hspace*{5cm}}{\bfseries Mazouz  Nail}   \\
            \vspace{1cm}
            \Large {\bfseries Theme }
        \end{center}
        \begin{center}
            \hrule width 460pt
            \bigskip
            \Large  \centering \textbf{ \textsc{ ................. } }
            \bigskip
            \hrule width 460pt
        \end{center}
        \raggedright
        \bigskip
        \vspace{1cm}
        \textit{\bfseries Proposed by : Dr. Madber Hayat }
        \vspace{2cm}
        \begin{center}
            \textit{\bfseries College year 2019/2020}
        \end{center}
    \end{titlepage}

    %%Dedicaces%%%%%%%%%%%%%%%%%%%%%%%%%%%%%%%%%%%%%%%%%%%%%%%%%%%%%%%%%%%%%%%%%%%%%%%%%%%%%%%%%%%%%%%%%%%%%%%%%%%%%%%%%%%%%%%%

    \thispagestyle{empty}
    \begin{figure}[h]
    	\centering
    		\includegraphics[width=14cm]{figurs/B.PNG}
    \end{figure}
    \newpage
    \thispagestyle{empty}
    \clearpage
    \newpage
    \pagenumbering{roman}
    \begin{center}
        \textbf{\\}
        \textbf{\\}
        \textbf{\huge \textsc{\itshape Dedications}}\\
        \textbf{\\}
        \textbf{\\}
        \begin{flushleft}
            \textsf{\qquad I dedicate this work to ............. }\\
        \end{flushleft}
    \end{center}
    \begin{flushright}
        \textbf{\textsc{\itshape Elkaim Moulay Abdellah}}
    \end{flushright}
    \newpage
    \begin{center}
        \textbf{\\}
        \textbf{\\}
        \textbf{\huge \textsc{\itshape Dedications}}\\
        \textbf{\\}
        \textbf{\\}

        \begin{flushleft}
            \textsf{\qquad I dedicate this work to ...................... }
        \end{flushleft}

        \normalsize{\itshape .....}
        \textbf{\\}
        \textbf{\\}
    \end{center}
    \begin{flushright}
        \textbf{\textsc{\itshape Mazouz Nail}}
    \end{flushright}

    %%Remerciement%%%%%%%%%%%%%%%%%%%%%%%%%%%%%%%%%%%%%%%%%%%%%%%%%%%%%%%%%%%%%%%%%%%%%%%%%%%%%%%%%%%%%%%%%%%%%%%%%%%%%%%%%%%%%%%%
    \newpage
    \begin{center}
        %\thispagestyle{myheadings}
        %\markboth{droite}{ }
        \textbf{\huge \textsc{\itshape thanks}}
    \end{center}
    \textsf{\qquad We thank .......... }

    %%Abstract %%%%%%%%%%%%%%%%%%%%%%%%%%%%%%%%%%%%%%%%%%%%%%%%%%%%%%%%%%%%%%%%%%%%%%%%%%%%%%%%%%%%%%%%%%%%%%%%%%%%%%%%%%%%%%%%
    \newpage
    \markboth{droite}{ }
    \begin{center}
    \textbf{\huge \textsc{\itshape \textit Abstract}}\\
    \end{center}
    \qquad ...........................

    %%resume %%%%%%%%%%%%%%%%%%%%%%%%%%%%%%%%%%%%%%%%%%%%%%%%%%%%%%%%%%%%%%%%%%%%%%%%%%%%%%%%%%%%%%%%%%%%%%%%%%%%%%%%%%%%%%%%
    \newpage
    \newcommand{\enteteresume}{\markboth{Resume}{Resume}} %il faut ajouter les commandes
    \begin{center}
    \textbf{\huge \textsc{\itshape \textit Résumé}}\\
    \end{center}
    \qquad ...............................

    % La table des matieres%%%%%%%%%%%%%%%%%%%%%%%%%%%%%%%%%%%%%%%%%%%%%%%%%%%%%%%%%%%%%%%%%%%%%%%%%%%%%%%%%%%%%%%%%%%%%%%%%%%%%%%%%%%%%%%%%%%%%%%%%%%%%%%%%%%%%
    \tableofcontents
    \listoffigures
    \listoftables

    %% introduction generale%%%%%%%%%%%%%%%%%%%%%%%%%%%%%%%%%%%%%%%%%%%%%%%%%%%%%%%%%%%%%%%%%%%%%%%%%%%%%%%%%%%%%%%%%%%%%%%%%%%%%%%%%%%%%%%%%%%%%%%%%%%%%%%%%%%%%%%%%%%
    \chapter*{General Introduction}
    \addcontentsline{toc}{chapter}{Introduction générale}
    \setcounter{page}{1}
    \lhead{}
    \cfoot{\bfseries \thepage}
    \rhead{Introduction générale}
    \pagenumbering{arabic}

    ..................................................

    %% chapitre 1%%%%%%%%%%%%%%%%%%%%%%%%%%%%%%%%%%%%%%%%%%%%%%%%%%%%%%%%%%%%%%%%%%%%%%%%%%%%%%%%%%%%%%%%%%%%%%%%%%%%%%%%%%%%%%%%%%%%%%%%%%%%%%%%%%%%%%%%%%%%%%%%%%%%%
    \include{chapiter1/main}

    %% chapitre 2%%%%%%%%%%%%%%%%%%%%%%%%%%%%%%%%%%%%%%%%%%%%%%%%%%%%%%%%%%%%%%%%%%%%%%%%%%%%%%%%%%%%%%%%%%%%%%%%%%%%%%%%%%%%%%%%%%%%%%%%%%%%%%%%%%%%%%%%%%%%%%%%%%%%%
    \include{chapiter2/main}

    %% conclusion generale%%%%%%%%%%%%%%%%%%%%%%%%%%%%%%%%%%%%%%%%%%%%%%%%%%%%%%%%%%%%%%%%%%%%%%%%%%%%%%%%%%%%%%%%%%%%%%%%%%%%%%%%%%%%%%%%%%%%%%%%%%%%%%%%%%%%%%%%%%%
    \chapter*{General conclusion}
    \addcontentsline{toc}{chapter}{General conclusion}
    \lhead{}
    \cfoot{\bfseries \thepage}
    \rhead{General conclusion}
    \markboth{droite}{General conclusion}

    .............................................

    %%References%%%%%%%%%%%%%%%%%%%%%%%%%%%%%%%%%%%%%%%%%%%%%%%%%%%%%%%%%%%%%%%%%%%%%%%%%%%%%%%%%%%%%%%%%%%%%%%%%%%%%%%%%%%%%%%%%%%%%%%%%%%%%%
    \clearpage
    \pagestyle{fancy}
    \addcontentsline{toc}{chapter}{Bibliography}
    \begin{thebibliography}{99}
        \lhead{}
        \cfoot{\bfseries \thepage}
        \rhead{Bibliography}

        \bibitem[1]{1}
        H. Altaama,
        Application Mobile Guide,
        Mémoire de Master en Informatique,
        Université Abou Bakr Belkaid de Tlemcen,
        2016.
        \bibitem[2]{2}
        \url{http://generationmobiles.net/2014/11/les-differents-types-dapps-mobiles/}, [consulté le 14/03/2018].

    \end{thebibliography}

\end{document}

    %% conclusion generale%%%%%%%%%%%%%%%%%%%%%%%%%%%%%%%%%%%%%%%%%%%%%%%%%%%%%%%%%%%%%%%%%%%%%%%%%%%%%%%%%%%%%%%%%%%%%%%%%%%%%%%%%%%%%%%%%%%%%%%%%%%%%%%%%%%%%%%%%%%
    \chapter*{General conclusion}
    \addcontentsline{toc}{chapter}{General conclusion}
    \lhead{}
    \cfoot{\bfseries \thepage}
    \rhead{General conclusion}
    \markboth{droite}{General conclusion}

    .............................................

    %%References%%%%%%%%%%%%%%%%%%%%%%%%%%%%%%%%%%%%%%%%%%%%%%%%%%%%%%%%%%%%%%%%%%%%%%%%%%%%%%%%%%%%%%%%%%%%%%%%%%%%%%%%%%%%%%%%%%%%%%%%%%%%%%
    \clearpage
    \pagestyle{fancy}
    \addcontentsline{toc}{chapter}{Bibliography}
    \begin{thebibliography}{99}
        \lhead{}
        \cfoot{\bfseries \thepage}
        \rhead{Bibliography}

        \bibitem[1]{1}
        H. Altaama,
        Application Mobile Guide,
        Mémoire de Master en Informatique,
        Université Abou Bakr Belkaid de Tlemcen,
        2016.
        \bibitem[2]{2}
        \url{http://generationmobiles.net/2014/11/les-differents-types-dapps-mobiles/}, [consulté le 14/03/2018].

    \end{thebibliography}

\end{document}

    %% chapitre 2%%%%%%%%%%%%%%%%%%%%%%%%%%%%%%%%%%%%%%%%%%%%%%%%%%%%%%%%%%%%%%%%%%%%%%%%%%%%%%%%%%%%%%%%%%%%%%%%%%%%%%%%%%%%%%%%%%%%%%%%%%%%%%%%%%%%%%%%%%%%%%%%%%%%%
    %! Author = moulay
%! Date = 10/21/19

% Preamble
\documentclass[english,a4,12pt]{report}

% Packages
\usepackage{amsmath}
\usepackage[utf8]{inputenc}
\usepackage{xcolor}
\usepackage[T1]{fontenc}
\usepackage{arabtex}
\usepackage{sectsty}
\usepackage[cyr]{aeguill}
\usepackage{rotating}
\usepackage{multirow}
\usepackage{tabulary}
\usepackage{tabularht}
\usepackage{acronym}
\usepackage{fancyhdr}
\usepackage{lscape}
\usepackage{amssymb}
\usepackage{pifont}
\usepackage[most]{tcolorbox}
\usepackage{slashbox}
\usepackage{multido}
\usepackage{caption}
\usepackage{graphicx,wrapfig,lipsum}
\usepackage{enumitem}
\usepackage {fancybox}
\usepackage{array,tabularx}
\usepackage{colortbl}
\usepackage[noend]{algorithmic}
\usepackage[linesnumbered,ruled,vlined,boxed,commentsnumbered]{algorithm2e}
\DeclareGraphicsExtensions{.jpg,.pdf,.PNG,.gif}
\usepackage[pdftex,colorlinks=true,linkcolor=black,citecolor=black,urlcolor=black]{hyperref}
\usepackage{pgfplots}
\usepackage{pgfplotstable}
\usepackage{subcaption}
\usepackage[tikz]{bclogo}
\pgfplotsset{grid style={dashed,gray}}
\pgfplotsset{minor grid style={dotted,green!50!black}}
\pgfplotsset{major grid style={dotted,green!50!black}}
\usepackage{anysize}
\marginsize{30mm}{20mm}{15mm}{15mm}
\newcolumntype{Y}{>{\raggedleft\arraybackslash}X}
\renewcommand{\baselinestretch}{1.5}
\newcolumntype{P}[1]{>{\raggedright}p{#1}}
\newcolumntype{M}[1]{>{\raggedright}m{#1}}%%declaration de page de garde
\newcommand*\rfrac[2]{{}^{#1}\!/_{#2}}
\setcounter{secnumdepth}{3}
\setcounter{tocdepth}{3}
\newcommand{\cmark}{\ding{51}}%
\newcommand{\xmark}{\ding{59}}%

% Document
\begin{document}
    \sloppy
    \begin{titlepage}
        \renewcommand{\baselinestretch}{1}
        \begin{center}
            \begin{RLtext}
                AljmhwryT AljzA'iryT AldymqrA.tyT Al^s`byT
            \end{RLtext}
            {PEOPLE'S DEMOCRATIC REPUBLIC OF ALGERIA}
            \begin{RLtext}
            {wzArT Alt`lym Al`Aly w Alb.h_t Al`lmy}
            \end{RLtext}
            {DEPARTMENT OF HIGHER EDUCATION AND SCIENTIFIC RESEARCH}
            \begin{RLtext}
                jAm`T mus.taf_A s.tmbOly bim`skar
            \end{RLtext}
            {UNIVERSITY OF MUSTAPHA STAMBOULI MASCARA}
            \begin{figure}[h]
            	\centering
            		\includegraphics[width=4cm]{figurs/logouniv.jpeg}
            \end{figure}
            \begin{RLtext}
                klyT Al`lwm AldaqyqaT
            \end{RLtext}
            {FACULTY OF EXACT SCIENCES}\\
            {COMPUTER SCIENCE DEPARTMENT}\\
            \textsc{\textbf{ \large Master memory} }
        \end{center}
        {\bfseries field :} Mathematics and Computer Science\\
        {\bfseries Faculty  {\hspace*{0.34cm}} :} Computer Science\\
        {\bfseries Option {\hspace*{0.26cm}} :} Networks and distributed system
        \vspace{0.5cm}
        \begin{center}
            \textsc{\textbf{ \large Realized by :}}\\
            {\bfseries Elkaim Moulay Abdellah }{\hspace*{5cm}}{\bfseries Mazouz  Nail}   \\
            \vspace{1cm}
            \Large {\bfseries Theme }
        \end{center}
        \begin{center}
            \hrule width 460pt
            \bigskip
            \Large  \centering \textbf{ \textsc{ ................. } }
            \bigskip
            \hrule width 460pt
        \end{center}
        \raggedright
        \bigskip
        \vspace{1cm}
        \textit{\bfseries Proposed by : Dr. Madber Hayat }
        \vspace{2cm}
        \begin{center}
            \textit{\bfseries College year 2019/2020}
        \end{center}
    \end{titlepage}

    %%Dedicaces%%%%%%%%%%%%%%%%%%%%%%%%%%%%%%%%%%%%%%%%%%%%%%%%%%%%%%%%%%%%%%%%%%%%%%%%%%%%%%%%%%%%%%%%%%%%%%%%%%%%%%%%%%%%%%%%

    \thispagestyle{empty}
    \begin{figure}[h]
    	\centering
    		\includegraphics[width=14cm]{figurs/B.PNG}
    \end{figure}
    \newpage
    \thispagestyle{empty}
    \clearpage
    \newpage
    \pagenumbering{roman}
    \begin{center}
        \textbf{\\}
        \textbf{\\}
        \textbf{\huge \textsc{\itshape Dedications}}\\
        \textbf{\\}
        \textbf{\\}
        \begin{flushleft}
            \textsf{\qquad I dedicate this work to ............. }\\
        \end{flushleft}
    \end{center}
    \begin{flushright}
        \textbf{\textsc{\itshape Elkaim Moulay Abdellah}}
    \end{flushright}
    \newpage
    \begin{center}
        \textbf{\\}
        \textbf{\\}
        \textbf{\huge \textsc{\itshape Dedications}}\\
        \textbf{\\}
        \textbf{\\}

        \begin{flushleft}
            \textsf{\qquad I dedicate this work to ...................... }
        \end{flushleft}

        \normalsize{\itshape .....}
        \textbf{\\}
        \textbf{\\}
    \end{center}
    \begin{flushright}
        \textbf{\textsc{\itshape Mazouz Nail}}
    \end{flushright}

    %%Remerciement%%%%%%%%%%%%%%%%%%%%%%%%%%%%%%%%%%%%%%%%%%%%%%%%%%%%%%%%%%%%%%%%%%%%%%%%%%%%%%%%%%%%%%%%%%%%%%%%%%%%%%%%%%%%%%%%
    \newpage
    \begin{center}
        %\thispagestyle{myheadings}
        %\markboth{droite}{ }
        \textbf{\huge \textsc{\itshape thanks}}
    \end{center}
    \textsf{\qquad We thank .......... }

    %%Abstract %%%%%%%%%%%%%%%%%%%%%%%%%%%%%%%%%%%%%%%%%%%%%%%%%%%%%%%%%%%%%%%%%%%%%%%%%%%%%%%%%%%%%%%%%%%%%%%%%%%%%%%%%%%%%%%%
    \newpage
    \markboth{droite}{ }
    \begin{center}
    \textbf{\huge \textsc{\itshape \textit Abstract}}\\
    \end{center}
    \qquad ...........................

    %%resume %%%%%%%%%%%%%%%%%%%%%%%%%%%%%%%%%%%%%%%%%%%%%%%%%%%%%%%%%%%%%%%%%%%%%%%%%%%%%%%%%%%%%%%%%%%%%%%%%%%%%%%%%%%%%%%%
    \newpage
    \newcommand{\enteteresume}{\markboth{Resume}{Resume}} %il faut ajouter les commandes
    \begin{center}
    \textbf{\huge \textsc{\itshape \textit Résumé}}\\
    \end{center}
    \qquad ...............................

    % La table des matieres%%%%%%%%%%%%%%%%%%%%%%%%%%%%%%%%%%%%%%%%%%%%%%%%%%%%%%%%%%%%%%%%%%%%%%%%%%%%%%%%%%%%%%%%%%%%%%%%%%%%%%%%%%%%%%%%%%%%%%%%%%%%%%%%%%%%%
    \tableofcontents
    \listoffigures
    \listoftables

    %% introduction generale%%%%%%%%%%%%%%%%%%%%%%%%%%%%%%%%%%%%%%%%%%%%%%%%%%%%%%%%%%%%%%%%%%%%%%%%%%%%%%%%%%%%%%%%%%%%%%%%%%%%%%%%%%%%%%%%%%%%%%%%%%%%%%%%%%%%%%%%%%%
    \chapter*{General Introduction}
    \addcontentsline{toc}{chapter}{Introduction générale}
    \setcounter{page}{1}
    \lhead{}
    \cfoot{\bfseries \thepage}
    \rhead{Introduction générale}
    \pagenumbering{arabic}

    ..................................................

    %% chapitre 1%%%%%%%%%%%%%%%%%%%%%%%%%%%%%%%%%%%%%%%%%%%%%%%%%%%%%%%%%%%%%%%%%%%%%%%%%%%%%%%%%%%%%%%%%%%%%%%%%%%%%%%%%%%%%%%%%%%%%%%%%%%%%%%%%%%%%%%%%%%%%%%%%%%%%
    %! Author = moulay
%! Date = 10/21/19

% Preamble
\documentclass[english,a4,12pt]{report}

% Packages
\usepackage{amsmath}
\usepackage[utf8]{inputenc}
\usepackage{xcolor}
\usepackage[T1]{fontenc}
\usepackage{arabtex}
\usepackage{sectsty}
\usepackage[cyr]{aeguill}
\usepackage{rotating}
\usepackage{multirow}
\usepackage{tabulary}
\usepackage{tabularht}
\usepackage{acronym}
\usepackage{fancyhdr}
\usepackage{lscape}
\usepackage{amssymb}
\usepackage{pifont}
\usepackage[most]{tcolorbox}
\usepackage{slashbox}
\usepackage{multido}
\usepackage{caption}
\usepackage{graphicx,wrapfig,lipsum}
\usepackage{enumitem}
\usepackage {fancybox}
\usepackage{array,tabularx}
\usepackage{colortbl}
\usepackage[noend]{algorithmic}
\usepackage[linesnumbered,ruled,vlined,boxed,commentsnumbered]{algorithm2e}
\DeclareGraphicsExtensions{.jpg,.pdf,.PNG,.gif}
\usepackage[pdftex,colorlinks=true,linkcolor=black,citecolor=black,urlcolor=black]{hyperref}
\usepackage{pgfplots}
\usepackage{pgfplotstable}
\usepackage{subcaption}
\usepackage[tikz]{bclogo}
\pgfplotsset{grid style={dashed,gray}}
\pgfplotsset{minor grid style={dotted,green!50!black}}
\pgfplotsset{major grid style={dotted,green!50!black}}
\usepackage{anysize}
\marginsize{30mm}{20mm}{15mm}{15mm}
\newcolumntype{Y}{>{\raggedleft\arraybackslash}X}
\renewcommand{\baselinestretch}{1.5}
\newcolumntype{P}[1]{>{\raggedright}p{#1}}
\newcolumntype{M}[1]{>{\raggedright}m{#1}}%%declaration de page de garde
\newcommand*\rfrac[2]{{}^{#1}\!/_{#2}}
\setcounter{secnumdepth}{3}
\setcounter{tocdepth}{3}
\newcommand{\cmark}{\ding{51}}%
\newcommand{\xmark}{\ding{59}}%

% Document
\begin{document}
    \sloppy
    \begin{titlepage}
        \renewcommand{\baselinestretch}{1}
        \begin{center}
            \begin{RLtext}
                AljmhwryT AljzA'iryT AldymqrA.tyT Al^s`byT
            \end{RLtext}
            {PEOPLE'S DEMOCRATIC REPUBLIC OF ALGERIA}
            \begin{RLtext}
            {wzArT Alt`lym Al`Aly w Alb.h_t Al`lmy}
            \end{RLtext}
            {DEPARTMENT OF HIGHER EDUCATION AND SCIENTIFIC RESEARCH}
            \begin{RLtext}
                jAm`T mus.taf_A s.tmbOly bim`skar
            \end{RLtext}
            {UNIVERSITY OF MUSTAPHA STAMBOULI MASCARA}
            \begin{figure}[h]
            	\centering
            		\includegraphics[width=4cm]{figurs/logouniv.jpeg}
            \end{figure}
            \begin{RLtext}
                klyT Al`lwm AldaqyqaT
            \end{RLtext}
            {FACULTY OF EXACT SCIENCES}\\
            {COMPUTER SCIENCE DEPARTMENT}\\
            \textsc{\textbf{ \large Master memory} }
        \end{center}
        {\bfseries field :} Mathematics and Computer Science\\
        {\bfseries Faculty  {\hspace*{0.34cm}} :} Computer Science\\
        {\bfseries Option {\hspace*{0.26cm}} :} Networks and distributed system
        \vspace{0.5cm}
        \begin{center}
            \textsc{\textbf{ \large Realized by :}}\\
            {\bfseries Elkaim Moulay Abdellah }{\hspace*{5cm}}{\bfseries Mazouz  Nail}   \\
            \vspace{1cm}
            \Large {\bfseries Theme }
        \end{center}
        \begin{center}
            \hrule width 460pt
            \bigskip
            \Large  \centering \textbf{ \textsc{ ................. } }
            \bigskip
            \hrule width 460pt
        \end{center}
        \raggedright
        \bigskip
        \vspace{1cm}
        \textit{\bfseries Proposed by : Dr. Madber Hayat }
        \vspace{2cm}
        \begin{center}
            \textit{\bfseries College year 2019/2020}
        \end{center}
    \end{titlepage}

    %%Dedicaces%%%%%%%%%%%%%%%%%%%%%%%%%%%%%%%%%%%%%%%%%%%%%%%%%%%%%%%%%%%%%%%%%%%%%%%%%%%%%%%%%%%%%%%%%%%%%%%%%%%%%%%%%%%%%%%%

    \thispagestyle{empty}
    \begin{figure}[h]
    	\centering
    		\includegraphics[width=14cm]{figurs/B.PNG}
    \end{figure}
    \newpage
    \thispagestyle{empty}
    \clearpage
    \newpage
    \pagenumbering{roman}
    \begin{center}
        \textbf{\\}
        \textbf{\\}
        \textbf{\huge \textsc{\itshape Dedications}}\\
        \textbf{\\}
        \textbf{\\}
        \begin{flushleft}
            \textsf{\qquad I dedicate this work to ............. }\\
        \end{flushleft}
    \end{center}
    \begin{flushright}
        \textbf{\textsc{\itshape Elkaim Moulay Abdellah}}
    \end{flushright}
    \newpage
    \begin{center}
        \textbf{\\}
        \textbf{\\}
        \textbf{\huge \textsc{\itshape Dedications}}\\
        \textbf{\\}
        \textbf{\\}

        \begin{flushleft}
            \textsf{\qquad I dedicate this work to ...................... }
        \end{flushleft}

        \normalsize{\itshape .....}
        \textbf{\\}
        \textbf{\\}
    \end{center}
    \begin{flushright}
        \textbf{\textsc{\itshape Mazouz Nail}}
    \end{flushright}

    %%Remerciement%%%%%%%%%%%%%%%%%%%%%%%%%%%%%%%%%%%%%%%%%%%%%%%%%%%%%%%%%%%%%%%%%%%%%%%%%%%%%%%%%%%%%%%%%%%%%%%%%%%%%%%%%%%%%%%%
    \newpage
    \begin{center}
        %\thispagestyle{myheadings}
        %\markboth{droite}{ }
        \textbf{\huge \textsc{\itshape thanks}}
    \end{center}
    \textsf{\qquad We thank .......... }

    %%Abstract %%%%%%%%%%%%%%%%%%%%%%%%%%%%%%%%%%%%%%%%%%%%%%%%%%%%%%%%%%%%%%%%%%%%%%%%%%%%%%%%%%%%%%%%%%%%%%%%%%%%%%%%%%%%%%%%
    \newpage
    \markboth{droite}{ }
    \begin{center}
    \textbf{\huge \textsc{\itshape \textit Abstract}}\\
    \end{center}
    \qquad ...........................

    %%resume %%%%%%%%%%%%%%%%%%%%%%%%%%%%%%%%%%%%%%%%%%%%%%%%%%%%%%%%%%%%%%%%%%%%%%%%%%%%%%%%%%%%%%%%%%%%%%%%%%%%%%%%%%%%%%%%
    \newpage
    \newcommand{\enteteresume}{\markboth{Resume}{Resume}} %il faut ajouter les commandes
    \begin{center}
    \textbf{\huge \textsc{\itshape \textit Résumé}}\\
    \end{center}
    \qquad ...............................

    % La table des matieres%%%%%%%%%%%%%%%%%%%%%%%%%%%%%%%%%%%%%%%%%%%%%%%%%%%%%%%%%%%%%%%%%%%%%%%%%%%%%%%%%%%%%%%%%%%%%%%%%%%%%%%%%%%%%%%%%%%%%%%%%%%%%%%%%%%%%
    \tableofcontents
    \listoffigures
    \listoftables

    %% introduction generale%%%%%%%%%%%%%%%%%%%%%%%%%%%%%%%%%%%%%%%%%%%%%%%%%%%%%%%%%%%%%%%%%%%%%%%%%%%%%%%%%%%%%%%%%%%%%%%%%%%%%%%%%%%%%%%%%%%%%%%%%%%%%%%%%%%%%%%%%%%
    \chapter*{General Introduction}
    \addcontentsline{toc}{chapter}{Introduction générale}
    \setcounter{page}{1}
    \lhead{}
    \cfoot{\bfseries \thepage}
    \rhead{Introduction générale}
    \pagenumbering{arabic}

    ..................................................

    %% chapitre 1%%%%%%%%%%%%%%%%%%%%%%%%%%%%%%%%%%%%%%%%%%%%%%%%%%%%%%%%%%%%%%%%%%%%%%%%%%%%%%%%%%%%%%%%%%%%%%%%%%%%%%%%%%%%%%%%%%%%%%%%%%%%%%%%%%%%%%%%%%%%%%%%%%%%%
    \include{chapiter1/main}

    %% chapitre 2%%%%%%%%%%%%%%%%%%%%%%%%%%%%%%%%%%%%%%%%%%%%%%%%%%%%%%%%%%%%%%%%%%%%%%%%%%%%%%%%%%%%%%%%%%%%%%%%%%%%%%%%%%%%%%%%%%%%%%%%%%%%%%%%%%%%%%%%%%%%%%%%%%%%%
    \include{chapiter2/main}

    %% conclusion generale%%%%%%%%%%%%%%%%%%%%%%%%%%%%%%%%%%%%%%%%%%%%%%%%%%%%%%%%%%%%%%%%%%%%%%%%%%%%%%%%%%%%%%%%%%%%%%%%%%%%%%%%%%%%%%%%%%%%%%%%%%%%%%%%%%%%%%%%%%%
    \chapter*{General conclusion}
    \addcontentsline{toc}{chapter}{General conclusion}
    \lhead{}
    \cfoot{\bfseries \thepage}
    \rhead{General conclusion}
    \markboth{droite}{General conclusion}

    .............................................

    %%References%%%%%%%%%%%%%%%%%%%%%%%%%%%%%%%%%%%%%%%%%%%%%%%%%%%%%%%%%%%%%%%%%%%%%%%%%%%%%%%%%%%%%%%%%%%%%%%%%%%%%%%%%%%%%%%%%%%%%%%%%%%%%%
    \clearpage
    \pagestyle{fancy}
    \addcontentsline{toc}{chapter}{Bibliography}
    \begin{thebibliography}{99}
        \lhead{}
        \cfoot{\bfseries \thepage}
        \rhead{Bibliography}

        \bibitem[1]{1}
        H. Altaama,
        Application Mobile Guide,
        Mémoire de Master en Informatique,
        Université Abou Bakr Belkaid de Tlemcen,
        2016.
        \bibitem[2]{2}
        \url{http://generationmobiles.net/2014/11/les-differents-types-dapps-mobiles/}, [consulté le 14/03/2018].

    \end{thebibliography}

\end{document}

    %% chapitre 2%%%%%%%%%%%%%%%%%%%%%%%%%%%%%%%%%%%%%%%%%%%%%%%%%%%%%%%%%%%%%%%%%%%%%%%%%%%%%%%%%%%%%%%%%%%%%%%%%%%%%%%%%%%%%%%%%%%%%%%%%%%%%%%%%%%%%%%%%%%%%%%%%%%%%
    %! Author = moulay
%! Date = 10/21/19

% Preamble
\documentclass[english,a4,12pt]{report}

% Packages
\usepackage{amsmath}
\usepackage[utf8]{inputenc}
\usepackage{xcolor}
\usepackage[T1]{fontenc}
\usepackage{arabtex}
\usepackage{sectsty}
\usepackage[cyr]{aeguill}
\usepackage{rotating}
\usepackage{multirow}
\usepackage{tabulary}
\usepackage{tabularht}
\usepackage{acronym}
\usepackage{fancyhdr}
\usepackage{lscape}
\usepackage{amssymb}
\usepackage{pifont}
\usepackage[most]{tcolorbox}
\usepackage{slashbox}
\usepackage{multido}
\usepackage{caption}
\usepackage{graphicx,wrapfig,lipsum}
\usepackage{enumitem}
\usepackage {fancybox}
\usepackage{array,tabularx}
\usepackage{colortbl}
\usepackage[noend]{algorithmic}
\usepackage[linesnumbered,ruled,vlined,boxed,commentsnumbered]{algorithm2e}
\DeclareGraphicsExtensions{.jpg,.pdf,.PNG,.gif}
\usepackage[pdftex,colorlinks=true,linkcolor=black,citecolor=black,urlcolor=black]{hyperref}
\usepackage{pgfplots}
\usepackage{pgfplotstable}
\usepackage{subcaption}
\usepackage[tikz]{bclogo}
\pgfplotsset{grid style={dashed,gray}}
\pgfplotsset{minor grid style={dotted,green!50!black}}
\pgfplotsset{major grid style={dotted,green!50!black}}
\usepackage{anysize}
\marginsize{30mm}{20mm}{15mm}{15mm}
\newcolumntype{Y}{>{\raggedleft\arraybackslash}X}
\renewcommand{\baselinestretch}{1.5}
\newcolumntype{P}[1]{>{\raggedright}p{#1}}
\newcolumntype{M}[1]{>{\raggedright}m{#1}}%%declaration de page de garde
\newcommand*\rfrac[2]{{}^{#1}\!/_{#2}}
\setcounter{secnumdepth}{3}
\setcounter{tocdepth}{3}
\newcommand{\cmark}{\ding{51}}%
\newcommand{\xmark}{\ding{59}}%

% Document
\begin{document}
    \sloppy
    \begin{titlepage}
        \renewcommand{\baselinestretch}{1}
        \begin{center}
            \begin{RLtext}
                AljmhwryT AljzA'iryT AldymqrA.tyT Al^s`byT
            \end{RLtext}
            {PEOPLE'S DEMOCRATIC REPUBLIC OF ALGERIA}
            \begin{RLtext}
            {wzArT Alt`lym Al`Aly w Alb.h_t Al`lmy}
            \end{RLtext}
            {DEPARTMENT OF HIGHER EDUCATION AND SCIENTIFIC RESEARCH}
            \begin{RLtext}
                jAm`T mus.taf_A s.tmbOly bim`skar
            \end{RLtext}
            {UNIVERSITY OF MUSTAPHA STAMBOULI MASCARA}
            \begin{figure}[h]
            	\centering
            		\includegraphics[width=4cm]{figurs/logouniv.jpeg}
            \end{figure}
            \begin{RLtext}
                klyT Al`lwm AldaqyqaT
            \end{RLtext}
            {FACULTY OF EXACT SCIENCES}\\
            {COMPUTER SCIENCE DEPARTMENT}\\
            \textsc{\textbf{ \large Master memory} }
        \end{center}
        {\bfseries field :} Mathematics and Computer Science\\
        {\bfseries Faculty  {\hspace*{0.34cm}} :} Computer Science\\
        {\bfseries Option {\hspace*{0.26cm}} :} Networks and distributed system
        \vspace{0.5cm}
        \begin{center}
            \textsc{\textbf{ \large Realized by :}}\\
            {\bfseries Elkaim Moulay Abdellah }{\hspace*{5cm}}{\bfseries Mazouz  Nail}   \\
            \vspace{1cm}
            \Large {\bfseries Theme }
        \end{center}
        \begin{center}
            \hrule width 460pt
            \bigskip
            \Large  \centering \textbf{ \textsc{ ................. } }
            \bigskip
            \hrule width 460pt
        \end{center}
        \raggedright
        \bigskip
        \vspace{1cm}
        \textit{\bfseries Proposed by : Dr. Madber Hayat }
        \vspace{2cm}
        \begin{center}
            \textit{\bfseries College year 2019/2020}
        \end{center}
    \end{titlepage}

    %%Dedicaces%%%%%%%%%%%%%%%%%%%%%%%%%%%%%%%%%%%%%%%%%%%%%%%%%%%%%%%%%%%%%%%%%%%%%%%%%%%%%%%%%%%%%%%%%%%%%%%%%%%%%%%%%%%%%%%%

    \thispagestyle{empty}
    \begin{figure}[h]
    	\centering
    		\includegraphics[width=14cm]{figurs/B.PNG}
    \end{figure}
    \newpage
    \thispagestyle{empty}
    \clearpage
    \newpage
    \pagenumbering{roman}
    \begin{center}
        \textbf{\\}
        \textbf{\\}
        \textbf{\huge \textsc{\itshape Dedications}}\\
        \textbf{\\}
        \textbf{\\}
        \begin{flushleft}
            \textsf{\qquad I dedicate this work to ............. }\\
        \end{flushleft}
    \end{center}
    \begin{flushright}
        \textbf{\textsc{\itshape Elkaim Moulay Abdellah}}
    \end{flushright}
    \newpage
    \begin{center}
        \textbf{\\}
        \textbf{\\}
        \textbf{\huge \textsc{\itshape Dedications}}\\
        \textbf{\\}
        \textbf{\\}

        \begin{flushleft}
            \textsf{\qquad I dedicate this work to ...................... }
        \end{flushleft}

        \normalsize{\itshape .....}
        \textbf{\\}
        \textbf{\\}
    \end{center}
    \begin{flushright}
        \textbf{\textsc{\itshape Mazouz Nail}}
    \end{flushright}

    %%Remerciement%%%%%%%%%%%%%%%%%%%%%%%%%%%%%%%%%%%%%%%%%%%%%%%%%%%%%%%%%%%%%%%%%%%%%%%%%%%%%%%%%%%%%%%%%%%%%%%%%%%%%%%%%%%%%%%%
    \newpage
    \begin{center}
        %\thispagestyle{myheadings}
        %\markboth{droite}{ }
        \textbf{\huge \textsc{\itshape thanks}}
    \end{center}
    \textsf{\qquad We thank .......... }

    %%Abstract %%%%%%%%%%%%%%%%%%%%%%%%%%%%%%%%%%%%%%%%%%%%%%%%%%%%%%%%%%%%%%%%%%%%%%%%%%%%%%%%%%%%%%%%%%%%%%%%%%%%%%%%%%%%%%%%
    \newpage
    \markboth{droite}{ }
    \begin{center}
    \textbf{\huge \textsc{\itshape \textit Abstract}}\\
    \end{center}
    \qquad ...........................

    %%resume %%%%%%%%%%%%%%%%%%%%%%%%%%%%%%%%%%%%%%%%%%%%%%%%%%%%%%%%%%%%%%%%%%%%%%%%%%%%%%%%%%%%%%%%%%%%%%%%%%%%%%%%%%%%%%%%
    \newpage
    \newcommand{\enteteresume}{\markboth{Resume}{Resume}} %il faut ajouter les commandes
    \begin{center}
    \textbf{\huge \textsc{\itshape \textit Résumé}}\\
    \end{center}
    \qquad ...............................

    % La table des matieres%%%%%%%%%%%%%%%%%%%%%%%%%%%%%%%%%%%%%%%%%%%%%%%%%%%%%%%%%%%%%%%%%%%%%%%%%%%%%%%%%%%%%%%%%%%%%%%%%%%%%%%%%%%%%%%%%%%%%%%%%%%%%%%%%%%%%
    \tableofcontents
    \listoffigures
    \listoftables

    %% introduction generale%%%%%%%%%%%%%%%%%%%%%%%%%%%%%%%%%%%%%%%%%%%%%%%%%%%%%%%%%%%%%%%%%%%%%%%%%%%%%%%%%%%%%%%%%%%%%%%%%%%%%%%%%%%%%%%%%%%%%%%%%%%%%%%%%%%%%%%%%%%
    \chapter*{General Introduction}
    \addcontentsline{toc}{chapter}{Introduction générale}
    \setcounter{page}{1}
    \lhead{}
    \cfoot{\bfseries \thepage}
    \rhead{Introduction générale}
    \pagenumbering{arabic}

    ..................................................

    %% chapitre 1%%%%%%%%%%%%%%%%%%%%%%%%%%%%%%%%%%%%%%%%%%%%%%%%%%%%%%%%%%%%%%%%%%%%%%%%%%%%%%%%%%%%%%%%%%%%%%%%%%%%%%%%%%%%%%%%%%%%%%%%%%%%%%%%%%%%%%%%%%%%%%%%%%%%%
    \include{chapiter1/main}

    %% chapitre 2%%%%%%%%%%%%%%%%%%%%%%%%%%%%%%%%%%%%%%%%%%%%%%%%%%%%%%%%%%%%%%%%%%%%%%%%%%%%%%%%%%%%%%%%%%%%%%%%%%%%%%%%%%%%%%%%%%%%%%%%%%%%%%%%%%%%%%%%%%%%%%%%%%%%%
    \include{chapiter2/main}

    %% conclusion generale%%%%%%%%%%%%%%%%%%%%%%%%%%%%%%%%%%%%%%%%%%%%%%%%%%%%%%%%%%%%%%%%%%%%%%%%%%%%%%%%%%%%%%%%%%%%%%%%%%%%%%%%%%%%%%%%%%%%%%%%%%%%%%%%%%%%%%%%%%%
    \chapter*{General conclusion}
    \addcontentsline{toc}{chapter}{General conclusion}
    \lhead{}
    \cfoot{\bfseries \thepage}
    \rhead{General conclusion}
    \markboth{droite}{General conclusion}

    .............................................

    %%References%%%%%%%%%%%%%%%%%%%%%%%%%%%%%%%%%%%%%%%%%%%%%%%%%%%%%%%%%%%%%%%%%%%%%%%%%%%%%%%%%%%%%%%%%%%%%%%%%%%%%%%%%%%%%%%%%%%%%%%%%%%%%%
    \clearpage
    \pagestyle{fancy}
    \addcontentsline{toc}{chapter}{Bibliography}
    \begin{thebibliography}{99}
        \lhead{}
        \cfoot{\bfseries \thepage}
        \rhead{Bibliography}

        \bibitem[1]{1}
        H. Altaama,
        Application Mobile Guide,
        Mémoire de Master en Informatique,
        Université Abou Bakr Belkaid de Tlemcen,
        2016.
        \bibitem[2]{2}
        \url{http://generationmobiles.net/2014/11/les-differents-types-dapps-mobiles/}, [consulté le 14/03/2018].

    \end{thebibliography}

\end{document}

    %% conclusion generale%%%%%%%%%%%%%%%%%%%%%%%%%%%%%%%%%%%%%%%%%%%%%%%%%%%%%%%%%%%%%%%%%%%%%%%%%%%%%%%%%%%%%%%%%%%%%%%%%%%%%%%%%%%%%%%%%%%%%%%%%%%%%%%%%%%%%%%%%%%
    \chapter*{General conclusion}
    \addcontentsline{toc}{chapter}{General conclusion}
    \lhead{}
    \cfoot{\bfseries \thepage}
    \rhead{General conclusion}
    \markboth{droite}{General conclusion}

    .............................................

    %%References%%%%%%%%%%%%%%%%%%%%%%%%%%%%%%%%%%%%%%%%%%%%%%%%%%%%%%%%%%%%%%%%%%%%%%%%%%%%%%%%%%%%%%%%%%%%%%%%%%%%%%%%%%%%%%%%%%%%%%%%%%%%%%
    \clearpage
    \pagestyle{fancy}
    \addcontentsline{toc}{chapter}{Bibliography}
    \begin{thebibliography}{99}
        \lhead{}
        \cfoot{\bfseries \thepage}
        \rhead{Bibliography}

        \bibitem[1]{1}
        H. Altaama,
        Application Mobile Guide,
        Mémoire de Master en Informatique,
        Université Abou Bakr Belkaid de Tlemcen,
        2016.
        \bibitem[2]{2}
        \url{http://generationmobiles.net/2014/11/les-differents-types-dapps-mobiles/}, [consulté le 14/03/2018].

    \end{thebibliography}

\end{document}

    %% conclusion generale%%%%%%%%%%%%%%%%%%%%%%%%%%%%%%%%%%%%%%%%%%%%%%%%%%%%%%%%%%%%%%%%%%%%%%%%%%%%%%%%%%%%%%%%%%%%%%%%%%%%%%%%%%%%%%%%%%%%%%%%%%%%%%%%%%%%%%%%%%%
    \chapter*{General conclusion}
    \addcontentsline{toc}{chapter}{General conclusion}
    \lhead{}
    \cfoot{\bfseries \thepage}
    \rhead{General conclusion}
    \markboth{droite}{General conclusion}

    .............................................

    %%References%%%%%%%%%%%%%%%%%%%%%%%%%%%%%%%%%%%%%%%%%%%%%%%%%%%%%%%%%%%%%%%%%%%%%%%%%%%%%%%%%%%%%%%%%%%%%%%%%%%%%%%%%%%%%%%%%%%%%%%%%%%%%%
    \clearpage
    \pagestyle{fancy}
    \addcontentsline{toc}{chapter}{Bibliography}
    \begin{thebibliography}{99}
        \lhead{}
        \cfoot{\bfseries \thepage}
        \rhead{Bibliography}

        \bibitem[1]{1}
        H. Altaama,
        Application Mobile Guide,
        Mémoire de Master en Informatique,
        Université Abou Bakr Belkaid de Tlemcen,
        2016.
        \bibitem[2]{2}
        \url{http://generationmobiles.net/2014/11/les-differents-types-dapps-mobiles/}, [consulté le 14/03/2018].

    \end{thebibliography}

\end{document}

    %%References%%%%%%%%%%%%%%%%%%%%%%%%%%%%%%%%%%%%%%%%%%%%%%%%%%%%%%%%%%%%%%%%%%%%%%%%%%%%%%%%%%%%%%%%%%%%%%%%%%%%%%%%%%%%%%%%%%%%%%%%%%%%%%
    \clearpage
    \pagestyle{fancy}
    \addcontentsline{toc}{chapter}{Bibliography}
    \begin{thebibliography}{99}
        \lhead{}
        \cfoot{\bfseries \thepage}
        \rhead{Bibliography}
        \bibitem[1]{1.1} 1995-2007 I.T. Young, J.J. Gerbrands and L.J. van Vliet.Fundamentals of Image Processing,Delft University of Technology
        \bibitem[2]{1.2} \url{https://northcoastphoto.com/digital-explained/#:~:text=A%3A\%20In\%20digital\%20imaging\%2C\%20a,dimensional\%20grid\%2C\%20represented\%20using\%20squares}
        \bibitem[3]{1.3} \url{https://www.cs.auckland.ac.nz/courses/compsci773s1c/lectures/ImageProcessing-html/topic3.htm}
        \bibitem[4]{1.4} Digital image basics Written by Jonathan Sachs
        \bibitem[5]{1.5} \url{https://homepages.inf.ed.ac.uk/rbf/HIPR2/colimage.htm}
        \bibitem[6]{1.6} \url{https://www.dynamsoft.com/blog/insights/image-processing-101-whats-an-image/}
        \bibitem[7]{1.7} \url{https://www.dynamsoft.com/blog/insights/image-processing/image-processing-101-color-space-conversion/}
        \bibitem[8]{1.8} Digital Image Processing.Lec.(3). 4 Th class
        \bibitem[9]{1.9} CYH/Image Enhancement /P.5
        \bibitem[10]{1.10} CYH/Image Enhancement /P.7
        \bibitem[11]{1.11} \url{https://theailearner.com/2019/01/30/contrast-stretching/}
        \bibitem[12]{1.12} CYH/Image Enhancement /P.9
        \bibitem[13]{1.13} International Journal of Computer Technology and Electronics Engineering (IJCTEE)Volume 1, Issue 2 / A Comprehensive Review of ImageEnhancement Techniques /H. K. Sawant, MahentraDeore
        \bibitem[14]{1.14} \url{http://matlab.izmiran.ru/help/toolbox/images/enhanc19.html}
        \bibitem[15]{1.15} \url{https://automaticaddison.com/difference-between-histogram-equalization-and-histogram-matching/}
        \bibitem[16]{1.16} K.K. Singh, and A. Singh., “A Study of Image Segmentation Algorithms for Different Types of Images,” International Journal of Computer Science, Vol. 7, Issue 5, Sept. 2010.
        \bibitem[17]{1.17} Amandeep, A., and Anju, Gupta., “Simulink Model Based Image Segmentation,” International Journal of Advanced Research in Computer Science and Software Engineering, Vol. 2, Issue 6, June2012.
        \bibitem[18]{1.18} Image Segmentation Algorithms Overview / Song Yuheng1, Yan Hao1
        \bibitem[19]{2.1} M. Kass, A. Witkin, and D. Terzopoulos. Snakes: Active contour models. International Journal of Computer Vision, 1(4):321{331, 1988}
        \bibitem[20]{2.2} R.J. Hemalatha, T.R. Thamizhvani, A. Josephin Arockia Dhivya, Josline Elsa Joseph, Bincy Babu and R. Chandrasekaran .Active Contour Based Segmentation Techniques for Medical Image Analysis.
        \bibitem[21]{2.3} \url{https://sg.inflibnet.ac.in/bitstream/10603/40751/9/09_chapter204.pdf}
        \bibitem[22]{2.4} Nikolas Petteri Tiilikainen  .A Comparative Study of Active Contour Snakes
        \bibitem[23]{2.5} Finite-element methods for active contour models and balloons for 2-D and 3-D images,” IEEE Trans. Pattern Anal. Machine Intell., vol. 15, pp. 1131-1147, Nov. 1993
        \bibitem[24]{2.6} A. Amini, T. Weymouth, and R. Jain. Using dynamic programming for solving variationalproblems in vision. IEEE Transactions on Pattern Analysis and Machine Intelligence, 12(9):855{867, September 1990}
        \bibitem[25]{2.7} D. J. Williams and M. Shah. A fast algorithm for active contours and curvature estimation. CVGIP: Image Understand ., 55(1):14{26, 1992..}
        \bibitem[26]{2.8} C. Xu and J. L. Prince. Snakes, shapes, and gradient vector flow. IEEE Transactions on Image Processing, vol 7 no .3 pp . 359-369 1998 ,
        \bibitem[27]{2.9} Alejandro Cartas-Ayala, s1056541 .Gradient Vector Flow Snakes.
        \bibitem[28]{3.1} \url{https://en.wikipedia.org/wiki/Convex_hull}
        \bibitem[29]{3.2} \url{https://iq.opengenus.org/quick-hull-convex-hull/}
        \bibitem[30]{3.3} \url{https://www.geeksforgeeks.org/quickhull-algorithm-convex-hull/}

    \end{thebibliography}

\end{document}