\pagestyle{fancy}
\chapter{Generalities on image processing}\label{ch:generalities-on-image-processing}
\chead{}
\lhead{\bfseries \chaptername {\,} \thechapter }
\cfoot{\bfseries \thepage}
\rhead{}
\rhead{\bfseries Generalities on image processing}

\section{Introduction}\label{sec:introduction-ch1}
\hspace{1cm}Images are one of the most important solutions that man has used to transmit and deliver knowledge and information since the dawn of humanity, in which a single image can gather a huge amount of information, Understanding the image and extracting information from the image to accomplish some works is an important area of application in digital image technology.
Modern digital technology has made it possible to manipulate images with systems that range from simple digital circuits to advanced parallel computers. The goal of this manipulation can be divided into three categories:
\begin{itemize}
        \item Image Processing\hspace{2cm} image in $\rightarrow$ image out.
        \item Image Analysis\hspace{2.4cm} image in $\rightarrow$ measurements out.
        \item Image Understanding\hspace{1.3cm} image in $\rightarrow$ high-level description out.
\end{itemize}
\hspace{1cm}In this chapter, we will introduce some general concepts in the field of image processing by setting some definitions and different methods used in image processing and analyses image (image segmentation).


\section{Digital Image}\label{sec:Digital-Image}
\subsection{Definition}
\hspace{1cm}A digital image a [m, n] described in a 2D discrete space is derived from an analog image a(x,y) in a 2D continuous space through a sampling process that is frequently referred to as digitization.
2D continuous image a(x,y) is divided into N rows and M columns.
The intersection of a row and a column is termed a pixel.
%The value assigned to the integer coordinates [m,n] with {m=0,1,2,\cdot \cdot \cdot,M\neg1} and {n=0,1,2,\cdot \cdot \cdot,N\neg1} is a[m,n].
In fact, in most cases a(x,y) which we might consider to be the physical signal that impinges on the face of a 2D sensor is actually a function of many variables including depth (z), colour (), and time (t).

\section{Image segmentation}\label{sec:image-segmentation}
        ................................


\section{Conclusion}\label{sec:conclusion-ch1}
        ................................
